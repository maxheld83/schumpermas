%!preamble
%\includeonly{./tex/how2.tex,./tex/how2.tex}

\documentclass[11pt,a4paper,oneside,openright]{article}
\usepackage{./tex/mystyle}
\hypersetup{pdftitle=(ECW), pdfauthor=(Maximilian Held), pdfcreator=(Maximilian Held), pdfsubject=(Essay), colorlinks=true, linkcolor=maxgreen, citecolor=black, filecolor=maxgreen, urlcolor=maxgreen}

\title{\emph{You Can't Run on One Leg}\\ ---\\ Negative Integration Amputates the Mixed Economy, Ails (CEEC) Welfare States}
\author{\href{http://www.maxheld.de}{Maximilian Held}}
\date{November 1, 2012}

\makeglossaries

\begin{document}
%!TEX root=../tax-democracy-held.tex
\newacronym{ca}
	{CA}
	{British Columbia Citizens' Assembly on Electoral Reform}

\newacronym{big}
	{BIG}
	{Basic Income Guarantee}

\newacronym{bigsss}
	{BIGSSS}
	{Bremen International Graduate School of Social Sciences}

\newacronym[longplural={Balances of Payments}]{bop}
	{BoP}
	{Balance of Payments}

\newacronym{cafe}
	{cafe}
	{Corporate Average Fuel Economy}

\newacronym[longplural={Central and Eastern European Countries}]{ceec}
	{CEEC}
	{Central and Eastern European Country}

\newacronym[longplural={Cost-Benefit Analyses}]{cba}
	{cba}
	{Cost-Benefit Analysis}

\newacronym{C}
	{C}
	{Consumption}

\newacronym[longplural={Corporate Income Taxes}]{cit}
	{CIT}
	{\hyperref[sec:CIT]{Corporate Income Tax} (p.~\pageref{sec:CIT})}

\newacronym{cpr}
	{CPR}
	{Common-Pool Resource}

\newacronym[longplural={Coordinated Market Economies}]{cme}
	{CME}
	{Coordinated Market Economy}

\newacronym[longplural={Dual Personal Income Taxes}]{2-pit}
	{Dual PIT}
	{\hyperref[sec:Dual-PIT]{Dual Personal Income Tax} (p.~\pageref{sec:Dual-PIT})}

\newacronym{dqi}
	{DQI}
	{Discourse Quality Index}

\newacronym{dna}
	{DNA}
	{deoxyribonucleic acid}

\newacronym{dp}
	{DP\textregistered}
	{Deliberative Poll \textregistered}

\newacronym[longplural={Deadweight-Losses}]{dwl}
	{DWL}
	{Deadweight-Loss}

\newacronym{d}
	{d}
	{Depreciation}

\newacronym{ec}
	{EC}
	{European Commission}

\newacronym{ecb}
	{ECB}
	{European Central Bank}

\newacronym{ecotax}
	{Ecotax}
	{\hyperref[sec:Ecotax]{Quantity Taxation of Energy} (p.~\pageref{sec:Ecotax})}

\newacronym{eu}
	{eu}
	{European Union}

\newacronym{efc}
	{EFC}
	{European Fiscal Compact}

\newacronym{emu}
	{EPL}
	{European Monetary Union}

\newacronym{epl}
	{EPL}
	{Employment Protection Legislation}

\newacronym{esm}
	{ESM}
	{European Stability Mechanism}

\newacronym[longplural={Expenditure Taxes}]{et}
	{ET}
	{\hyperref[sec:ET]{Expenditure Tax} (p.~\pageref{sec:ET})}

\newacronym{fce}
	{FCE}
	{Final Consumption Expenditure}

\newacronym{fdi}
	{FDI}
	{Foreign Direct Investment}

\newacronym{fpe}
	{FPE}
	{Factor Price Equalization}

\newacronym{fptp}
	{FPTP}
	{First-Past-the-Post}

\newacronym{frg}
	{FRG}
	{Federal Republic of Germany}

\newacronym{ftt}
	{FTT}
	{\hyperref[sec:FTT]{Financial Transaction Tax} (p.~\pageref{sec:FTT})}

\newacronym{G}
	{G}
	{Government Spending}

\newacronym{gdr}
	{GDR}
	{German Democratic Republic}

\newacronym{gdp}
	{GDP}
	{Gross Domestic Product}

\newacronym{gnp}
	{GNP}
	{Gross National Product}

\newacronym{gfcf}
	{GFCF}
	{Gross Fixed Capital Formation}

\newacronym[longplural={Head Taxes}]{ht}
	{HT}
	{Head Tax}

\newacronym{I}
	{I}
	{Investment}

\newacronym{ifi}
	{IFI}
	{International Financial Institution}

\newacronym[longplural={Information \& Communication Technologies}]{ict}
	{ICT}
	{Information \& Communication Technology}

\newacronym{isi}
	{ISI}
	{Import-Substitution-Industrialisation}

\newacronym{kjv}
	{KJV}
	{King James version}

\newacronym{lbt}
	{LBT}
	{\hyperref[sec:LBT]{Local Business Tax} (p.~\pageref{sec:LBT})}

\newacronym{ldc}
	{LDC}
	{Less Developed Country}

\newacronym{lme}
	{LME}
	{Liberal Market Economy}

\newacronym[longplural={Land Value Taxes}]{lvt}
	{LVT}
	{\hyperref[sec:LVT]{Land Value Tax} (p.~\pageref{sec:LVT})}

\newacronym{nfcf}
	{NFCF}
	{Net Fixed Capital Formation}

\newacronym{nhs}
	{NJS}
	{National Health Service}

\newacronym[longplural={Negative Income Taxes}]{nit}
	{NIT}
	{\hyperref[sec:NIT]{Negative Income Tax} (p.~\pageref{sec:NIT})}

\newacronym[longplural={New Trade Theories}]{ntt}
	{ntt}
	{\hyperref[itm:NTT]{New Trade Theory (p.~\pageref{itm:NTT})}}

\newacronym{nw}
	{NW}
	{Net Worth}

\newacronym{noma}
	{NOMA}
	{Non-Overlapping Magisteria}

\newacronym{mece}
	{MECE}
	{mutually exclusive and comprehensively exhaustive}

\newacronym{ms}
	{MS}
	{Member State}

\newacronym{oca}
	{OCA}
	{\hyperref[sec:OCA]{Optimal Currency Area (p.~\pageref{sec:OCA})}}

\newacronym{oecd}
	{OECD}
	{Organisation for Economic Co-Operation and Development}

\newacronym{omc}
	{OMC}
	{Open Method of Coordination}

\newacronym{omo}
	{OMO}
	{Open Market Operations}

\newacronym{osn}
	{OSN}
	{\hyperref[sec:OSN]{Ordinary Savings Norm (p.~\pageref{sec:OSN})}}

\newacronym{paygo}
	{PAYGO}
	{Pay-As-You-Go}

\newacronym[longplural={Payroll Taxes}]{payroll}
	{Payroll}
	{\hyperref[sec:Payroll]{Payroll Tax (p.~\pageref{sec:Payroll})}}

\newacronym[longplural={Prisoner's Dilemma}]{pd}
	{PD}
	{Prisoner's Dilemma}

\newacronym[longplural={Progressive Consumption Taxes}]{pct}
	{PCT}
	{\hyperref[sec:PCT]{Progressive Consumption Tax (p.~\pageref{sec:PCT})}}

\newacronym[longplural={Personal Income Taxes}]{pit}
	{PIT}
	{\hyperref[sec:PIT]{Personal Income Tax} (p.~\pageref{sec:PIT})}

\newacronym{ppaca}
	{PPACA}
	{Patient Protection and Affordable Care Act, popularly known as ``Obamacare''}

\newacronym{ppf}
	{PPF}
	{Production Possibility Frontier}

\newacronym{ppp}
	{PPP}
	{Purchasing Power Parities}

\newacronym[longplural={Property Taxes}]{pt}
	{PT}
	{\hyperref[sec:PT]{Property Tax} (p.~\pageref{sec:PT})}

\newacronym{qe}
	{QE}
	{Quantitative Easing}

\newacronym{rbct}
	{RBCT}
	{Real Business Cycle Theory}

\newacronym{ret}
	{RET}
	{Rational Expectations Theory}

\newacronym{rnd}
	{R\&D}
	{Research \& Development}

\newacronym[longplural={Stamp Duties}]{sd}
	{SD}
	{\hyperref[sec:WT]{Stamp Duty} (p.~\pageref{sec:WT})}

\newacronym{sgp}
	{SGP}
	{Stability and Growth Pact}

\newacronym{sic}
	{SIC}
	{Social Insurance Contributions}

\newacronym{sme}
	{SME}
	{Small and Medium-sized Enterprise}

\newacronym{stv}
	{STV}
	{Single Transferable Vote}

\newacronym{swf}
	{SWF}
	{Sovereign Wealth Fund}

\newacronym{tax-lse}
	{Tax LSE}
	{Tax Liability Side Equivalence, equivalently known as \emph{invariance of incidence proposition} or \emph{Dalton's Law}}

\newacronym{tfr}
	{TFR}
	{Total Fertility Rate}
	% the average number of children a woman would have by age 50 based on the current age-specific fertility rates

\newacronym{tfp}
	{TFP}
	{Total Factor Productivity}

\newacronym[longplural={Value-Added Taxes}]{vat}
	{VAT}
	{\hyperref[sec:VAT]{Value-Added Tax} (p.~\pageref{sec:VAT})}

\newacronym{vnm}
	{vNM}
	{\hyperref[sec:rational] {von-Neumann-Morgenstern} (p.~\pageref{sec:rational})}

\newacronym[longplural={Wealth Taxes}]{wt}
	{WT}
	{\hyperref[sec:WT]{Wealth Tax} (p.~\pageref{sec:WT})}

\newacronym{Y}
	{Y}
	{Output}

\newacronym{y2c}
	{Y2C}
	{\hyperref[sec:Y2C]{yield-to-capital (p.~\pageref{sec:Y2C})}}
\input{./tex/glossaries}

\makeglossaries

%!frontmatter
\maketitle
\thispagestyle{empty}
	\begin{center}	\vspace{15pt}
		{\large Term Paper}\\ 	\vspace{20pt}
%		{\large Drafts}\\ 	\vspace{20pt}
\begin{tabular*}{0.35\textwidth}{@{\extracolsep{0cm}}rl}
	{\large Student ID:}	&{\large 089145}\vspace{10pt}\\
	{\large Class:}		&{\large States and Markets}\\\vspace{10pt}\\
	{\large Instructor:}	&{\large Dr. Stormy-Annika Mildner}\\\vspace{40pt}\\ 
%					&{\large Prof. Dr. Claus Offe}\vspace{40pt}\\ 
\end{tabular*}
\end{center}

\newpage

\begin{abstract} %test
%english version
	Welfare states are best understood as mixed economies, where free market exchange is supplemented by planned state command in the service of equity, efficiency, or both. 
	In well-designed mixed economies, market and plan co-exist with minimal mutual distortions, and democratic sovereigns can trade off efficiency and equity at relatively small marginal cost. 
	Intact mixed economies rely on a set of regulatory, monetary and fiscal tools that operate on the same scale as markets (Chapter \ref{sec:mixed_economy}).
	
	In negative integration, markets expand to larger scale, but states remain organized at lower levels, crippling the command tools of the mixed economy. 
	As a result, democratic sovereigns can no longer take primacy over the economy and any remaining welfare states will be inefficient, inequitable, unsustainable or all.
	
	European integration is negative integration, and much of the 2010ff Euro-crises and the demise of European welfare states can be fruitfully analyzed as defunct mixed economies. 
	%!versions: this is for Offe only.
	\gls{CEEC}, in particular, followed a bad route to liberalization and built largely dysfunctional welfare states, because they did not, and could not, engage the trade-offs and contradictions of mixed economies in the context of negative integration (Chapter \ref{sec:EU_reality}).
	
	Some of the existing literature on (\gls{CEEC}) welfare and retrenchment fails to acknowledge the true constraints and alternatives of a mixed economy, and thereby fails to criticize and explain the societal and political choice of negative integration (Chapter \ref{sec:who_dunnit}).
	
	The current acquis threatens welfare, and, ultimately democracy and regional integration. If the \gls{EU} is to succeed, more economic integration must again always beget more political integration (Chapter \ref{sec:growth_solidarity}).
\end{abstract}

\newpage

%!TEX root=../tax-democracy-held.tex

\section*{How To \ldots} \label{chap:how2}

\paragraph{How to Use This Document.}
I have highlighted some words in \href{chap:how2}{green} to point to further information in footnotes, elsewhere in the document or on the internet. 
In the digital version of this document, these highlights hyperlink to their respective place of finding.

I have also hyperlinked in-text citations to their entry in the bibliography.

Many of the e-books I have read do not include the original print-edition page numbers. 
I reference these e-books with their (proprietary) \emph{Amazon Kindle \circledR} reading locations, identified as ``Kxxx'', such as in \citet[K50]{McCaffery2002}.
You can:
\begin{enumerate}
	\item 
		Find the original source on \href{http://kindle.amazon.com/profile/Maximilian-Held/1675396}{my \emph{Kindle \circledR} account} at\\ \url{http://kindle.amazon.com/profile/Maximilian-Held/1675396}.
	\item 
		Find the print-edition page number by typing the respective quote into \url{http://books.google.com}.
	\item
		Contact me to find the original source.
\end{enumerate}

\paragraph{How Not to Use This Document.} 
This is an early draft of unpublished work that has not been reviewed. Please do not cite or circulate any of this.
		
\paragraph{Correspondence.} 
Write to \href{mailto:maximilian.held83@gmail.com}{maximilian.held83@gmail.com} or see \\* \url{http://maxheld.de/contact} for more information.


\newpage

\tableofcontents

\listoffigures
\listoftables

%!mainmatter

\newpage

\begin{quote}
	\emph{``The Secretary, after mature reflection on this point, entertains a full conviction, that an assumption of the debts of the particular states by the union, and a like provision for them, as for those of the union, will be a measure of sound policy and substantial justice''\\}
	--- Alexander Hamilton, first US treasury secretary, 14 January 1790
\end{quote}

\newpage

\section{First-Order Questions First}%Better Title?

\paragraph{First-Order Questions} ask what is good (normative) and what is true (positive). 
Social science asks, for example, what makes a political system democratic (political theory from \citeauthor{Aristoteles} to \citeauthor{Dahl-1989-aa}) or what makes an economy rich (economics from \citeauthor{Smith-1776-lq} to \citeauthor{Hicks1939}).

\paragraph{Second-Order Questions} ask who or what decides first-order questions. 
Social science asks, for example, why and how welfare states --- a first-order \emph{answer} --- evolve(d): 
	because modernization replaced inherited, familial status with citizenship and the market (e.g. \citeauthor{Titmuss1974}; \citeauthor{Marshall-1950-aa}), 
	because industrialization required an appeased, reliable and healthy workforce (e.g. \citeauthor{Flora1981}, \citeauthor{Wilensky1975}), 
	because workers gained power (\citeauthor{Korpi1983} \& Palme; \citeauthor{Jessop2002}), 
	because institutions prevail (e.g. \citeauthor{Rothstein}), 
	because ideas matter (e.g. \citeauthor{Stiller2009}) because initial class cleavages lead to different degrees of commodifications (\citeauthor{Esping-Andersen-1990-aa}) or 
	because capitalism comes in variants (\citeauthor{HallSoskice-2001-aa}) (for a good overview, see \citealt{Beland2008}).

And so it is with the second-order theories of \emph{social conflict} of building welfare states in \gls{CEEC}. 
Competing theories suggest that elites played a role \citep{Aidukaite2006}, 
	that there are altogether new regimes in the East \citep{Cerami2006} or 
	that they were just ``muddling through'' (\citealt{Kovasc} as cited in \citealt{Fuchs2008}: 28), 
	that \gls{CEEC} welfare is weak because labor is \citep{Crowley2002}, 
	that transnational actors mattered \citep{Orenstein2009}, or not so much \citep{Sissenich2005}, 
	that mobilization was key \citep{Vanhuysse2006a}, 
	that pre-socialist institutions endured \citep{Inglot2008} or that norms did the trick \citep{Schimmelfennig2001}. 
Extensive narrative accounts chronicle the manifold changes since 1990 (e.g. \citealt{Deacon1992} ed.). 

These are variations on the questions of sociology: 
what binds us together (social integration), and what keeps us apart (social inequality
\footnote{
	The first question of sociology, according to \citeauthor{Dahrendorf1966} (\citeyear{Dahrendorf1966}: 66).
}) 
These are also the questions of political science: 
How do power, norms, ideas and institutions rule human interactions? 
And important questions, they are, asking us, that lone ``hypercultural'' species \citep{Henrich2004}, how we make our own history. 

In our rich time and in our unequal place, welfare may be \emph{the} historical battleground, on which these sociological and political forces operate. 
%maybe add a section on BIGSSS field text here, but only for diss, also present this in the colloquium.

\paragraph{First Order Theory \emph{First}.}
But to answer the question of who or what makes the welfare state in \gls{CEEC} or elsewhere, we must first know if there are alternative designs, and if so, what they are. 
Alternative designs must be \emph{materially possible}, and at least somewhat \emph{normatively desirable}.

If there was only \emph{one} materially possible welfare state design, there would be no social conflict to be explained, much like there is no need for a sociological theory of gravity. 

If there was only \emph{one} at least somewhat desirable welfare state design, there would be no social conflict to be explained, much like there is no need for a political science of wearing sunscreen.

However, if there \emph{are} several possible and desirable designs, social science needs to explain the \emph{absence} of all but the presently observed design. 
Any second-order theory of social change must be able to explain how the social conflict under study resulted in the non-occurrence of these alternative designs, especially the attractive ones. 

These second-order questions can be entirely positive inquiries.

The first-order question, however, is not just a positive one, but also a normative one: 
what possible, desirable welfare state designs are there? 
It is also not primarily a sociological or political science question, but an \emph{economic} question concerned with the production and distribution of scarce, material resources\footnote
	{\ldots and maybe the psychological and anthropological question about human frailties and capacities. More on this in the assumptions about \hyperref[it:homo_economicus]{homo economicus} (p. \pageref{it:homo_economicus}.)}.

And so, we must ask first: 
what \emph{can}, \emph{could} and \emph{should} welfare states do. 

Six disclaimers apply to my tentative answer on this very big question:\phantomsection \label{sec:disclaimers}

\begin{enumerate}
	\item \label{it:not_original} \emph{Not Original}. 
		The perspective I take here is hardly original. 
		Many others have, in greater width \citep{Stiglitz2002} or depth \citep{Sinn2004}, with narrower \citep{Scharpf1997} or different foci \citep{Zurn-2000-aa} discussed the first-order shortcomings of regional integration in the \gls{EU}, and economic liberalization everywhere \citep{Stiglitz2002}. 
		I aim here to reasonably comprehensively review the works of others and to restate some fairly conventional economic concepts in order to build a first-order checklist of welfare state design. 
		I then discuss how much of the \hyperref[sec:Literature]{current literature lacks awareness} of alternative, possible and desirable welfare states and why that matters (p. \pageref{sec:Literature}).
	%note that this can still be original for a PhD, says alex
	
	\item \label{it:no_test} \emph{No Positive Test}. 
		I cannot myself provide the methodological rigor or econometric data, to test the first-order theories of welfare state design, but rely on the mainstream literature instead. 
		The economics of the welfare state are vastly complex, incompletely understood and any policy initiative requires careful (empirical) investigation to balance the often contradictory imperatives of economic policy. 
	
	\item \label{it:no_calibration} \emph{No Calibration}. 
		I offer no calibration of the mixed economy and its institutions, and, for the purpose of this paper, advocate no \emph{particular} balance between market and state, efficiency and equity or any of the other tradeoffs a mixed economy may face. 
		Instead, I highlight the capacities and dysfunctions of markets and \hyperref[sec:ends]{list the state institutions} \emph{potentially} able to mitigate these shortcomings (p. \pageref{sec:ends}). 
		I consider under \hyperref[sec:means]{which conditions these institutions can work} (p. \pageref{sec:means}) and \hyperref[sec:defunct]{check} whether they are present and able under \gls{EU} economic integration (p. \pageref{sec:defunct}).
	
	\item \label{it:little_macroeconomics} \emph{Little Macroeconomics.} 
		I limit this discussion to very basic concepts of the real economy, and ignore many of the more complex models, especially of finance and money. 
		Modern macroeconomics, including such powerful frameworks as the IS/LM model are important (originally \citealt{Hicks1937}), but would go beyond the already lengthy treatment here. 

	By limiting the discussion to a few rudimentary, but deeply understood concepts of the real economy, I also hope to reconnect regional integration and the welfare state to an \emph{econonomic imagination} (paraphrasing \citealt{Mills-1959-aa}) of our material affairs as a household --- only with a cast of billions\footnote
		{\ldots as the etymology of economics would suggest: 
		the science of managing the \emph{oikos}, greek for household.}. 
	Inevitably, much of the detail and complexity that policy makers have to consider, will fall by the wayside. 

	\item \label{it:little_europeanese} \emph{Little Europeanese}. 
		Likewise, I will give \gls{EU} and \glspl{IFI}  only a very cursory treatment. 
		I hope to show that what defines the \gls{EU} --- and certainly the Washington consensus --- is the \emph{absence} of most mixed economy institutions. 
		From that vantage point, the actual design of present, embryonic \gls{EU} institutions is fairly inconsequential.

	\item \label{it:homo_economicus} \emph{Pragmatic Institutions for Homo Economicus.} 
		Any prescription for how to organize production and distribution rests on a model of human nature. 
		For evolutionary anthropology, psychology, behavioral economics and other offspring of now disreputable \citep{Wright1994} sociobiology \citep{Wilson1975} this is an empirical question about our evolutionary baggage
			\footnote{
				Because I sometimes refer to such Neo-Darwinian arguments \citep{Wright1994} and hear them often misunderstood I must reiterate the ontological status of evolution:
				\begin{enumerate}
					\item Evolution disposes us to think and act in certain ways, but does not determine us to do so (\emph{deterministic fallacy}).
					\item Evolution yields (more or less) stable \emph{equilibria} between environmental conditions and (more or less) adaptive traits of organisms. 
					It does not necessarily yield \emph{optimal} configurations, just survival of the relatively fitt\emph{er}. 
					\item Evolution is an aimless process. 
						There is no reason to praise \emph{(naturalistic fallacy)} or criticize (\emph{moralistic fallacy}) evolutionary results on any moral basis.
		\end{enumerate} 
			}.
	
	We now know that we are neither a blank slate
		\footnote{
			\ldots \emph{aka.} \emph{tabula rasa} or \emph{homo sociologicus}, as \cite{Dahrendorf1965} quipped.
		} 
	for behaviorism to condition or society to write on, nor an essentially determined animal, but somewhere \emph{between} nature and nurture, as genes and the capacity for memes \citep{Dawkins1976} co-evolved to make us the ``hypercultural species`` that we are  \citep[loc 175]{Henrich2007}
	\footnote
		{Relatively ill-equipped in instincts, we need to \emph{learn} (or \emph{imitate}) what to eat and hunt --- and how to build a blast furnace. 
		This is the evolutionary blockbuster of humans: 
		we moved the locus of our evolutionary adaptation from genes to memes \citep{Dawkins1976}. 
		Instead of ``hard-coding'' all adaptive traits, we learn (or imitate) the more complex and more malleable software of culture (\citealt{Boyd1985}, \citealt{Henrich2007}: loc 196ff). 
		Our nature (genes) and culture (memes) co-evolved: 
		As our brain allowed us to learn easily, our culture developed cooking, and our digestive tract adapted to broken-down proteins (\emph{ibid.})
	}. 
	Perplexingly, it is in our \emph{nature} to rely on \emph{culture}  --- distilled in somewhat binding or accepted institutions --- to guide us in our production. 
	As enlightened and reflexive animals, we now \emph{consciously} add to our set of memes: 
	
	we can build or break institutions, as an act of will. 
	
	But still, we may not be able to make arbitrary institutions, with no regard to our evolved nature: 
	not all our acts are willful, let alone reflexive. 
	What we \emph{can} do, at least since enlightenment dawned, is build progressive institutions: 
	they help us achieve desired ends by first responding to our innate behavioral, cognitive and emotional dispositions, then transcending them
		\footnote{
			Turning reflexive on our innate features may be best emancipatory gesture to make, knowing that evolution --- and, incidentally, markets --- are aimless processes. 
			As mere containers of ``Selfish Genes'' \citep{Dawkins1976} our biological existence is meaningless and humble, as \citeauthor{Bryson2003} reminds us: 
			``Life wants to be, but it doesn't want to be much'' \citeyearpar{Bryson2003}.
		}.
	
	State and market are the essential institutional innovations of modernity. 
	They hyper-charged functional differentiation to reach near-planetary breadth (e.g. international trade, citizenship) and near-universal depth (e.g. commodification, family law), altering or destroying much of the evolved culture (e.g. kin) and institutions (e.g. tribe) adaptively fit only at smaller scale \citep{Diamond1997}. 
	
	Both state and market rely on a \emph{homo economicus} model of human nature. 
	The state gets an amoral, but opportunistic \emph{homini lupus} to behave by threatening monopolistic force (as in \citealt{Hobbes-1651-aa}). 
	The market, in turn, lures a self-seeking, but rational ``butcher, [...] brewer, or [...] baker'' into providing ``our dinner'' --- and all else --- ``from their regard to their own interest'' \citep{Smith-1776-lq}.

	As the empirics go, this rational, atomistic, utility-maximizing model of human nature might be incomplete --- but not entirely incorrect --- on all counts
	\footnote{
		Humans may be hard-wired altruists (e.g. \citealt{Zak2004}), are only boundedly rational \citep{Simon-1999-aa,Kahneman2011}, poor planners of utility (recent summary in \citealt{Gilbert2006}), think in relative, not absolute terms \citep{Frank2005} and display diminishing marginal utility \citep{Ng-1997-aa,Veenhoven-2000-aa,Nickell2008}.
	}. 
	Nevertheless, I suggest, for now and for much of this paper, we take homo economicus as a starting point, for five pragmatic reasons:%this works for the EU paper, but will need to be changed for other sections.
	
	
\begin{figure}[htbp]
	\centering
	\includegraphics[width=1\linewidth]{./img/modes_human_nature}  
	\caption{Modes of Production, Distribution and Human Nature}
	\label{fig:modes_human_nature}
\end{figure} 

	\begin{enumerate} 
		\item Market and the state are not the only means to organize cooperation, but they are the only institutions that have demonstrably orchestrated large-scale production and distribution of many kinds of goods. 
		For now, only states can solve commons problems (e.g. \citealt{Hardin-1968-aa}), and only markets efficiently and credibly gather, process and signal dispersed information (\citealt{Hayek1931}). 
		Other institutions to facilitate cooperation, such as kinship \citep{Van-den-Berghe-1981-aa,Hammond2006}, the nuclear family (on which the conservative/continental welfare state still relies heavily, according to \citealt{Esping-Andersen-1990-aa}) or community \citep{Ostrom1990} are often narrow in scope and reach. 
		Eventually, great hopes set in volunteerism, a communal ``governing the commons'' \citep{Ostrom1990} or some other alternative may come to fruition
		\footnote{
			Self-organizing scientists (e.g. The Human Genome Project), programmers (e.g. Linux OS) and web-users (e.g. Wikipedia) have lately accomplished impressive achievements, but their mode of production seems to complement state and market, rather than replace it: 
			scientists are often paid state salaries, free software runs on commercial hardware, and wikipedians need day jobs. 
			These goods, incidentally, are also all common or public goods, the non-state production of which we are only just beginning to understand \citep{Ostrom1990}.
		}. 
		In the meantime, we should stick to the institutions that we know to work, and know how they work (and fail)
			\footnote{
				Which is more than can be said about suggested remedial institutions such as ``governance'', ``Big Society'' (Cameron, 2011) or a philanthropic ``Third Sector'' \citep{Anheier2002}, all of which are shrouded in impenetrable \hyperref[sec:newspeak]{Newspeak} (``problem solving'', ``community'' and ``giving back'', respectively), as I argue later (p. \pageref{sec:newspeak}). 
				They lack a coherent model of human nature, and an account of their sucesses and failures. Civil society, in particular, is yet only negatively defined (it is \emph{not} the state, \emph{not} the market and \emph{not} the family), its mode of production (volunteerism?) is underspecified and its vaguely optimistic ignorance of structure and material interest border on (hegemonial?) ideology.
			}.
			
		\item State and market made their own humans: 
		as we live under hierarchy and competition, we adapt to it, for better of for worse (e.g. \citealt{Schwartz2010}
			\footnote{
				\cite{Schwartz2010} show how excessive regulation (the mode of states) and incentives (the mode of markets) can crowd out intrinsic motivation, ``Practical Wisdom'' and other, better angels of our nature. 
				Unfortunately, this ship has probably sailed. Still, their insistence on a broader, human capacity to ``do well by doing good'' and crusade against dehumanizing choice is important.}). 
		Good institutional design does not ask for a new man, but makes do with the women and men we have now, and helps them unfold their greater capacities. 
		In the \gls{OECD}-world and in our time, many people (including me) will expect and respond to incentives and regulation. 
		Consequently, market and state institutions have to reckon with homo economicus, even if --- and because --- it is partly of their own making.
		
		\item Even if, as I suggest, there are no alternatives to state or market in \emph{some} realms of modern society, we need not ``economicize'', commodify or regulate \emph{all} aspects of life. 
		At the economy or industry level, markets may be our best bet, but perhaps not at the firm or team level, where we can tap into other human motivations. 
		Similarly, a state may need to impose health and safety standards, but not a teacher's lesson plan (as \citeauthor{Schwartz2010} alarmingly report). 
		Market and state, along with their impoverished view of human nature, \emph{can} be applied selectively, to some domains. 
		As alternatives become available
			\footnote{
				For example, the \gls{BIG} is a courageous suggestion to, among other things, decommodify \emph{and}, because it is no longer means-tested, ``de-regulate'' the livelihood (for instance, \citealt{Offe2009}) and allow volunteerism.
			}, 
		market and state can recede. 
		For the time being, the domains exclusive to market and state will remain considerable. 
		
		\item I here try to formulate a prescription for the institutions of a mixed economy and compare it to actual \gls{EU} regional integration and \gls{CEEC} welfare states. 
		Not to utopically ``compare ideal oranges with actual apples'' (\citealt{Dahl-1989-aa}: 84), this prescription must be doable. 
		It is hard to decide just what is doable and what not. 
		A helpful analytic heuristic is to treat that as manipulable which is under study, holding everything else constant. 
		Under study here is the mixed economy and regional integration whose institutions I assume to be perfectly malleable. 
		\emph{Not} under study here are the mode of economic production, the condition of modern society or human nature
			\footnote{
				An anthropologically and psychologically informed discussion of the welfare state is direly needed, but I cannot provide it here. 
				We must know whether, and under which (institutional!) circumstances humans are ``knights, knaves, pawns or queens'' \citep{LeGrand2003}, what our capacity for altruism is \citep{Henrich2007} and how it can be fostered \citep{Axelrod1981a}. 
				Promising evidence suggests that we suffer from \emph{relative} inequality \citep{Pickett-2009-kx}, that we are hard-wired to cooperate \citep{Zak2011}, and that we are a deeply ``social animal'' \citep{Brooks2011}. 
				We might have a greater capacity to be \emph{homo reciprocans} than utility-maximizing, self-seeking, atomistic \emph{homo oeconomics} would have us.
			}.
		I assume these to be constant in the medium run (see prior point) and reasonably approximated by \hyperref[it:homo_economicus]{homo economicus} and \hyperref[sec:perfect_competition]{mainstream economics} and note in the following where I relax or question these assumptions (table \ref{tab:assumptions_failures})
		\footnote{
			None of this is to preclude a study or reform of economic production, modern society or human nature. 
			But the mixed economy rests on these assumptions, as much as it attempts to remedy their shortcomings.
		}. 
		
		\item \phantomsection \label{it:economize_moral} I take the below, often orthodox or even neoclassical positions not always out of conviction, but for their simplicity and to ``economize on moral disagreement'', as \citeauthor{GutmannThompson-2004-aa} advise us (\citeyear{GutmannThompson-2004-aa}: 7,  loc. 226). 
		I hope these positions will be widely acceptable to readers on the right and provisionally tolerable to those on the left. 
		If I can show \gls{EU} regional integration and \gls{CEEC} welfare states to be wasteful, unfair and unsustainable assuming even such liberal orthodoxy, sweeping reform should be all the more obvious.
		
		%\cite{Pfaller2011} llc:  172 notes that a Materialist Philosophie Not Idealismus are better 2 bring change exactly because they value this world as the only and best that we could have.
		
	\end{enumerate}
\end{enumerate}
%needs ueberleitung

I aim here to highlight in welfare state design and \gls{EU} regional integration what Edward \citeauthor{McCaffery2002} urges us to do about tax: 
``The devil may indeed dwell in the details, but we first need to find an angel or two in the abstractions that govern [\ldots]'' (\citeyear{McCaffery2002} loc. 117). 

I look for these angels in an \emph{idealized} closed, mixed economy. 
The account I provide in the\hyperref[sec:mixed_economy]{following pages} (p. \pageref{sec:mixed_economy} -- \pageref{sec:EU_reality}) does not resemble any real existing economy, where abstractions are often shrouded in historical idiosyncrasies, and angels rarely found amidst imperfect policies. 
But this is a question of the first order, and to know what is materially possible and normatively desirable we need reason, not reality.

Even without the details, the abstractions alone need considerable space to be explained. 
I urge readers to take the time, even if much will seem familiar and some things appear remote to welfare state design. 
They are not: 
from \hyperref[sec:adverse_selection]{adverse selection} (p. \pageref{sec:adverse_selection}) to \hyperref[sec:winner-take-all]{winner-take-all} network effects (p. \pageref{sec:winner-take-all}), \hyperref[sec:why_mixed_economy_matters]{it all matters} (p. \pageref{sec:why_mixed_economy_matters}). 
Missing any one of these abstractions, we cannot know what a welfare state can, and should do.

\newpage

	\input{./tex/mixed_economy}

\newpage

\section{The Crime Scene: The EU as Defunct Mixed Economy} \label{sec:EU_reality}
I now turn to the real world of european integration up to 2012 and compare it to the institutions of an ideal, closed mixed economy.
	
	\subsection{The EU Economy}
The \gls{EU} differs from the above stylized mixed economy in two key respects:
\begin{inparaenum}[1)] 
	\item it is deeply, but unequally \hyperref[sec:EU_Acquis]{integrated} (p. \pageref{sec:EU_Acquis}) and 
	\item very \hyperref[sec:sources_of_wealth]{heterogenous} (p. \pageref{sec:sources_of_wealth}).
\end{inparaenum}

\subsubsection[Integration]{Integration: The EU Acquis} \label{sec:EU_Acquis}
Internal boundaries in the \gls{EU}, at least nominally, are wide open \emph{open}. %add reference
Primary european legislation from the 1986 Single European Act to the 2009 Lisbon treaty grant free movement of goods, services, capital and people. 
Nominally at least, \gls{EU} member countries have completely open markets for factors and goods. 
Market production and distribution of many goods and services occurs at the union level.

\paragraph{Regulation} To a significant extent, the regulatory means of mixed economy are also provided at the union level. 
The \gls{EU} plays a dominant role in, among other things, consumer protection, standardization and competition policy. 
Potential regulation in other fields, including labor markets, is still in its infancy and largely under national jurisdiction. %citations needed.

Some observers also note that where the \gls{EU} has formal jurisdiction, it has also taken a laxer, \emph{deregulatory} stance and favored liberalization or even privatization, for example in infrastructure. %citation needed
In other fields, notably antitrust regulation, the \gls{EC} has been fairly aggressive. 
There appears to be no conclusive evidence that \gls{EU} regulation, where applicable, would be more laissez-faire than at the member state level. %citaiton needed 
Where the \gls{EU} has restricted member states from intervening in markets
	\footnote{
		For example, the \gls{EC} has criticized state (Land) equity in Volkswagen AG and public liability for savings banks in Germany.
	}, 
it has done so to level the playing field in the internal market. 
Any such discretionary interventions by only \emph{some} \gls{MS} governments, in only \emph{some} otherwise \gls{EU}-wide markets are indeed tantamount to barriers to trade --- in the case of Volkswagen --- or subsidies --- in the case of savings banks.  %does this paragraph belong in here?%citiations needed

While the \gls{EU} has not fully replaced national regulators in all sectors, it has made significant strides into some of the union-wide markets, such as passenger air transport. 
This may --- or may not --- suffice yet for a mixed economy, but it need not regulate at the highest level possible, but only at whichever level matches the effective mobility of factors and goods in question to avoid \hyperref[sec:regulatory]{regulatory arbitrage} (p. \pageref{sec:regulatory}). 
\emph{Subsidiarity} binds the \gls{EU} to organize regulation at the lowest level possible.

\paragraph{Money} A subset of \gls{EU} members have formed the \gls{EMU} and conferred full monetary authority to the union. 
The \gls{ECB} buys and sells \gls{EMU} bonds to enforce the targeted interest and inflation rates and commands all other means of monetary policy. 
Much like its German role model, it is constitutionally independent and devoted to price stability defined as at or below 2\% inflation. %citation needed

\paragraph{Fiscal} While the regulatory and monetary tools of a \gls{EU} mixed economy are well-developed, the union has no significant fiscal institutions to speak of. 

%There is very, very little tax coordination in europe with the exception of the meager, anyway dysfunctional, soft-law business tax code of conduct, cf. bratton

The budget accounts for a little more than 1\% of union \gls{GNP}, compared to 40-50\% in member states, of which almost half are agricultural subsidies. 
The roughly 30\% of the budget devoted to redistribution in the structural and cohesion funds roughly equal the budgets of two or three european metropolis. %citation needed

The \gls{EU} also does not levy its own taxes, but, except for negligible income from tariffs, depends on member state contributions (tied to their \gls{VAT} revenue and \gls{GNP}). 

The acquis does restrict the spending of \gls{MS}, for example through the \gls{SGP}, limiting annual deficits to 3\% and national to 60\% of \gls{GDP}
	\footnote{
		The 1997 \gls{SGP} as well as the no-bailout clause in the 1992 Maastricht Treaty (article 104b) are both meant to curb moral hazard resulting from the \gls{EMU}. 
		The link between monetary and fiscal means is further in \hyperref[sec:fiscal_CPR]{fiscal common-pool resource problem} (p. \pageref{sec:fiscal_CPR})
	}.%add source 
By contrast, the acquis features almost no public revenue cooperation. 
Aside from some very limited harmonization in (\gls{CIT}) tax bases, member states conduct their own taxation on always formally, and sometimes factually union-wide markets for factors and goods.

\subsubsection[Heterogeneity]{Heterogeneity: The Sources of Wealth} \label{sec:sources_of_wealth} The \gls{EU} is a union of very unequal member states. 
On of the richer large member state, France (\$ 44,747 nominal \gls{GDP} per capita according to the IMF 2010) is six times richer than the poorest member state, Bulgaria (\$ 6,334 nominal \gls{GDP} per capita, \emph{ibid.}.). 
Within \gls{EU}-15, Luxembourg's \gls{GDP} per capita in \gls{PPP} was 3,5 that of Portugal. 
Within \gls{EU}-27, that multiple has widened to 7,4 times Luxembourg and Bulgaria (\citealt{Alber2008}: 1). %add gini, other union level statistic?

Why this difference?

\paragraph{Productivity} The wealth of nations, as Adam \cite{Smith-1776-lq} taught, ultimately depends on labor productivity. 
Both nations and people prosper when they can get much out of the ultimately scarcest resource, human effort. 

This labor productivity, as \gls{GDP} per capita, varies widely between rich (e.g. France 116) and poor (e.g. Bulgaria 41.3, both in 2010, indexed to EU-25 average, Eurostat 2010)\footnote{
	Current labor productivities, as \gls{GDP} per capita probably understate the discreptancies. They are both after (much) regional integration. 
	Without trade and investment --- ex-ante EU membership --- domestic economic output might be much lower still.}. 
Labor productivity, in turn, depends on technology (\gls{TFP} or Solow residuals), human capital (skills) and stocks of physical capital (e.g. factories). 
These achievements, tangible or not, are all past surplus production coagulated into some form of capital. 
At some point in the past, someone had to have an extra bit of time or food on her hands to build something lasting, above and beyond subsistence consumption.

Why does it matter?

\begin{itemize}
	\item Because capital is transformed surplus production, it is always an inheritance from the past. 
	And so, much of current output depends on past capital accumulation and its historical circumstance in geography, geology, culture and political institutions. 
	It matters a great deal, which circumstances encourage saving and innovation, but that question need not concern us here. %add link saving

	However, we need to remember here, that current performance is \emph{not} sufficiently linked to current output. 
	Romanian workers at Dacia, a car manufacturer, produce less value in an hour primarily \emph{not} because they are lazier or sloppier, but because Romania lags in capital accumulation and cannot provide them with the skills and machinery that their French counterparts at Renault have. 

	\item Because capital is transformed surplus production, differences in output between people, countries and regions are here to stay, at least for some time.
\end{itemize}

%more stats are here: http://epp.eurostat.ec.europa.eu/tgm/table.do?tab=table&init=1&plugin=1&language=en&pcode=tsieb030, 

I now discuss some broad economic dynamics likely operating on this partially integrated, and enormously heterogeneous economy. 

%money is: medium of exchange, unit of account and store of value. It can be either commodity money (intrinsic value), or it can be fiat (paper, paypal, bitcoins). The money stock is very difficult to know because the definitions are vague.


\subsection[Trade]{Trade}

\begin{verse}
	\emph{``Red Square full of Burger Kings, Komsmolski kids on dope \\ 
	Gimme gimme vodka all the Russians sing abandon any hope''\\}
	--- The Busters (1997): \href{http://www.youtube.com/watch?v=xsxRMOnpMTY}{Mickey Mouse in Moscow}\\
\end{verse}

\gls{EU} member states trade freely with one another, in many ways similar to commerce within the closed economy. 
The same \hyperref[sec:space]{trade theories} may apply, but with hypercharged welfare and distributive dynamics (p. \pageref{sec:space}). 
		
\paragraph{Absolute \& Comparative Advantage.} 
\gls{EU} member states trade with one another based on different (labor) productivities: 
for any unit of inputs, Greek farmers will reap more sun-kissed olives than Swedish growers. 
Member states also trade with one another based on different opportunity costs, or \emph{relatively} different (labor) productivities: 
even though France might be just as good or better at making wine and wheat than Germany, the difference is greatest in wine, on which France will specialize. 

Absolute and comparative advantage are the most general formulations and gains from trade hold in more than these stylized cases based on climate and soils. 
In today's Europe, absolute and comparative advantages also stem from a particular regions specialty (e.g. City of London finance), an industry's superior technology (e.g. French diesel engine particle filters) or a member state's economic culture, history and institutions (e.g. German \gls{SME} engineering).

The gains from european trade are enormous, but so are its distributive effects and adjustment costs. 

Absent any meaningful fiscal institutions at the union level, the \gls{EU} cannot alter the distribution of the gains from trade. 
The fruits of economic integration will be shared according to relative productivities and terms of trade.

Trade on absolute and comparative advantage makes member states converge if and to the extent that the poorer countries can use the gains from trade to accumulate capital
	\footnote{
		Again, capital accumulation in the broadest terms, including technology and skills.
	}
\emph{faster} than the rich countries are. 
Rapid capital accumulation, in turn, will partly depend on the ability of poor \glspl{MS} to redistribute resources through taxation towards education, \gls{RnD} and investment.

Convergence under absolute and comparative advantage is an empirical question that I cannot address here, but it seems unlikely that poorer \gls{MS} will be able to accumulate capital much faster than the richer trading partners. 

In the short and medium run, specializing on absolute and comparative advantage will be disruptive, and expensive, especially as an \gls{EU}-25 becomes more heterogeneous and whole industries will have to be gutted, workers retrained and new sectors established. 

Such adjustment costs are relevant to both equity and efficiency. Within countries, they burden those factor owners (workers, capitalists) specific to uncompetitive industries. 
In the best of circumstances, they have to retrain or retool at a cost. 
Often, these losers from trade will have to accept lower wages, unemployment or diminished profits. 
If the economic costs to and political backlash from these losers from trade is too large, structural transformations progress slowly and diminish the overall gains from trade.

The \gls{EU}, lacking any significant fiscal institutions, has no way to ease the pain of adjustment. 
It, again, relies on \glspl{MS} to sidepay the losers in an otherwise Pareto-improving exchange. %is this pareto?

Structural transformation may also be slowed by rigid labor markets, including \gls{EPL} and powerful unions, if and to the extent they extract monopoly wages. 
Here, too, jurisdiction remains in the hands of \glspl{MS}. 
The \gls{EU} has very limited regulatory competence over labor markets. 
To the extent that production factors and/or goods and services are mobile at the union level, as is assumed under trade, regulatory arbitrage can ensue and national level regulation may be severely constrained or ineffective --- instead of democratically agreed-upon.

\paragraph{Factor Endowments.} 
\gls{EU} member states trade with one another based on different factor endowments. Germany --- relatively abundant in capital, but scarce in labor --- may specialize in capital-intensive machine engineering. 
Romania --- abundant in labor, but scarce in capital --- may specialize in labor-intensive electronics assembly. 
Both countries then exchange their surplus machines and electronics. 
Over time, predict Hekscher and Ohlin, Romanian labor and German capital will become ever scarcer, until factor prices are equal in both countries. %citation needed

Before factors prices equalize, however, they may diverge as trade commences. 
In Germany, newly expanding machine engineering may demand more capital, and in Romania, recently opened assembly lines may demand more labor. 
Conversely, German labor and Romanian capital may see diminished demand as industries specialize. 
Counterintuitively, the relatively \emph{abundant} factor in each country may initially command \emph{higher} prices. 

This may have troubling implications for convergence. 
In the poorer, and capital-scarce countries, precisely the route to growth --- more capital --- may seem less attractive. 
The conundrum also plays out, when unskilled and skilled labor are the two factors of production: 
physical capital here, as always, is only one of several possible transformations of surplus production. 

As the richer countries are presumably more abundant in skilled labor, the poorer countries will specialize in \emph{unskilled} labor. 
After opening up to trade, education will pay \emph{less}, as it did in Estonia, where the net wage premium for tertiary over primary education dropped from 93\% in 1997 to 32\% in 2006 (\citealt{Piatkowski2008}: 33).

%\begin{quote}
%	\emph{''It’s not ok to go to another country to work, because your country should offer you the possibility to work in your country, to have enough money to do all these things you need. But, they go because they want to assure a better life for their children, or to buy a house, to buy a car, I don’t know. Or they just go because they just want a better live. Our country is so poor.''}\\
%	-- Alex, 18, Romania (in Eurolektionen \citeyear{DeRuffray2010})
%\end{quote}

Trade on factor endowments also requires severe, costly and inequitable adjustment within the trading partners. 
As relative factor returns for capital and labor diverge upon trade, there will be winners and losers in both countries. 

Trade between people or regions with different factor endowments may always face \hyperref[sec:space]{these dynamics}, but ideal mixed economies have developed institutions to mitigate diverging factor returns (p. \pageref{sec:space}). 
For example, government can subsidize factor incomes, education or generally transfer capital to regions and people to let them leap-frog to higher value production. 
For example, (West) Germany strongly supported initially agrarian dominated Bavaria and Baden-Wurttemberg with transfer payments in the early post-war era. 
Since --- surely not simply \emph{because} of it --- the two southern states have developed a flourishing capital-intensive industry. 
Alternatively, government can postpone trade liberalization until it has accumulated enough capital to be competitive in capital-intensive industry. 
For example, emerging ``Asian Tiger'' economies in Southeast Asia have opened their borders relatively relatively late to nurture ``infant industries'' to resilience, and have deliberately substituted superior imports with inferior domestic production to jumpstart capital accumulation.

None of these mitigating policies are even available in the EU. 

The union cannot transfer between winners and losers, forerunners and laggards of capital accumulation because there is no union-level revenue to pay for it. 
The nominal structural and cohesion funds are again, nowhere near sizeable enough to make a dent in changing factor returns. 

Because trade is always liberalized abruptly, universally and reciprocally as new \gls{MS} join, they can also not shelter infant industries. 
Any such protection or subsidies would be in violation of the \gls{EU} acquis.

\paragraph{New Trade Theory.} 
\gls{EU} member states trade with one another to reap returns to scale that previous specialization or clustering brought. 
Already within the \gls{EU}-15, prior to the 2004ff Eastern Enlargement, large volumes were traded between countries of \emph{similar} factor endowments and \emph{little} comparative advantages. 
For example, Germany traded its automobile parts, produced in a densely clustered network of suppliers in the south-west with France, for its processed foods, produced at great scale and density in the north-west. As the common market expanded further after 2004, these industrial champions were able to supply even larger markets, at even greater scale and diversity. %add citation

If production indeed agglomerates, as \gls{NTT} suggests, this trade on scale and network effects may cause divergence, rather then convergence between \gls{EU} \gls{MS}, for at least two reasons:
\begin{enumerate}
	\item Agglomeration is self-reinforcing. Existing clusters and specialization will beget more clustering and specialization. 
	\item For network effects, distance and space matter. Economic activity may be increasingly concentrated in the center at the expense of the periphery.
\end{enumerate}

Production in closed, ideal mixed economies also tends to agglomerate, but these governments can set industrial and structural policies to mitigate concentration. 
Government can, for example, foster specialized clusters or subsidize nascent industries until they reach sufficient scale and scope. 
When a closed economy opens to trade slowly and deliberately, government can also initially shelter infant industries from competition.

Under the acquis, again, these policy responses are strictly limited. 
\gls{MS} governments cannot subsidize select industries, and they must open trade on all sectors.

\paragraph{Finance.} 
I have here discussed only trade, and not finance, even though flows of capital between \gls{MS} --- such as \glspl{FDI} --- probably make up more of the economic dynamic than flows of goods and services. 
Still, I concentrate on trade for four pragmatic reasons:
\begin{enumerate}
	\item International finance is a lot more \emph{complex}, both conceptually and empirically, than trade, and I cannot include it in this already lengthy treatment
		\footnote{
			For example, \cite{Narula2005} reminds us, that not all \glspl{FDI} are created equal: 
			some create substantial spillovers in complementary assets such as skills, or build clusters and others do not.
		}.
	\item International finance, in the final analysis, is \emph{equivalent} to international trade. 
	Broadly, whether you move the good or service --- as in trade --- or the factor to produce it --- here, international capital flows --- does not matter for the welfare and distributive effects of the interaction. 
	For example, whether a German IT firm \emph{outsources} electronics manufacturing to a foreign-owned firm in Romania, or \emph{offshores} the same job to its Romanian subsidiary will not alter the resulting trade flows. 
	Either way, the finished electronics have to be shipped back to Germany, or wherever customers want them. 
	In both cases, too, the reasons for moving goods, or goods \emph{and} factors will be the same: 
	electronics manufacturing enjoyed a comparative advantage in Romania.
	\item International finance, in the final analysis, is also \emph{epiphenomenal} to international trade. 
	Anytime money is invested in a foreign country, real things will follow in the same direction. 
	For example, if a German IT firm invests in its Romanian subsidiary and builds a factory, that transaction will, in some form and at some point in time, materialize into trade of some good or service from Germany to Romania. 
	Invested capital, here, as always, is coagulated surplus production, and money is just the conventional denomination of such surplus. 
	To invest into a new factory in Romania means for the German IT firm (or its owners) to transform their surplus into a something else, somewhere else. 
	Maybe, the IT firm owners will transform a partial ownership in a residential building somewhere in Germany (through a real estate fund) into a factory building in Romania. 
	As it moves from place to place, and transforms from residence to factory, this surplus eventually takes a material form that shows up as trade. 
	It may, for example, show up as German-grown potatoes that feed the Romanian workers while they build the factory --- even though that would be unlikely, given Germany's trade advantages. 
	More plausibly, it may show up as a German-built used car that moves from Berlin to Bucharest, as compensation for the construction workers efforts. 
	\emph{Which} of a multitude of material forms this exchange takes is fully explained by the dynamics of \emph{trade}, which always underly any international financial transaction. 
	Properly imagined as a ``household with a cast of billions'', as economics should, a German IT firm investing in a Romanian factory is, ultimately, a \emph{trade}, of, for example the service of building a factory for the good of a german-built used car. 
	All else is formal detail.
	
	Macro-level national balance of payments equations also reveal the correspondence of trade and financial transactions. 
	All other things equal --- such as remittances, factor incomes, or reserves --- any net import, including German-built used cars, must be balanced by a net inflow of capital, including investment in foreign-owned factories.
	
	\item Lastly, international financial transactions are \emph{inconsequential}. 
	In the final analysis, they do not bear on the welfare and distributive effects of liberalization. 
	Financial transactions --- within and across borders --- change the location and form of surplus production coagulated into capital, but they do not alter its ownership. 
	
	For example, the capital invested by a German IT firm into a Romanian factory still belongs to that firm, or more accurately, its owners. 
	The offshored factory may be located in Romania, it may be staffed or even managed by Romanians, but it in a market economy with reliable ownership rights, the very building still belongs to the owners of the German holding. 
	Even if manufacturing is outsourced to a Romanian-\emph{owned} firm, the plant will revert to foreign ownership on default, if and to the extent that it relied on debt financing from abroad, as is likely in a capital-stripped economy as Romania.
	
	With ultimate ownership abroad, so too, the factor income will accrue to people abroad. 
	Whether as dividends on equity financing, or interest on debt finance, the share of earnings attributable to the production factor ``capital'' will flow out to those who provided it. 
	These relative factor incomes --- or prices --- are determined, as always, by overall supply and demand for, for example, Romanian-owned labor and German-owned capital. 
	
	Here again, financial transactions follow the same dynamics as trade: 
	to the extent that domestic-owned capital is still relatively scarce in Romania and domestic-owned labor still relatively abundant, foreign-owned capital may, initially, command a higher price, or, equivalently, receive a higher income
	\footnote{
		If and to the extent that the \gls{FPE} applies, the factor income of foreign-owned capital will, however, still be \emph{lower} than the income received by domestic-owned capital under autarky. 
		According to Stolper and Samuelson, the now slightly more abundant factor capital will receive less than before, but may --- or may not --- still receive more than the factor labor.
	}.
		%add reference, link
	In addition, the relative factor earnings of domestic-owned labor, and foreign- or domestic-owned capital will depend on the worldwide supply and demand of the two. 
	For many of the reasons outlined before, capital, as other easily coordinated and scalable production factors may often, but not necessarily always, receive the lion share of income. %add links, thin of other reasons.
	
	Whatever the worldwide relative supply and demand for capital and labor, it will equally influence the relative returns extracted through financial transactions and those extracted through trade. 
	For example, instead of building a plant for electronics manufacture in Romania, a German IT firm might decide --- hypothetically, if implausibly --- to do all the capital-intensive production steps, including circuit printing, in Germany, and then ship these semi-manufactured goods to Romanian homes, where Romanian workers can assemble them by hand with almost no capital required. 
	Crucially, such a hypothetical \emph{trade} is broadly equivalent to the above-described \gls{FDI} in a Romanian factory. 
	In both scenarios, global relative supply and demand for capital and labor will determine how much of the income each of those two factors reaps. 
	In the trade scenario, the relative global factor incomes show up as the price of the semi-manufactured, capital-intensive circuit boards. 
	In the financial transaction scenario, the same relative global factor incomes show up as the interest or dividends that foreign-owned capital receives. 
	For the distributive and welfare effects of liberalization, it matters a great deal what the country-specific factor endowments and overall relative factor demand and supply are. 
	By contrast, whether these transactions are organized as financial transactions, or as trade, is largely inconsequential.
	
	%draw comparison to national accounts. How does it follow?
\end{enumerate}

I argue here that all we really need to know about the welfare and distributive effects of liberalized cross-border transactions is captured by trade, with finance only layering formal complexity on top.

This, of course, is a gross simplification. 
International financial capitalism is complex and crucially important to understand economic liberalization in the short- and medium-term, in the \gls{EU}, and beyond. 
From monetary crises, to taxation, to balance of payments problems, the details of finance matter a great deal. 
The broad abstractions I have drawn here, may be empirically inaccurate, and inadequate to make policy. 
For example, \glspl{FDI} in the real world are \emph{not} fully equivalent to portfolio investments or foreign debt finance, let alone trade, as I suggest in the above: 
building and managing a factory abroad brings a host of spillover effects, many of them positive.

But I do not aim to explain individual events, or calibrate policy. 
Instead, borrowing Mill's grandiose ambition, I want to summarize some deep abstractions to re-imagine our European economy as a household, only with a cast of billions, to see more clearly whether that economy is still matched by our government institutions. 
For that purpose, I do not need the detail and formalisms of finance.

Instead, I metaphorically re-imagine the European household as a glass house, with many translucent floors and rooms. 
What matters are the \emph{real} exchanges in this house, who is doing what and for what reward. 
Trade theory is that birds-eye view of our common household. 
Finance, by contrast, is an optical projection of these exchanges onto a screen behind the house. 
On this screen we see the nominal shadows that the real household draws, and we read on it the financial interactions that accompany the real ones. 
The projections on the screen provide important summary indicators for experts, but they do not turn the world. 
Much like shadows, they \emph{follow} real movement, but they do not determine it.

Capital mobility, of course, adds greatly to the volume of trade and dramatically deepens economic integration. 
It makes possible some exchanges that, without it, would not be feasible. 
For example, a German IT firm may not in fact be able to accomplish the capital-intensive component of production at home, and then ship the circuits for assembly to Romania. 
Reaping this, and many other comparative advantages, may require a \gls{FDI} or some other movement of capital to abroad. 
Still, allowing capital mobility in addition to free trade in goods in services does not alter the fundamental logic of trade, it only expands its reach. 
Again in \citeauthor{Mankiw-2004-aa}'s elegant metaphor, international capital mobility is an extension of the magical technology of trade, applied not only to directly transform US corn into Japanese cars, but also to transform German residential real estate in Bavaria into a used car, into hundreds of construction manhours, into a factory floor in Transylvania, all orchestrated by a magic wand of capital, all initiated by an aba-cadabrus, mouse-clicking electronic money transfer. 
The trick underneath it all remains the same.

\paragraph{Migration.} Just as I have here ignored capital movements, I also ignore movements of that other factor of production, labor migration, and for similar reasons:
\begin{enumerate}
	\item Economic migration, too, is very complex, both empirically and conceptually. 
	People migrate for different reasons, with different skills and for different durations, to name just a few dimensions of migration patterns (e.g. \citealt{DeSimone2008}, \citealt{Bems2008}). 
	I cannot --- and could not --- model or even describe it in these pages.
	\item Economic migration, too, is broadly captured by the theories of trade in goods and services. 
	Whether a low-skilled construction worker builds an export-oriented factory in Romania, or labors --- for similarly low pay --- on the same factory in Germany, for domestic consumption, matters little. 
	\item Economic migration, because of cultural, language and other barriers, is still limited. 
	If and to the extent that labor ever becomes as mobile as capital, goods and services, markets would equilibrate so quickly, that space would almost cease to be a meaningful dimension for the mixed economy.
\end{enumerate}

In the real world, again, the details dwelling in migration, matter a great deal. 
For example, when people go abroad to work, they are often ill-served by existing, national citizenship regimes and fall through the cracks of welfare states, unions, and even employment legislation. 
These are important concerns I cannot address here. 
To re-imagine the european economy as a household, these details must take a back-seat.

\paragraph{Conclusion.} Abstracting away such details, relying on theoretical models rather than empirical data has downsides. 
I need to be cautios in the conclusions I draw from the account of European trade I have here summarized: 
\begin{enumerate}
	\item I must not be \emph{over-confident} in the economic dynamics I have suggested here. 
	Empirical reality of trade liberalization will be a lot messier than these models, our understanding will remain incomplete, and contested. 
	
	A lot more than the abstractions presented here will matter. 
	%For example, the World Economic Forum presents a number of soft, institutional conditions for growth that I do not mention here at all.
	Some models may  be simply untrue, no matter their conceptual elegance. 
	For example, the Stolper-Samuelson \gls{FPE} theorem has not received a lot of empirical support. 
	
	\item I must not be \emph{overly critical} of trade liberalization. 
	Precisely because Europe is such a heterogenous place, not only in capital endowments, but also in climate and worker skills, trade offers tremendous gains. 
	In \citeauthor{Mankiw-2004-aa}'s admiring words, economic integration is akin to a formidable technology, magically turning surplus ``US corn into Japanese cars'', or German engineering prowess into Spanish citrus fruits and Romanian budget cars (\citeyear{Mankiw-2004-aa}: 212). 
	Trade on the continent, in other words, is an unambiguous blessing. 
	
	For example, trade has led to unprecedented growth and prosperity in the new member states, and especially capital mobility has driven convergence \citep{Abiad2007}. %refine, add source.
	
	\item I must not jump to \emph{government interventions} into free trade. 
	It may not be necessary, or even harmful.
	
	For example, interventions such as subsidies or infant industry protection carry grave downsides: 
	they are very difficult to time, target and calibrate correctly, easily invite rent-seeking and can even corrupt a political system. 
	Fundamentally, government is always ill-equipped to \emph{discretionarily} intervene, to, for example, pick the next industrial cluster or winning technology. 
	They cannot easily access the local and dispersed information that markets can collect. 
	To some extent, it may not even be possible to \emph{know} which technology or industry or firm is a future winner; if any person \emph{could} know it, she could become very rich. 
	In this sense, hypothesized (!) efficient markets bite governments, too, and by extension, policy advisors.

\end{enumerate}

But even exercising such caution, these abstractions question whether really, if we only open \gls{EU} borders, the vastly different living conditions on the continent will converge quickly, or at all. 
Trade may often drive convergence and may sometimes hinder it. 
But government, too, can sometimes drive convergence, through transfers, industrial or structural policy. 
As always in the mixed economy compromise, exchange and command modes of production and distribution shine at different things.

I do not advocate, let alone calibrate any specific trade intervention here, but hold that \gls{EU} democratic government must be able to adopt such policies, if markets alone fail to converge living conditions on the continent.

Absent any meaningful \gls{EU} fiscal institutions, but under complete market liberalization, member state and union government are left with \emph{no} tools of the mixed economy to enact such policy. 
No matter the merits and demerits of any such potential trade intervention, government has lost all of the tools to enact it.

\subsection{Money}

\begin{quote}
	\emph{``The European Central Bank has the perfect interest rate for the Eurozone, just for none of its member countries.''}\\
	--- Unknown
\end{quote}

The Eurozone members of the \gls{EU} not only share a currency, but more importantly, \emph{a} monetary policy. 
Within the \gls{EMU}, there is only \emph{one} targeted interest rate and, equivalently, only \emph{one}, union-wide money supply, all set by the \gls{ECB}.

Monetary policy, in the \gls{EMU} as elsewhere, aims to keep prices stable\footnote{
	The \gls{ECB}, like its model, the German Bundesbank, enjoys statutory independence, and must maintain price stability (Art. 27), but the mandate is imprecise. 
	It does not include a time horizon for price stability, or an acceptable range.
	
	The \gls{ECB} has filled this void and defined price stability as at or below 2\% in the medium run.}, 
and maintains a constant money supply over the business cycle. 
During economic downturns, the \gls{ECB} supplies more money to make up for the liquidity frozen up in panicked markets. 
During economic booms, the \gls{ECB} supplies less money to cool down overheated, frenzied markets. 

With only one such monetary policy, the economic area it covers must fulfill certain criteria for the currency union to succeed.

\paragraph{Optimal Currency Area.} \phantomsection \label{sec:OCA} The theory of \gls{OCA} suggests that economic areas can successfully share a currency if they react symmetrically to exogenous shocks, for example an oil price hike, or depressed global demand \citep{Mundell1961}. 
If the entire area reacts roughly similarly, and at the same time to such a shock, the now unified monetary policy can adequately respond to, for example, liquidity shortages. 
If, by contrast, constituent parts react differently, or at different times to such a shock, there can be no single monetary policy to effectively smooth the business cycle.

According to \citeauthor{Mundell1961}'s \citeyearpar{Mundell1961}'s  model with stationary expectations, an area can forms an optimal currency union if:
\begin{enumerate}
	\item factors and goods and services are mobile across the region, 
	\item prices and wages are flexible across the region,
	\item business cycles correlate closely across the region and, as a backstop,
	\item regions share risks and transfer from winners to losers.
\end{enumerate}

In the \gls{EMU}, at least nominally, markets for factors, goods and services are wide open. 
Capital and goods are, indeed, very mobile, but services, certainly labor lag behind. 
For the foreseeable future, workers cannot easily relocate on a continent still divided by language and culture. 
Similarly, prices, and especially wages may remain inflexible, with expectations and labor unions making wages downwardly rigid. 
Lastly, business cycles the \gls{EMU} often do not neatly correlate, as when Spain and Ireland boomed in the 2000s, while Germany was in recession.

To make matters worse, \gls{EMU} founders rejected \gls{OCA} accession criteria in favor of nominal compliance with inflation, deficit, debt, exchange and interest rate targets (\citealt{Begg2008}: 4). 
The \gls{EMU} that happened was a union between \gls{MS} who passed --- or cheated their way through --- a character test of monetary and fiscal discipline, not necessarily a team that had similar needs or skills. 
Nominal compliance may still be a necessary condition for monetary union because a monetary union is a commons --- but it is not a sufficient condition.

Just like the \gls{EU}, closed, mixed economies are also sometimes imperfect currency areas, as for example, when the \gls{GDR} joined the \gls{FRG} in 1990. 
There, too, labor was imperfectly mobile between East and West, wages were downwardly rigid, and inflated in the East and business cycles out of sync. 
As a backstop to such asymmetries, or to entice and compensate the would-be losers from perfect mobility and flexibility, mixed economies can share risks and transfer resources from booming to depressed regions, from winners to losers. 
For example, mixed economies often have currency-wide unemployment insurance, sharing the burden of local downturns.

The Eurogroup, facing more dramatic asymmetries than most other currency unions can --- absent union-level fiscal institutions --- not offer any such backstops. 
In fact, the --- now de-facto defunct --- 1997 \gls{SGP} explicitly forbade fiscal bailouts between \gls{MS}. 
The \gls{EMU} we have today cannot, as the original 1970 Werner blueprint for economic and monetary union envisaged, provide any fiscal stabilization or transfer, but must instead make do with a single monetary response to divergent economic realities. 
It is not only an imperfect, but, much worse, also an impotent currency union.

How does this handicapped monetary regime do its job, in the short term and in the long term?
			
\paragraph[Short Term]{The Short Term: 
Smoothing the Business Cycle.}
In case of an asymmetric shock to aggregate demand, or asynchronous business cycles, \emph{any} money supply the \gls{ECB} sets will be too loose for some, and too tight for other regions. 

Consider what a union-wide central bank interest rate does to a booming outlier region. 
In that region, say, 2000s Spain, too many projects are initiated, too many condominiums built, can reasonably be expected to be successful. 
People are manically optimistic. 
Still, union-wide monetary policy will signal that everything is ok: 
moderate --- instead of high --- central bank interest rates and reserve requirements will incentivize commercial banks to maturity transform many small, short-term deposits into many large, long-term debts, generating fresh money along the way. 
For example, Spanish banks will be eager to attract savings, offer lavish interest rates and invest these deposits into a real estate project, as long as the \gls{ECB} requires such few reserves and exchanges money for securities at such low interest rates. 
Economic activity will persist at above-optimal levels, and cause unambiguous losses: 
many projects will be initiated and condominiums constructed that are not worth the effort. 
More technically, the \emph{relatively} loose liquidity will fuel asset bubbles or even built up inflationary pressures, if and to the extent that savers start to factor in the excessive risk in the interests they demand.

Consider, by contrast, what the same union-wide central bank interest rate does to a depressed outlier region. 
In that region, say, 2000s Germany, many profitable projects are not undertaken, many sound mortgages never signed, because people are depressively pessimistic. 
Still, union-wide monetary policy will signal that everything is ok: 
moderate --- instead of low --- central bank interest rates and reserve requirements will make commercial banks very stingy in their maturity transformation. 
They will look long, and very, very hard at any proposal to pool small, short-term deposits into, large, long-term debt, generating precious little fresh money. 
Economic activity will persist at below-optimal levels, and cause unambiguous losses, too. 
More technically, the \emph{relatively} tight liquidity might cause unemployment, or even risk debt-deflationary spirals, if and to the extent that liquidity-squeezed defaults feed on themselves.

Forcing a single monetary policy without fiscal complements onto an imperfect currency area is like running a psychiatry with only one type of drug --- either an anti-depressant or a antipsychotic --- to treat a diverse group of depressed and manic patients
Whichever prescription you write, it will never be right for all, maybe not even for anyone.

The present, ill-governed \gls{EMU}, notwithstanding all the benefits it has brought, will always breed either inflation and asset bubbles in some regions, or rampant unemployment and depressed growth in some other regions, or maybe even both. 
Absent fiscal complements, \gls{EMU} is destined for economic instability and recession.

\paragraph[Long Term]{The Long Term: 
Resolving Trade Imbalances.} 
Monetary policy has also traditionally helped open, mixed economies in restoring international competitiveness, or, equivalently, resolving resultant trade imbalances. 
In non-free exchange rate regimes, central banks could buy up foreign currencies to make the domestic currency externally cheaper. 
In free exchange rate regimes, the forces of global currency supply and demand would, ideally, equilibrate at an exchange rate where imports equal exports. In both cases, central banks or market forces would, if an economy was externally uncompetitive, make all exports cheaper, and all imports more expensive until trade deficits were resolved. 
Conversely --- if rarely --- central banks or market forces could also let a currency appreciate, making imports cheaper and exports more expensive to resolve trade surpluses.

To be sure, regional differences in competitiveness also occur \emph{within} mixed economies, and they can be resolved by other means than exchange rates. 
Whenever one region runs a persistent trade deficit with another region, over time, a change in the \emph{real exchange rate} between these two regions will resolve the deficit. 
Real exchange rates occur even under nominal parity if and to the extent that the purchasing power of a given amount of currency diverges between regions. 
For example, if Berlin runs a persistent trade deficit with Bavaria, office space, wages, and other immobile factors, goods and services in Berlin will be in relatively less demand than in Bavaria, and will, accordingly, become cheaper. 
A given Euro will buy more office space or labor in Berlin and in Bavaria, effectively making it cheaper to produce things in Berlin and export them to Bavaria.

Unfortunately, such real exchange rate equilibration takes a very long time, especially when prices for factors, goods and services are (downwardly) sticky, as some are likely to be. 
If real exchange rates cannot react swiftly to trade imbalances, these imbalances persist and potentially make the overall economy less stable. 
In addition, welfare is unambiguously lost as a region, marred by uncompetitive prices, produces below its equilibrium output.

Here, again, a mixed economy endowed with fiscal tools can provide a backstop. 
By granting transfers, sharing risks or even subsidizing production in the deficit region, it can make it nominally competitive again, or, preferably, ease the pain of structural transformation and downward-adjusted wages. 

The intra-\gls{EMU} exchange rates were set before years before \glspl{MS} joined, and domestic currencies were then converted to Euros according to that agreed-upon rate. 
Ever since joining the \gls{EMU} --- and in fact, years prior to it to fulfill nominal convergence criteria --- the Eurozone regions had to rely exclusively on slow real exchange rate changes. 
Changes in real exchange rates, or, less cryptically, lower rents and wages, were often tacitly, and, since the 2009ff sovereign debt crisis, openly and fiercely opposed. 
Absent any fiscal mechanism to ease such transformation, real exchange rates have been insufficient to prevent \gls{EMU} from building up excessive imbalances, often precipitating equally excessive private and public debt in the deficit regions. 

This does not mean that all imbalances, let alone all public and private debt in the \gls{EMU} or \gls{EU} would be the result of misguided macroeconomic policy. 
In Greece, Ireland, Spain, Italy and elsewhere, the crisis is also homemade: 
much-needed structural adjustment was delayed, nominal macroeconomic data falsified and tax collection lenient, if not, as other administration, outright corrupted. 
Likewise, the relative success of countries such as Germany, the Netherlands or Scandinavian \gls{MS} is hard-earned: 
painful, and often needlessly inequitable, structural transformation was undertaken and public as well as private spending sometimes kept relatively tight.

Still, here, as so often in a market economy, people, countries and regions are not the only ones making their destinies. 
The deficit \gls{MS} were also \emph{allowed} not to reform, and they were \emph{fated} to remain uncompetitive, all the while surplus \gls{MS} such as Germany from Kohl to Merkel mindlessly cheered on their export records, seemingly ignorant of the deficits and debts elsewhere that their surpluses necessarily implied. 
In this \gls{EMU} without fiscal complements, beggar-thy-neighbor idiocy is reborn, with a Frankensteinian twist: 
instead of racing to competitive devaluation, \gls{EMU} countries are locked into unsustainable exchange rates with any resulting imbalances wantonly misrepresented as southern sloth, profligacy and waste by the oh-so surprised, diligent and thrifty north (e.g. \citealt{Featherstone2011}: 200).

%actually I might need more on this stuff. 
%Consider what it means to run a trade surplus. 

\paragraph{Arguments for the Euro} At this point, skeptical readers and Europeans alike might wonder, why, with all the monetary shenanigans, the union cannot just stick with the common market and leave all else to \gls{MS}. 
Alas, it is not so easy.

A currency union also has advantages. 
In his scarcely-noticed, later take on \gls{OCA} \cite{Mundell1972} argues that if managed correctly, and \emph{if} risks are shared effectively, a larger currency area will always trump a smaller one. 
In a smaller currency area with a flexible exchange rate, exogenous shocks such as a natural disaster will have to be absorbed completely, and at once by that smaller area, as the exchange reacts promptly. 
Exports will immediately become more expensive, possibly causing social unrest.

An exogenous shock such a natural disaster is always an unambiguous material loss. 
But in a larger currency area, the affected sub-area can dissave --- or go into debt --- and pay with existing currency to receive material goods and services from other, unaffected union members. 
Eventually, the debt will have to paid back, but the shock can be dampened and the pain distributed over a longer period of time.

Of course, \cite{Mundell1972} is optimistic about larger currency areas only if, and to the extent that they fulfill the above, restrictive criteria, or --- one might add --- offer fiscal backstops\footnote{
	Mundell himself is a fiscal hawk and arguably not wild about fiscal complements to monetary unions.}.
His central insight remains: 
\emph{not} joining a currency union also has its risks.

These risks apply particularly to an area of open commerce such as the \gls{EU}s common market, where traditional macroeconomic responses to shock-induced exchange-rate volatility are unavailable. 
According to the \emph{unholy trinity} of macroeconomics, governments can only choose two out of three macroeconomic desiderata (\citealt{Mundell1963}, \citealt{Fleming1962}):
\begin{enumerate}
	\item free capital movement
	\item a fixed exchange rate
	\item an independent monetary policy.
\end{enumerate}

 \begin{figure}[htbp]
	\centering
	\includegraphics[width=1\linewidth]{./img/triangle_macro}  
	\caption{The Iron Triangle of Macroeconomics}
	\label{fig:triangle_macro}
\end{figure} 

In the \gls{EU}, free capital movement is a given, and if convergence is to continue it must remain so \citep{Abiad2007}. 
The common market, indeed, is the minimal substrate if regional integration is to mean anything. 
With that corner of the impossibility triangle already occupied, \gls{EU} governments can only choose between fixed exchange rates or independent monetary policy. 
If governments wish to retain monetary sovereignty, they have to accept free exchange rates and any resultant volatility. 
If and to the extent that governments wish to limit exchange rate volatility, they \emph{must} coordinate their monetary policy, that is, enter a currency union.

\paragraph[Fiscal-CPR]{Common Pool Resource Problem.} \phantomsection \label{sec:Fiscal-CPR} If, to stabilize exchange rates, a common market must beget a common currency, to limit budget deficits, a common currency must also beget a common fiscal policy (Feldstein 2005: 7 as cited in \citealt{Begg2008}: 13). 

Without it, a common currency creates a \gls{CPR} of public debt. 
Inside the union, all \gls{MS} governments enjoy the low, average interest rates on its bonds. 
\glspl{MS} can feast on the common good of these low interest rates and take on large amounts of debt without the otherwise defining consequence of heightened interest rates as investors price in greater risk of default. %add link
Other union members may initially try to avoid any potential sovereign default, because in a highly integrated currency union, such defaults may cause systemic risks, especially with foreign union-level banks. 
Additionally, remaining union members may fear loss of investor confidence in the common currency if one of its members is allowed to default on its union-denominated debt. 
In particular, investors always worry that their sovereign bonds will be inflated away by debtor governments and therefore value a track record of sound fiscal and monetary performance not just in debtor governments, but in the entire currency, too. 
Currencies, in effect, are precious brands, vouching for the trustworthiness --- or lack thereof --- of the sovereign debt in which it is denominated.

Anticipating that other currency union members may want to bail out would-be sovereign defaulters, investors will downgrade the risk of any \gls{MS} debt and accept lower interest. 
Reacting on such incentives, \glspl{MS} may borrow more than they otherwise would or should, free-riding on the good reputation of other union members, they might eventually force to bail them out. 
As all currency union members face the full social cost of their sovereign debt, but the overall creditworthiness of the sum of all members naturally remains rival, choosing levels of sovereign debt at the \gls{MS} level becomes a common good: 
no one can be excluded from enjoying low, average interest rates, but overall solvency remains rival.

In a full fiscal union, this \gls{CPR} is resolved: 
the level of debt is decided at the same, highest level at which its costs eventually accrue. 

The \gls{EMU} --- seeking to counteract this moral hazard of communal interest rates --- originally forbade bail-outs, but that commitment, as it now shows, was not credible enough as Feldstein foresaw (2005: 7, as cited in \citealt{Begg2008}: 13). 
Absent any meaningful fiscal institutions to decide together, how much the union should raise, spend and dissave, the \gls{EMU} has fallen prey to its \gls{CPR}, or rather, its offending, extortionate, commons-exhausting members. 

\subsection{Tax}

\begin{quote}
	\emph{``The revenue of the state \emph{is} the state.''}\\
	--- Edmund \citeauthor{Burke1790} (\citeyear{Burke1790}: 111, emphasis added.)
\end{quote}

%In the European Union, only indirect taxes (VAT) are harmonized with limited cooperation in other fields. 
%Member state tax authority is retained to the extent that it is compatible with the free movement of capital, goods and services. 
%Discriminatory (income, capital) taxes based on the country of origin and widely differing VAT rates are regarded as incompatible with the Common Market. 
%To date, the acquis communitaire in tax harmonization comprises of#:
	%Unanimous Council decision making on Commission-proposed provisions for the harmonization of indirect taxation (Article 93 of the Treaty Establishing the European Economic Community of 1957, Article 113 in the Treaty of Lisbon of 2009). Implementations of this norm include:
		%EU Directives legislating harmonized VAT systems (67/227/EEC2 of 1967) and, later, a common minimum rate of 15% (2006/112/EC of 2006). 
		%The minimum VAT rate could be regarded as somewhat ineffective: with the exception of Luxembourg (15%) and Spain (16%), most EU member states have VAT rates between 17-22%.
		%Harmonization of base and minimum rates of excise duties on alcohol, tobacco and energy.
	%As part of a "tax package" in 1997, the Council agreed to a political Code of conduct, to eliminate harmful business tax regimes (preferential tax regimes, PTRs)
	%A multitude of bilateral double taxation agreements exist between Member States, regulating the direct taxes on personal and corporate incomes, possibly conflicting with EU norms. 
		%They are occasionally subject to review by the European Court of Justice for possible violations of Common Market rules.
	%Concerning personal income taxation, the Council passed the European Union Savings Directive (adopted 2004, effective 2005) to automatically exchange information on debt-related income (interest) on foreign nationals of other member states, so that interest incomes from abroad could be taxed domestically. 
	%Member states Austria, Belgium and Luxembourg were granted an exemption. 
	%No common base or rate structure of personal income taxation exists.

	%As a part of the "tax package" in 1997 and as backup to information exchange, the EU Savings Directive instituted a source-based withholding tax on foreign-owned accounts of 35% (phased in until 2011) in the three exempted countries. 
	%The savings directive applies only to bank interest, bond incomes and analogous financial instruments, but not to dividends, capital gains or any other profit realised on investments. 
	%The witholding tax is also below even the EU average personal income tax rate at 37,8%, and much below EU-15 PIT rates, and respective marginal rates. 
	%The witholding tax revenue is shared between sending and receiving country (25-10). The witholding tax, by definition, is linear, not progressive as the income tax it is meant to replace.

	%Concerning corporate income taxation, the Council adopted the Interests and Royalties Directive (2003/49/EC, adopted 2003, phased in in New Member States (NMS) until 2014) abolishing all withholding taxes on corporate interest and royalty incomes of firms registered for taxation in another member state. 
	%Similar directives have been passed with regard to mergers (90/434/EEC), and parent-subsidiary relationships (2003/123/EC) to streamline the taxation of cross-border intra-company flows of income. 
	%No common base or rate structure of corporate income taxation exists.
	%Tax in the Treaties

	%Article 113, Treaty of Lisbon, 2009 / Article 93, Treaty Establishing the European Economic Community, 1957
	%The Council shall, acting unanimously on a proposal from the Commission and after consulting the European Parliament and the Economic and Social Committee, adopt provisions for the harmonisation of legislation concerning turnover taxes, excise duties and other forms of indirect taxation to the extent that such harmonisation is necessary to ensure the establishment and the functioning of the internal market within the time limit laid down in Article 114.

	%Article 115, Treaty of Lisbon, 2009 /  Article 94, Treaty Establishing the European Economic Community, 1957
	%Without prejudice to Article 114, the Council shall, acting unanimously in accordance with a special legislative procedure and after consulting the European Parliament and the Economic and Social Committee, issue directives for the approximation of such laws, regulations or administrative provisions of the Member States as directly affect the establishment or functioning of the internal market.
	%Today, all EU tax harmonization legislation, even under its limited mandate (indirect taxes) according to Article 113, is passed by unanimous vote in the Council, based on proposals by the Commission, after consultation with the Parliament. 
	%All other, more general harmonization of taxation, including most of the above, is admissible only if it "directly affects the establishment or functioning of the internal market" pursuant Article 115. Also in this case, unanimity is the Council decision rule.
	%In addition, the European Court of Justice can rule on (discriminatory) tax regimes that are held incompatible with the four freedoms. 

	%The Commission has lobbied for qualified majority voting (QMV) in certain tax areas. The Intergovernmental Conference (IGC) on a Constitutional Treaty for the EU (2003-2004), notably, was supposed to include QMV with regard to some tax issues. 

	%The Commission is further committed to establishing a common base, not rate, for corporate income taxation in the Union. It advocates taxation of incomes based on permanent residence (PIT) and seat (CIT). 
	%The Commission is undertaking negotiations to pioneer seat-based corporate income taxation for small- and medium-sized businesses.
	
		%Background: Taxation Trends in the European Union

	%The Directorate General for Taxation and Customs Union issues an annual report on taxation in the European Union. Results from the 2009 edition include:
	%Tax ratios in the EU-27 remain relatively high (39.8% of GDP), but differ greatly between old and NMS (Romania 29.4%, Denmark 48.7%).
	%NMS raise relatively more revenue by indirect, non-redistributive taxes in immobile bases.
	%Almost all member states increase the (indirect) burden in (relatively immobile) consumption through VAT and excise duties. 
	%Top average (not marginal) Personal Income Tax (PIT) rates (37.8%) are in decline across almost all EU member states, but continue to vary dramatically between old and NMS (Bulgaria 10%, Denmark 59%).
	%Corporate Income Tax (CIT) rates are in rapid decline, from 35.3% in 1995 to 23.5% now. Again, NMS tend to have lower tax rates than older member states. 
	%Implicit Corporate Income Tax rates (CIT-ITR) are however, stable if diverging, possibly due to cyclical effects, base broadening or cannibalizing on the PIT.
	%Is there European (Corporate Income) Tax Competition?

	%In a liberalized Single Market, where capital and, to a lesser extent, labor through trade, investment and migration flow to their most profitable use, it appears reasonable to assume that these factors of production will also respond to taxes. 
	%Investors and workers will, as much as they can, flock to locations where tax rates are lower than the respective costs of relocation. 
	%To sustain output and growth, states would then have to compete for factors of production with low tax rates.

	%Before turning to the central question of whether this would be a desirable or undesirable dynamic, first ask whether it is real, and if so, how it came about.

	%Genschel, Kemmerling and Seils (2009) suggest (corporate income) tax competition is shaped by four interrelated institutional mechanisms of the EU:
	%Market integration (↑) reduces transaction costs of cross-border arbitrage (think: exchange rate volatility, tariffs) and thereby facilitates tax competition.
	%Enlargement (↑) adds new, attractive markets: new members are diverse in size (!) and economic development (GDP/Capita). 
	%Smaller and poorer countries have greater incentives and possibilities to lower taxes.
	%Tax coordination (↓) makes taxation more similar, and therefore reduces the ways in which governments can compete for capital and labor.
	%Supranational judicial review (↑) enforces the non-discriminatory liberalization (think: Cassis de Dijon) and limits the ability of governments to unilaterally defend against tax competition (think: Tobin Tax).

	%Genschel et al. find that tax competition is different and greater within the European Union than outside of it, and that it accelerates with time and enlargement.

	%They also suggest that larger countries suffer disproportionately from tax competition as smaller countries can boost their revenues (and growth) with the low rates on the massive inflowing capital from larger neighbors. 
	%Conversely, poorer countries have greater incentive and ease to compete for scarce capital.
	%The competition is less harsh amongst (now delegitimized) preferential tax regimes (PTR) (think: Hong Kong). 
	%When countries target only the most mobile of factors with tax breaks, the overall revenue effect tends to be smaller and more symmetric.

	%(A Prisoner's Dilemma Game of Tax Competition. Payoffs are tax revenues#.)

	%If these empirical results are correct, EU tax competition can be modeled as a Prisoner's Dilemma game, where states (strictly) dominantly prefer low taxes over high taxes and (Nash) equilibriate in suboptimal, mutual low taxation.

	%The Case against European Tax Competition: It's a Race to the Bottom

	%The argument against European tax competition builds on the assumed Prisoner's Dilemma dynamic, and suggests that EU states may not only be incapable to maximize public revenue under competition, but that thereby depressed tax revenue will lead to debt crises, the underprovision of public or common goods and impaired redistributive ability or a retrenchment of the Welfare State. 
	%It could also be argued that a harmonized European Tax regime better equips member states to withstand shocks with countercyclical tax and spend policies.

	%A related argument can be made about a presumed inability to respond to structural misalignments caused by liberalization. 
	%When trade, migration and investment allow countries to specialize even more according to their factor endowments (think: Romanian Nokia, German Management Consulting), remaining, relatively scarce factors (think: unskilled laborer in Germany) may find their market wages# fall even further below the respective socially acceptable minimum income#. 
	%Rich states may then be forced to redistribute income to these individuals, but find themselves unable to raise the necessary revenues (progressively) without further reducing their competitiveness. 
	%Tax competition could then exacerbate vicious dynamics of structural unemployment.

	%Tax competition is harshest on tax bases (capital, labor, firms) that are highly mobile (think: private equity). 
	%EU member state tax codes may be forced to converge on certain kinds of taxes. 
	%To appreciate this possibility, consider the qualities of different redistributive-/general-revenue taxes#.

Taxation in the \gls{EU}, for the most part, remains an exclusive competence for \gls{MS}. Under the acquis, only indirect taxes (VAT) are harmonized --- at ineffective, minimal levels --- with very limited cooperation in other fields (e.g. \citealt{EuropeanCommission2009}, \citealt{TaxCoordinationandTaxCompetitionintheEuropeanUnion-EvaluatingtheCodeofConductonBusinessTaxation2001}). 
Per its treaties, the \gls{EU} can harmonise only \emph{indirect}, always proportional taxes --- such as \gls{VAT} --- and only by \emph{unanimous} decision of the Council on proposals by the Commission (Article 113, Treaty of Lisbon, 2009 / Article 93, Treaty Establishing the European Economic Community, 1957). 
Union members do not even fully cooperate in collecting existing, national taxes: 
instead of full reporting of all incomes, some incomes (e.g. dividends) and some countries (e.g. Luxembourg) are exempted and instead levy a proportional, much lower withholding tax. 
It is, in short, no exaggeration to say that the \gls{EU} has no fiscal institutions or even coordination to speak of.

In the \gls{EU}, there is no match between the scope of economic activity --- the union-wide common market --- and the scope of taxation. 

With no union-level taxation, what does this mismatch do to 
\gls{MS}-level taxes? 
As the abstractions of the mixed economy suggest, tax competition results. 
For example, \cite{Genschel2009} find that tax competition is different and greater within the \gls{EU} than outside of it, and that it accelerates with time and enlargement. 
\gls{EU} tax competition is a \gls{PD}, where states (strictly) dominantly prefer low taxes over high taxes and (Nash) equilibriate in suboptimal, mutual low taxation (table \ref{tab:EU_Tax_PD}, p. \pageref{tab:EU_Tax_PD}).

\input{./tex/intl_tax_competition_PD}

If, and to the extent that such a race-to-the-bottom is at play in the \gls{EU}, it will affect both levels and schedules of national taxation.

\paragraph{Levels.} Straightforwardly, \gls{MS} will be strictly limited in the overall level of taxation they can sustain. 
Governments will no longer be free to set a level of taxation, or, equivalently, determine the command-exchange components of the mixed economy. 
In some cases, tax levels may even fall, as has been shown for \gls{CIT} rates \citep{Piatkowski2008}.

%re-read Piatkowski about this stuff.

\paragraph{Base.} Moreover, and more importantly, competition will also alter the base composition of taxation. 
To avoid large \glspl{DWL}, as they should, governments will turn to bases that are relatively price inelastic, that is, economic transactions that cannot be altered to escape taxation. 
\gls{EU} integration opens up a lot of new escape routes, especially for newly mobile capital, and, to a lesser extent, high-skilled labor: 
they can relocate their economic activity to wherever the tax burden will be lowest. 
This causes welfare-depressing distortions in the high-tax economy: 
rather than face a now voluntary tax, these pareto-optimizing exchanges will not be made at all, and instead happen elsewhere. 
For example, a rich entrepreneur otherwise willing to open a new factory in high income-tax Germany, may, faced with the new alternative of building the same facility in a low-tax location, forego his original plan. 
Germany unambiguously looses welfare, both because the investment is not made, and also because it does not even generate any fiscal revenue.

Faced with these dynamics, governments will, again rightly so, shift their taxation to bases that are less prone to \glspl{DWL}, or equivalently, bases that are relatively less mobile. 
Relatively less mobile bases in the \gls{EU} will be consumption and labor incomes, because consumers and workers cannot easily do their shopping and working in another country. 

Other --- partly dysfunctional --- taxes traditionally used to raise revenue for mixed economies will be rolled back or falter altogether. 
This applies especially to --- anyway defunct --- national \glspl{CIT} that large corporations can often evade easily, in part because nailing down the locale of any particular increment of income of a multinational firm will always be conceptually difficult. 
For example, the German holding of Deutsche Bank AG can easily reassign a particular income stream to a Luxembourg-based subsidiary, arguing that a crucial business process occurred there. 
Tax administrations will always, and necessarily, be unable to argue where any particular value was created (\citealt{Ganghof2006}, \citealt{Ganghof}, \citealt{Ganghof2007}: 5). 
Similarly, higher brackets of progressive \gls{PIT} will also cause large \glspl{DWL} or, more likely and wisely, disappear, as high-income individuals change residence or citizenship, offshore their income-generation to other countries, or at least shelter it in foreign corporations no longer affected by a high, backstop \glspl{CIT}. 
For example, a rich German entrepreneur can establish a new holding in Ireland to buy up his German-based firm, and have it retain most if not all of the earnings, effectively escaping german income taxation. 

If and to the extent that Pigouvian taxes, or even fees fall on mobile bases, these will also either cause excessive market distortions, or, more likely, disappear. 
For example, a German steel producer may, (hypothetically!) faced with the German ecotax, relocate to Poland, avoiding the higher energy price, \emph{without}, as the Pigouvian tax intended, raising the price of steel. 
Overall energy intensity will remain the same, steel production will simply fall below equilibrium levels in Germany.

\paragraph{Schedule.} Crucially, by shifting the base, \gls{EU} \gls{MS} will also alter the schedule of their tax regimes. 
By relying more on taxing labor incomes, schedules will become more regressive: 
most large incomes in developed capitalist economies are not labor, but capital incomes. 
By relying more on (pre-paid) consumption and other indirect taxes, schedules will become regressive or --- at best --- proportional: 
even if rich people, just as others, eventually spend all their income, they will only pay the same percentage in \gls{VAT} or similar taxes. 
If union member governments wish to avoid, as they should, excessive \glspl{DWL} they will have to sacrifice progressivity in tax. 
The already impaired, but lone vestige of progression, the \gls{PIT} together with its ugly, but necessary backstop\footnote{
	Personal income taxes on capital necessitate a corollary taxation of corporate income. 
	Without it, taxpayers could easily evade payment by incorporating capital it in a firm, withdrawing earnings only slowly.}, 
the \gls{CIT} will either disappear altogether, or, largely equivalent, depress their schedules and degenerate into effective labor income taxes, with, at best, some residual but proportional taxation of capital.

\subsection{Dysfunctions} \label{sec:defunct} What kind of an economic reality results from this open, but heterogeneous \gls{EU}, with unbounded trade, mis-configured currency union and rampant tax competition? 
It is, and must be, a deeply dysfunctional design, boxing the European household in unattractive policy dilemmas, wasting its communal resources, ever building new imbalances, harboring new crises, and, ultimately, fracture the social contract.

\subsubsection{Underfunding} \label{sec:public_squalor} Straightforwardly, the strictly limited revenues of \gls{EU} \gls{MS} confine them to structural underfunding, or at least, constrain the command-exchange \hyperref[sec:tradeoffs]{tradeoffs} that mixed economies are otherwise free to make (p. \pageref{sec:tradeoffs}). %add reference.
By subjecting taxes to competition, any increment in more command production and distribution must be bought at an increasing price in lost economic activity, or \gls{DWL}. 

In this scenario, governments rebalance their Haig-Simons identities --- as they always must --- by cutting public consumption or dissaving out of their wealth. 
This can take many, but entirely equivalent forms. 
For example, governments can save on public goods, such as road maintenance, or it can reduce transfer payments, such as welfare benefits. 
It can also dig into its savings, and take on new debt, or, let infrastructure fall into disrepair.

Here, too, government is faced with unattractive choices: 
to either cut public spending to suboptimal levels, to go into debt or to otherwise dissave.

%need evidence

\subsubsection{Unemployment}
The neoliberal agenda promised that if states cut their spending, at least their economies would grow faster. 
Under a dysfunctional mixed economy, that is not necessarily so. 
In the \gls{EU}, mixed economies cannot have the cake and eat it, they cannot even do one of the two. 
Instead, its welfare states are faced with a twin crisis that is mutually reinforcing: 
one of structural unemployment, and one of structural underfunding, as illustrated in figure \ref{fig:twin_crisis} (p. \pageref{fig:twin_crisis}). 

Structural underfunding, aside from causing \hyperref[sec:public_squalor]{public squalor} (p. \pageref{sec:public_squalor}), in the long run may also diminish the kind of public and common goods that drive future economic growth, such as basic research or infrastructure, and especially, education. 
Over the long haul, structurally underfunded states will ill-equip workers for a global marketplace, and leave them with comparatively poor labor productivities.

In addition, structurally underfunded governments are increasingly unable to transfer resources to low- and middle-income workers, or, at least, exempt them from taxation. 
In particular, the greater tax burden on immobile labor, and the constrained progressivity of tax under competition will make it even harder for workers to make ends meet at any given market gross income. 
They have to pay more --- not less --- to the state, or, misleadingly named ``social insurance'' and keep even less as net income for consumption.

As a result, at least some workers will be relatively unproductive and face high taxes on their already low or middle market incomes. 
If and to the extent that \gls{EU} welfare states maintain some minimum socially acceptable living standard, either through a minimum wage, or --- equivalently --- welfare transfers, these low and some middle income earners will find it increasingly difficult to earn enough on the market to meet this standard. 
Any --- already diminished --- income will be further depressed by a tax wedge driven between the gross and net disposable incomes. 
Both in a minimum wage, and a welfare transfer regime, those workers with productivities too low to make the minimum income on the market will exit the market, and collect welfare instead --- not out of laziness, but out of necessity. 

The ensuing structural unemployment, in turn, reinforces the structural underfunding of the mixed economy government. 
First, it creates greater needs for transfers, putting further strain on public transfers. 
Secondly, it also depresses growth, and in the long run, diverges the economy from its long-term growth path, as segments of the labor force lie needlessly idle.

\begin{figure}[htbp]
	\begin{center}
	\includegraphics[width=1\textwidth]{./img/dual_crisis}  
	\caption{The Dual Crisis of the Late Welfare State}
	\label{fig:dual_crisis}
	\end{center}
\end{figure} %is this already used elsewhere?

Alternatively, of course, governments can lower effective price floors by cutting welfare benefits, minimum wages or by raising work requirements. 
By lowering minimally acceptable social standards --- frequently euphemised as ``structural realignments'' or ``labor market flexibility'' --- states will have to abandon central welfare tenets and accept, once again, widespread working poverty. 
%add some butterwegge results here.
That is, \gls{EU} welfare states can brake the vicious cycle of underfunding and unemployment if they cease to be welfare states, a configuration that \cite{Streeck2010c} has aptly called a ``permanent austerity regime''. 
%visualize this already as an unattractive choice?

Governments of dysfunctional mixed economies, here, as always, are faced only with equally unattractive options: 
to either save social standards at the price of structural unemployment and depressed growth, or to abandon them and risk widespread working poverty.

This dual crises, and the uneasy choices it forces, will only be exacerbated in a modern and open economy. 
Modern economies already produce highly unequal returns, as winners take all and, equivalently, Baumols cost disease looms. 
A modern economy will, by its very structure tend to produce people whose productivities are much lower than the overall productivity of their host countries. 
%add hyperref.  
Trade, migration and capital mobility add even more pressure. 
As countries specialize even more according to their factor endowments (think: Romanian Nokia, German Management Consulting), remaining, relatively scarce factors (think: unskilled laborer in Germany) may find their market wages fall even further below the respective socially acceptable minimum income. 
Especially rich states may then be forced to redistribute income to these individuals, but find themselves unable to raise the necessary revenues (progressively) without further reducing their competitiveness. 

%need evidence

\subsubsection{Inequality} Lacking any union-level fiscal institutions and marred by tax competition between the \gls{MS}, the \gls{EU} mixed economy lacks effective tools to redistribute market outcomes. 
Both \emph{within} and/or \emph{between} member states, rampant inequality will remain unchecked, or even further widen.

At home, the mixed economy has lost its ability to dampen (possibly accelerating) winner-take-all dynamics, and to compensate the loosers from trade and economic transformation (e.g. \citealt{Beckfield2006}\footnote{
	\citeauthor{Beckfield2006} finds for the \gls{EU}-12 that nearly half of the rise in within-country inequality between 1973 and 1997 can be explained by regional integration e.g. (\citeyear{Beckfield2006}: 979).}): 
inequality almost everywhere in the \gls{EU}, has been widening, at least in part due to regional integration (e.g. \citealt{DaudUngl2008}: 265). 
The great U-turn back towards more inequality, in Europe as elsewhere in the \gls{OECD} is well under way \citep{AldersonNielsen-2002-aa}. 
As tax competition both erodes the base and depresses the progressivity of taxation, market allocations, increasingly, are final. 
Moreover, structural underfunding and associated public squalor will also hit hardest the lower and middle income earners, further widening the divide in living standards. 
Rich people can afford to exit from public provision, for example by paying doctors out of pocket, by sending their children to private schools or even walling their gardens and gating their communities. 
Lower and middle income earners have no such exit option, but are stuck with decrepit public provision.
%need evidence

Between \gls{MS}, too, inequality will remain unchecked, as member and union level governments have no instruments to alter distributive dynamics of trade, that may --- or may not --- lead to fast convergence of productivities, and related, living standards. 
What is worse, the poorer mixed economies are especially constrained: 
at low productivities, they can least afford to burden mobile capital, and other mobile, high-earning factors, such as professionals with any, let alone progressive taxation. 
With open borders, but without coordinated tax or union-level transfers, these poorer \gls{MS} currently can only take the hard, unmitigated route to economic convergence: 
they tax mostly (low-productivity) labor, proportionally if not regressively, and at low overall public spending levels (e.g. \cite{DaudUngl2008}: 267). 
By contrast, in the higher-productivity, rich \gls{MS}, corporatist arrangements, strong trade unions and substantial, if increasingly dysfunctional welfare regimes can still eek out pockets with sometimes generous welfare provision: 
in always capital-intensive, often high-value add and sometimes oligopolistic --- not commodity ---production, these economies can, at least in some sectors, afford welfare. 
For example, a Bavarian specialist engine builder with high capital and relatively low labor inputs and maybe a handful competitors on the world market,  may, faced with strong unions, accept above-equilibrium, possibly efficiency wages. 
Not so in Romanian manufacturing: 
Producing low-margin commodities, with little capital but hundreds of competitors and easy access, firms have to compete tooth-and-nail on labor costs. 
And so, \gls{MS} may not only fail to converge as quickly, or as closely as they could, or hoped to, but the very \emph{conditions} for economic development will diverge widely. 
The rich, high-productivity \gls{MS} can still, if inefficiently and incompletely, dampen and distribute the pain of whichever economic shock hits or transformation sets in. 
In the poor, low-productivity East and South, it will be bare-bones laiss\'{e}z-faire capitalisms\footnote{
	\citeauthor{Galbraith2002a} (\citeyear{Galbraith2002a}: 25) precisely describes an analogous, worldwide dynamic:
	\begin{quote}
		``In sum, it is not increasing trade \emph{as such} that we should fear. 
		Nor is technology the culprit. 
		To focus on `globalization' as such misstates the issue. 
		The problem is a process of integration carried out since at least 1980 under circumstances of unsustainable finance, in which wealth has flowed upwards from the poor countries to the rich, and mainly to the upper financial strata of the richest countries. 
		In the course of these events, progress toward tolerable levels of inequality and sustainable development virtually stopped. 
		Neocolonial patterns of center-periphery dependence, and of debt peonage, were reestablished, but without the slightest assumption of responsibility by the rich countries for the fate of the poor. 
		It has been, it would appear, a perfect crime. 
		And while statistical forensics can play a small role in pointing this out, no mechanism to reverse the policy exists, still less any that might repair the damage. 
		The developed countries have abandoned the pretense of attempting to foster development in the world at large, preferring to substitute the rhetoric of ungoverned markets for the hard work of stabilizing regulation. 
		The prognosis is grim: 
		a descent into apathy, despair, disease, ecological disaster, and wars of separatism and survival in many of the poorest parts of the world. 
		Unless, of course, the wise spirits of Kuznets and Keynes can be summoned back to life, to deal more constructively with the appalling disorder of the past twenty years.''
	\end{quote}}.

In the \gls{EU}, Kuznet's and Keynes' grand hopes, that welfare would --- and should --- always follow growth, are dashed (as cited in \citealt{Galbraith2002a}: 22). 
If they have any choice at all, it is a very unattractive one for the governments of the union:
they can either stay in the common market and reap the gains from trade and abandon all or some welfare, \emph{or} they can exit the union, stall economic integration, save their welfare regimes and retreat to autarky and recession.

\subsubsection{Imbalances and Crises} 

\begin{verse} 
	\href{http://www.npr.org/blogs/money/2012/03/01/147720368/50-ways-to-leave-your-lender}{50 Ways to Leave Your Lender}
	
	The problem is all inside your head, she said to me,\\
	You can't pay back 200 percent of GDP,\\
	You have to negotiate, if you want your country free,\\
	There must be 50 ways to leave your lender.

	You really don't want the IMF to intrude,\\
	Furthermore, they'll force austerity for the interest that's accrued,\\
	Imagine your middle class, subsisting on cat food,\\
	There must be 50 ways to leave your lender.\\
	Fifty ways to leave your lender.

	You just stretch out the loan, Joan,\\
	Cut the creditors' hair, Claire,\\
	Or boost GDP, Lee,\\
	Just listen to me.

	Print more money, honey.\\
	No need to pay back, Jack!\\
	Structure a default, Walt.\\
	And get yourself free.\\
	--- Planet Money / National Public Radio, 2012
\end{verse}

Democratic governments, firms and households alike will be under great strain from the underfunding, unemployment and inequality that a dysfunctional mixed economy creates. %add hrefs.
They may take on any possibility to temporarily relief the pressure they are under, even if it will not solve, or even exacerbate the situation in the long run: 
here, too, humans and the institutions they man, suffer from time inconsistency.

In modern financial capitalism and complex societies, there are some powerful painkillers to numb the effects of a dysfunctional mixed economies. 
As painkillers go, they treat the symptoms, not the disease, and have serious side effects. 
And so it is with the macroeconomic temptations in the \gls{EU}: 
as powerful drugs, they seemingly let economies transcend their material means, spread euphoria and frenzy. 
Only their cure is, ultimately, delusional, their treatment addictive. 
While under the charm of such chimerical boom and prosperity, economies keep building pressures and imbalances, that, one day, will unload in financial shocks and systemic crises, that, if sufficiently large, can disturb or bring down entire markets. 
After this kind of ecstasy always comes a day of reckoning, with a catastrophic hangover.

Just when an economy is in such drug-enduced delusion, and living beyond its long-term growth path is, as always in uncertain markets, hard to tell. 
National balance of payments accounts provide as in figure \ref{fig:BoP} (p. \pageref{fig:BoP}) an intuitive, if rough-and-dirty first indication. 
Between economies, too, an identity akin to Haig-Simons and the conservation of matter, holds: 
for any good or service that leaves the country, there must ultimately be imports of equal value, or, a change in ownership of foreign assets, that is, the promise of \emph{future} imports of goods and services. 
Conversely, any import must be matched by exports of equal value or it will be offset in \emph{foreign} ownership of domestic assets, that is, claims against future domestic production. 
As all the most important economic abstractions, this one is simple: 
balance of payments accounts are double-entry bookkeeping, only at the economy level.  

The components of balance of payments accounts, as the Haig-Simons identity, break down over households, firms and governments. 
For example, in a fictions German account, households can import olive, or firms can import particle filters as semi-manufactured inputs, or governments can import commuter trains for public transportation (figure \ref{fig:BoP}, p. \pageref{fig:BoP}). 
In these accounts too, positions of one owner are offset by positions of other owners in the same economy. 
For example, German household exports of home-made cuckoo clocks can be offset by said firm imports of particle filters in a roundabout way, when clockmakers buy French-particle-equipped German cars, or following some other chain of exchanges. 
Positions  also offset across the equality sign. 
For example, German household imports of olive oil may be offset by government issues of German bonds to Greek oil producers, with the government channeling the revenue to oil-consuming welfare recipients, or through a myriad of other transfers.

\begin{figure}[htbp]
	\begin{center}
	\includegraphics[width=1\textwidth]{./img/BOP}  
	\caption{A (German) Balance of Payment Account with Examples}
	\label{fig:BoP}
	\end{center}
\end{figure}

Balance of Payment accounts are easily misunderstood or oversold, for a four reasons:
\begin{enumerate}
	\item Trade deficits and surpluses between any pair of countries are frequently reported, but meaningless and entirely unproblematic, just as shoppers need not worry about a trade deficit with the local supermarket. 
	Trade deficits --- as consumer debt --- are potentially worrying only if they are \emph{net} of all exchanges with all trading partners.
	\item Conversely, balance of payments accounts do not apply  only between countries, as is easily assumed, but is, in fact a meaningful and true identity between any group of market participants and the rest of their trading partners, all the way down from nations to households. 
	For example, a trade deficit may also arise between laggard regions, impoverished demographics or even generations, and the rest of an economy, with much the same possible problems.
	\item In the short term, even such trade deficits may not be problematic, but, in fact, help to stabilize economies from exogenous shocks. 
	%reference OCA argument
	\item Even in the medium and long run, persistent trade deficits may be ok if and to the extent that the resultant capital inflows can reasonably be expected to currently, or in the future, earn whichever factor income was promised. 
	For example, emerging economies may well experience persistent trade deficits for some time, while machinery is imported to equip the workforce, if and to the extent that the resulting, now capital-deepened production pays off as expected.
\end{enumerate}

Still, balance of payments accounts are an immensely enlightening abstraction, without which trading mixed economies cannot be well understood:
\begin{enumerate}
	\item Trade deficits are not a sufficient, but still a necessary condition for building macroeconomic imbalances. 
	Not every trade deficit will betray an economy living beyond its means, but every economy artificially held above its long-term growth path by said financial drugs \emph{will} leave a grave trade deficit in its wake.
	\item The balance of payments identity shows, as Keynes  argued forcefully, if somewhat ineffectively at the Bretton-Woods conference in 1944, that these macroeconomic imbalances know no \emph{one} culprit. 
	The loaded language notwithstanding, both deficit \emph{and} surplus economies are, equally, at fault. 
	One parties excessive imports are another parties dumping exports. 
	To get to equilibrium, where imports equal exports, either of the two parties, or both, must change its prices. 
	To pride oneself, as German leaders frequently do, in being an export champion --- but not an import champion --- is but a mindless return to the folly of beggar-thy-neighbor, and, before that, mercantilism.
	\item Financial flows always track flows of tangible goods and services, as well as vice versa.
	\item It does not much matter \emph{who} --- households, firms or government --- in an economy creates the trade deficit. 
	Only the deficit net of all economic actors in a given region matters. 
\end{enumerate}

How, then, do we know the acceptable trade deficits, from the unsustainable ones? 
We look at the offsetting changes in the capital account, and check whether these are intertemporally efficient, or whether they were enabled by failed markets. 
In the \gls{EU}, as in any other mixed economy, we must beware of these smoke and mirrors, that only forestall and worsen the inevitable day of reckoning: 
credit bubbles, asset bubbles, inflationary pressure, and nominally invisible, but real dissavings.  %add hrefs.

\begin{description}
	\item[Credit Bubbles \& Default.]  Trade deficits can precipitate in capital inflow, as new foreign-held debt. 
	To receive their extra imports, deficit economies issue different forms of IOUs, including government bonds, corporate debt and household credit, sometimes backed by physical collateral, as in a mortgage.%gls IOU

	If the sum of these debts, is sound, so is the trade deficit. 
	If debts, sour, or were overly optimist to begin with, the trade deficits cannot stand. 
	Consider the two scenarios:
	\begin{enumerate}
		\item The loans are performing as long as, if, and to the extent that whichever projects they financed generate sufficient earnings to pay back interest and principal. 
		For example, if the extra, imported surplus production the IOUs enabled were transformed into a competitive factory that now churns out export merchandise, the loan can be paid be back out of these exports and revenues. 
		
		In the \gls{BoP}, the initial trade deficit is first offset by the loaned capital inflow, which later flows out again as the loan amortizes, offset by foreign factor payments and exports of the produced merchandise. 
		In effect, the loan has, as efficient credit should, inter-temporally balanced past trade deficits with future trade surpluses and/or foreign payments. 
		It matters little whether, and in which proportion the amortization on the capital account is offset by either foreign payments or equivalent actual exports, and whether the factory's merchandise is actually for export or domestic consumption. 
		In the balance of all economic transformations and exchanges, successful factories and other projects can always honor their loans \emph{without} curtailing the living standard of the population. 
		Interest, and maybe even collateral, are paid back out of \emph{extra} production that would not have otherwise occurred. 
		We need not worry about this kind of trade deficit: 
		because it moves everyone closer to the long-term growth path, is an inter-temporal Pareto, or at least Kaldor-Hicks optimization.

		\item The loan goes bad as soon as, if, and to the extent that whichever projects they financed do not generate sufficient earnings to pay back interest and principal. 
		For example, if the factory is not competitive, or --- more to the european point --- no one needs or can afford the airports, malls and mansions into which the extra imports were coagulated, there are no revenues or exports to pay back the loan. 
		In the extreme, but conceptually similar and now plausible case, the extra imports were not meaningfully coagulated into capital at all, but were simply consumed away at present.
		
		Come the day of inevitable reckoning, the deficit economies have two choices:
		\begin{enumerate}
			\item If the loan in question carries effective recourse, the deficit economy has to return the loaned capital in other, \emph{material} ways. 
			As when a leasing company repossesses a car on which payment the lessee has fallen behind, deficit countries must return the surplus production in some form. 
			For example, the deficit economy may ship back the foreign-financed machinery in the project, or, more likely, return the same amount of surplus production transformed into some other good or service. 
			
			This is the hard way of rebalancing the \gls{BoP}: 
			the inevitable, promised outflow of capital on the capital account can be balanced only with an often painful trade \emph{surplus}, because there are no factor incomes to be otherwise offset on the current account. 
			Either way, a failed investment enforces a later, and often painful trade surplus to return principal and return.

			\item Alternatively, if and to the extent that debtor economies (can) forego recourse and exert sovereignty vis-a-vis creditor economies, they (partially) default on their commitments and simply refuse to return the coagulated surplus production. 
			In that case, creditors are stuck with their claim. 
			By fiat, the original loans become full, or partial \emph{transfers} from the debtor to the creditor economies. 
			Here as always, the two sides of the \gls{BoP} identity cancel out: 
			the original trade deficit is matched by a later, ex-post, enforced, foreign payment in the form of debt forgiveness, haircut or default.
		\end{enumerate}

		No matter the choice, this kind of trade deficit is never an optimization, but an unavoidable \emph{redistribution}, either from surplus future to deficit present if and to the extent that debtors pay, or from creditors to debtors, if and to the extent that debtors default.
		
		Crucially, it matters very little who in the deficit economy --- households, firms or government --- initially took on debt. 
		These non-performing loans will redistribute from future to present, or creditor to debtor no matter who signed them. 
		In many cases, government will be forced to act as the lender of last resort and take on, or guarantee all the bad loans, both to counteract adverse selection and, often to save an exposed banking system from systemic crash. 
		Even if and to the extent that government, or, equivalently, future taxpayers, can avoid to take on the bad loans, the redistribution is merely concentrated on whoever remains nominal debtor. 
		Domestic policy can force only \emph{some} people --- ideally those responsible --- to repay, but, short of default --- another redistribution --- it cannot void the need to repay. 
		Here, as always, something akin to economic conservation of matter reigns: 
		when credit bubbles burst, someone will have to pay back the future, either some debtors, all taxpayers, some creditors, or any combination thereof.
		
		In addition to these mere inter-temporal distributions, credit bubbles and associated mass defaults or austerity, of course, also waste economic welfare, because of the turmoil they harbor, not to mention the hardship they imply. 
		As the business cycle fluctuates wildly in such crises, the economy diverts from the long-term growth path, leaving resources either depressively idle, or manically scarce.
		
		Credit bubbles are a market failure that may plague any economy --- not just the \gls{EU} --- but the european, defunct mixed economy is particular prone to them, for at least three reasons:
		\begin{enumerate}
			\item The \gls{EU}, until at least 2012, exercised most macroprudential oversight and otherwise mostly regulated financial markets at the \gls{MS} level. 
			Here, even the regulatory arm of the mixed economy was impaired, and regulations might have been arbitraged to suboptimal levels. %find source.
			\item Monetary policy drives bank lending. 
			The \gls{ECB}, because it can set only \emph{one} monetary policy, was unable to react to credit bubbles in individual markets or regions, such as Spain or Greece. 
			\item Equivalently, if there were in the \gls{EMU}, or ever hoped to meet the its nominal convergence criteria, deficit and credit-crazed \gls{MS} also could not devalue their currency through monetary interventions.
		\end{enumerate}
	\end{enumerate}
	
	\item[Asset Bubbles \& Crashes.] Broadly similar, and often concomitant to credit bubbles, asset bubbles can also fuel trade deficits. 
	As some assets in the deficit economy are persistently overvalued, foreign investors buy up these domestic assets, offsetting the trade deficit on the current account with a capital inflow on the capital account. 
	Real estate, stock or some other asset that did not previously exist, or belonged to domestic investors, changes hand to foreign investors, expecting an ex-post unreasonable return. 
	Come the day of reckoning, asset prices plunge, and much the same process sets in as when credit bubbles burst, only in asset bubbles, the default incidence is on the foreign investor, because she will usually, if not always, have taken risk-bearing equity in the asset. 
	
	Asset bubbles, too, are a redistribution from a future day of reckoning to a manic present, and, in that future, a redistribution from foreign investors to the domestic economy. 
	In addition, asset bubbles also waste welfare: 
	when they burst, they spiral downwards, often cause grave systemic risk and generally divert the economy from the long-term growth path.
	
	Asset bubbles, too, are a universally looming market failure, but the \gls{EU} is particularly vulnerable, again, because of likely regulatory arbitrage and ill-fitting monetary responses to local business cycles.%add href.
	
	\item[Monetary Expansion \& Inflation] Overly expansive monetary policy can also enable unsustainable trade deficits. 
	In this scenario, central banks simply inject more fiat money into the economy to offset the current account deficit with a Potemkin inflow of capital. 
	Fiat money, of course, never creates capital, and if, when and to the extent that this bluff is called, inflation ensues: 
	money supply and demand equilibrate at new, higher price levels. 
	
	This problem is widespread in the \gls{EMU} with asynchronous business cycles, a single interest rate target and no transferred stimulus to speak of. 
	In those regions where a low interest rate pumped too much money into the economy, as now appears to have been the case in Ireland, Spain, Portugal and Greece preceding the 2008ff crisis, loose money silently credit and asset bubbles, and might have already built yet-to appear inflationary expectations. %I really don't know what I'm talking about, here
	%Need data.
	
	Inflation, here, as always, wastes resources and redistributes arbitrarily. 
	%add /href.
	The middle class and older people, with frequently nominal denominated assets (pensions), but real denominated liabilities (rents) will be particularly vulnerable to whichever level of inflation this crisis might, eventually, bring.
	
	Inflation, too, as the other temporary diversions from the long-term growth path, redistributes from the future to the present. 
	Even inflation does not spiral to double-digits or more, \emph{any} additional increment in medium-term inflation and expectations is costly, as disinflating to previous levels is painful and often causes prolonged unemployment. %add source. %reference 70s disinflation
	
	\item[Dissaving \& Depletion] Trivially, economies can also go into unsustainable  trade deficits by real dissaving. 
	Instead of, say, selling shares in domestic companies, the deficit economy can just burn more of strictly limited fossil carbon, diminishing its real, if not its nominal assets.
	
	Because such real assets, including an economies infrastructure, demography, environment or CO2e levels as unresolved commons have no defined ownership rights, they do not nominally show up in the capital account of an economy. 
	Whatever this assets are transformed into, however, may well show up as an export in the current account. 
	For example, a deficit economy can dig into its coal and iron ore resources, transform them, and export them as steel, offsetting other imports. 
	Because the dissaving in natural resources is not usually recored, and thus triggers no change in the capital account, the steel export revenue will erroneously be attributed as a domestic \emph{income} in full, when in truth, much of the revenue comes from dissaved domestic assets, that ought to be recorded on the capital account.
	
	Such dissavings --- by definition --- redistribute from the future to the present. 
	As dissaving most easily and nominally invisible occurs out of vulnerable commons, it also wastes the welfare of some of our most precious, communal resources. 
	If, when and to the extent that they are depleted to sustain trade deficits, we may never or only at great cost be able to restore them.
\end{description}%might need to strengthen the crisis and imbalances components in the above.

I cannot marshall evidence here to show how each of these dynamics caused the 2008ff Euro, let alone the broader sovereign debt crisis. 
Nor can anyone, in 2012, reliably predict which of imbalances may yet turn out to be unsustainable, and why. 
What I do claim here is that whatever the actual imbalances and crises of the embattled \gls{EMU} and \gls{EU} are, or will be, they will, underneath it all, follow these scripts. 
Using some economic imagination, \emph{these} are imbalances and resulting periodic crises we would expect to plague any such internally open, but dysfunctional mixed economy.

While this european mixed economy may be of its own kind, the market failures that enable these imbalances and trigger the resulting crises are in no way \emph{sui generis}. 
The herding and information externalities that inflate asset and credit bubbles, the systemic chain-reactions that loom on large defaults and the tragic commons depleted by real dissavings are the kind of market failures that plague all real-existing capitalism. 
As such, they must be meet the appropriate regulatory, fiscal, and --- to a lesser extent --- monetary responses. 
%add \href
Similarly, loose money tempts governments of all market economies, not just --- in fact, probably, least of all --- in the \gls{EU}. 
It, too, must everywhere be curtailed by policy: 
a constitutionally-enshrined monetary governance, preferably a robustly independent central bank, bound to a well-defined goal. 
Because as dangerous drugs endemic to capitalism, these problems are not European problems, they also do not require a European solution. 

Still, the deficient european acquis exacerbates the imbalances and crises looming everywhere, in at least three ways (echoed by \citealt{Bordo2011}: 25):
\begin{enumerate}
	\item Without a monetary union or nominal convergence criteria thereto, trading economies can intervene in their exchange rate, or, at least, let their currencies depreciate freely. 
	As adjustment mechanisms, none is as fast as currency devaluation to get out of trade deficits. 
	In an instant, imports become more expensive and exports become cheaper, ideally, until import and exports equilibrate at the free exchange rate. 
	Alternative --- and ultimately equivalent --- domestic readjustment of (higher) prices and (lower) wages often takes longer, maybe too long to avert a \gls{BoP} crisis. 
	
	Discretionary exchange rate interventions are difficult to get right, and easily deteriorate into competitive devaluation, or beggar-thy-neighbor by a fancy name. %add source. 
	Freely fluctuating exchange rates, in turn, are costly as the second theory of optimal currency areas reminds us, %add source
	and they, too, might be the result of herding or otherwise failing global currency markets.
	
	As drug addiction therapies goes, the methadone of devaluation is, at best, a mixed blessing.
	
	And still, it is a prescription, the european economy has to do without, no matter the indication. 
	Within the \gls{EMU}, everyone in the \gls{EU} who ever wants to join, currencies cannot fluctuate. 
	To readjust, member economies can only hope their wages will not be too downwardly sticky.
	
	\item Creditors and debtors alike will anticipate the systemic risk and spillovers that the monetary union bestows on all its members. 
	They know that other members too, would suffer from defaults or, related, \gls{EMU}-exit, and, therefore, will likely bail them out. 
	%add more stuff here, why is that so? I wrote that somewhere.
	With systemic default risk as a union-level commons, but decisions in individuals, firms and, at best, \gls{MS}-hands, credit everywhere, but particularly in the high-risk economies, will be too loose.
	
	The \gls{EU}, in other, metaphorical words, is not only plagued by powerful and addictive drugs, but dealers and addicts alike can reasonably expect to be saved --- as the should --- if they overdose.
	
	\item Lastly, and familiarly, the european mixed economy lacks the fiscal means to otherwise rebalance internal demand, that intact mixed economies use to mitigate regional imbalances, including public works, industrial policy, or even straightforward transfers.
\end{enumerate}

The imbalances that have built up over the last years of European integration, and the crises in which they now seem to erupt, tell of the market failures of our capitalist economy. 
But they also betray the underlying dysfunctions and unfairness of an impotent mixed economy, that built these pressures in the first place. 
To bemoan only the market failure, and to seek to redress it is as naive as it is dishonest. 
Even worse, to simply wish away the crises, and to blame someone (``the banks'') --- \emph{anyone} (``'the markets'') for our misfortune, is to shoot the messenger, rather then to heed her warning.

In drug policy, if you are faced with a rampant substance abuse, you have to follow \cite{Mills-1959-aa}, and sociologically re-imagine the saddening observation of an overdosed corpse: 
you have to ask how, and why, people socially turn to harmful drugs in the first place, and then, if you can, cure this anomie, whatever it may be. 
If you only wage a war on drugs, they will always win.

And so it is with the imbalances and crises facing the \gls{EU} today: 
we have to use our economic imagination to explain how, and why, the european economies turned to delusional market failures in the first place, and, if we can, strengthen them to resist any siren call. 

Our anomie, now, should be clear enough: 
it is the underfunding, unemployment and inequality left untouched by an impaired command arm, that slowly, but steadily, unravels the social contract of the mixed economy. 
Faced with such pressures, it is little wonder that individuals, firms, states and the household-writ-large they collectively make up turn to the sirens of delusional growth.

Boxed in, as it is, the dysfunctional mixed economy, and especially its poorer constituents, find ways to relieve such economic pressure \emph{somewhere}, to postpone such austerity to \emph{somewhen} and disguise such anomie \emph{somehow}. 

To now, as many do, deplore only the failing markets\footnote{
	For some social science that seems to physically placing ``markets'' in quotation marks, see, for example \citealt{Beckert2012}. 
	In lieu of an economic, or social scientific explanation of failing or corrupted markets for sovereign debt, this rhetorical device works to distance the social scientist from these market messengers, as if their price signals were merely social constructs. 
	There is, however, such a thing as objectively given, materially tangible, unsustainable debt that higher interest rates might merely communicate (\citealt{Wihlborg2010}: 55).
	 ``Punctuation'', in any event, does not replace an explanation.}, 
to demonize investors or politicians is an act of exorcism. 
It was \emph{we}, who made a Faustian bargain with these devils: 
to let them reign free, if only they could numb the economic pain. 
They obliged us. But no one, as Doctor Faustus, should be surprised if some day, there is hell to be paid.

\subsection[An Old Deal]{An Old Deal} %or: why it matters

\begin{quote}
	\emph{``We can't start another new deal.''\\
	``How about fighting for the old one [...]?''\\}
	 --- The West Wing (Season 5, Episode 5), created by Aaron Sorkin.
\end{quote}

To insist on an intact mixed economy is not so innovative. 
The mixed economy is, in fact, a very old deal, prepared by the social reforms of Chancellor Bismarck, forged by Presidents Roosevelt and Truman, institutionalized by Lord Beveridge and, with miraculous success, reactivated in war-torn Germany, by Chancellors Adenauer and Erhardt. 
If there is such a thing as a European social model, or really, \emph{any} capitalist social model, it is the mixed economy.

To insist on an intact mixed economy is also not radical. 
The mixed economy, is, at heart, a compromise of exchange and command, of market and state, of individual freedoms and duties, of efficiency and equity. 
The mixed economy hopes not for an end of history, nor harbors any overhaul of society: 
it makes amends with capitalism.

Tax --- the cornerstone of the mixed economy --- in particular, is a reformist, never a revolutionary project. 
Good taxation, especially of consumption, accepts private property as given, even legitimate and desirable, and merely adjusts the sticks and carrots that people reap for their personal enjoyment. 
Minimizing their \gls{DWL}, good taxation maximizes the freedom of all market participants to do as they would \emph{absent} the tax.

Today, Europe is reneging on this old deal. 
Without union-level taxation to speak of but with full factor and goods mobility, tax schedules are compressed and levels lowered. 
Without fiscal complements, the common currency allows imbalances and lets diverging business cycle fluctuate widely. 
Even regulation is yet incomplete, such as in labor market legislation, and is arbitraged away by competition. 

On these institutions rides it all: 
that we can efficiently produce public good cancer research or preserve our environmental commons, that we can pool the risks of the healthy and the frail, that we can restore some fairness between market winners and losers, that our children will receive at least what we have received, and that our neighbors prosper, too. 
%add hrefs
Tax especially, and together with regulatory and monetary institutions, \emph{are} the social contract of modern capitalism. 
%add source

European regional integration does not rewrite the social contract, but sins by blithely omitting much of passages on tax, monetary and regulatory policy. 
Absent them, the european polity is no longer free to choose between %add old graph
command and exchange, but --- without explicit popular consent --- defaults to ever more market, and ever more present consumption. 
Our household-writ-large now occupies a greatly constrained coordinate space (figure \ref{fig:coordinate_space_constrained}), boxed in by chronic underfunding,  looming unemployment and rampant inequality, shaken by recurring imbalances and crises. 

\begin{figure}[htbp]
	\begin{center}
	\includegraphics[width=1\textwidth]{./img/coordinate_space_constrained}  
	\caption{Constrained Coordinate Space of a Dysfunctional Mixed Economy}
	\label{fig:coordinate_space_constrained}
	\end{center}
\end{figure}

By its dysfunctional design, the european mixed economy yields ever more to markets while the other part of the mixed economy, the state, is on the retreat. 
Bereft of their old, flexible and capable social contracts, the acquis will, nay, already \emph{has} --- however fortuitously --- remade european society in the neoliberal, consumerist image. 
``Neoliberalism'' and ``consumerism'' are, in this case, not catch-all labels of disaffection, but I choose them with equal anger and care, and they apply precisely. 
The aquis is:
\begin{description}
	\item[Neoliberal,] because, eviscerating the state, it morphs markets from one of several \emph{means}, to inescapable \emph{reality} or even ultimate \emph{end}, neither of which it is, nor should be.
	\item[Consumerist,] because, it cannot set a positive savings rate, and leaves austere members no choice but to loot real savings, and give into the temptations of bubbles. 
	As a result, much of the resources of the communal household will be devoted to near-term consumption\footnote{
		The \gls{EU} commission is quite explicit about this:
			\begin{quote}
				\emph{``The Single Market Review put \emph{(sic!)} citizens, consumers and \glspl{SME} at the centre of policy-making.''}\\
				--- \citealt{Commission2008}: 3	 
			\end{quote} 
		One wonders, at least, why, in addition to \emph{citizens}, consumers and \emph{some} (though not other) firms are also mentioned, when, in a functioning market, the latter two should be served only as proxies of the ultimate beneficiary, the citizen.}.
	
\end{description}

Just how angelic the postwar mixed economy really was, I do not know\footnote{
	To mention just a few red flags, it is unclear what costs Postwar prosperity extracted from others (dependence or world systems theory), we do know whether or how Western affluence can be repeated without the same gigantuan carbon footprint and we worry whether broad-based growth in value-add is but a historical episode (cost disease).}. 
Still, \emph{relative} to 19th century mass poverty \citep{MarxEngels-1848-aa}, 20th century ``mob politics'' (\citealt{Crouch2004}: 158) and the spectre of 21st century neo-laissez-faire, the mixed economy, and the welfare state it has enabled, appears as singular achievements of the modern era. 
The economic institutions of Postwar Western Europe have brought about unseen prosperity and equity, all in relative peace and freedom. 

That is an old deal worth fighting for.

\subsection{Catch-Up}
\begin{quote}
	\emph{``Three tomatoes are walking down the street --- a poppa tomato, a momma tomato, and a little baby tomato. 
	Baby tomato starts lagging behind. 
	Poppa tomato gets angry, goes over to the baby tomato, squishes him, and says: catch up.''} \\
	--- \href{http://www.youtube.com/watch?v=5D_QKY0_Bxk}{Mia Wallace in Pulp Fiction (1994)}
\end{quote}

	%Another perspective in favor of EU tax competition construes different tax rates as an essential component of the comparative advantage at play in European integration. 
	%There are different variations on this argument.
	%László Kovács (2004), then Commissioner for Taxation and Customs asked that tax rates must be allowed to differ 6-8% simply to make up for the remote location of some markets.
	%More broadly speaking, tax competition could be construed as convergence "through the back door", enabling small and relatively poor countries to catch up faster than they otherwise would-

	%Parallel to the optimum currency debate, non-harmonized, competing taxes can serve states to more flexibly respond to asymmetric shocks. 

	%Lastly, it could be argued that the EU, given its wide economic opening to the rest of the world, is playing a Prisoner's Dilemma at another, global level, too. 
	%Even suboptimally low tax rates would then be individually rational for the EU to pursue, and internal tax competition could be regarded as either inconsequential or desirable to "defect well".


	%Tax harmonization has dramatic redistributive potential between the union's member states. 
	%Construing tax competition as convergence through the back door points in this direction already: small and poorer countries stand to loose from tax harmonization, as evidenced by their opposition to EU tax cooperation (Kellermann & Kammer 2009: 138).
	
	
	%In an extremely asymmetric configuration, it may be the case, that tax coordination becomes a zero-sum (rather than positive-sum Prisoner's Dilemma), where harmonization only redistributes wealth to the richer and larger states.

	%Harmonized, more progressive tax rates across the European Union 	will reduce the competitive advantage of emerging markets. 
	%This is an unavoidable redistributive trade-off.
	%A feasible tax harmonization would still require different real tax rates, so as to account for  lower factor productivities in less developed economies. 
	%The ultimate goal for tax harmonization should be roughly equal post-tax factor prices, weighted by factor endowments with equal progressions everywhere in the Union. 
	%That would define the absence of any tax competition; given such a degree of harmonization, the EU (under autarky) could set arbitrary tax rates and progression schedules.

	%Such harmonized taxation - all other things equal - would slow down the growth of emerging markets dramatically, rendering it a similarly bleak outlook in terms of sustainability as race-to-the-bottom tax competition
	%Economically less developed member states are unlikely to ever agree to this.

	%To counteract this effect, a harmonized tax system must be accompanied by substantial fiscal redistribution between the member states, in other words: fiscal federalism, with large "eastwards" transfers to hasten convergence along. 
	%There are good normative reasons to favor tax harmonization. 
	%The continent's integration requires progressive schedules to strengthen social cohesion, further employment and redistribute the unions' fruits. 
	%The throes of transition, and building of a sustainable, prosperous (think: Lisbon Strategy) future require well-funded governments that can dampen shocks and invest wisely and substantially. 
	%Tax competition must be overcome, because it is both inequitable and inefficient in the longer run. 

	%Some of the arguments against harmonization-cum-redistribution are mistaken. 
	%Given widely available and sophisticated economic statistics and the experience of domestic fiscal federalism, some (surely imperfect) redistributive compensation does in fact seem technically doable. 
	%Moreover, harmonization-cum-redistribution does not necessarily imply a larger public sector overall: it is conceivable and preferable at any level of public spending. 
	%While it does imply a larger public expenditure quota particularly for NMS, this need not be equated with greater, potentially undesired provision of public and common goods. 
	%If preferred, electorates can choose to channel the side-payments to private consumption.

	%If tax competition and a move to regressive, immobile bases is happening, as it appears, it has very real consequences on the life-chances of the economically disadvantaged. 
	%Similarly, the growth trajectories of the NMS have very dire implications for the wellbeing of everyone, and particularly the poor in those countries. 
	%The downsides of both tax competition and harmonization have a social gradient, in other words. 
	%By naturalizing the inevitability of the status quo, this working group ignores the questions of distributive justice which ultimately underly any choice or non-choice of how to do economic liberalization. 

	%Moreover, by inducing more economic hardship, structural unemployment and fiscal imbalance, the very concerns about what the EU can reasonably achieve may put in danger the prospects of unification.

	%Harmonized taxation cum fiscal federalism will allow the European Union to grow, wide and deep, not just "united in diversity" but "integrated in cohesion".


Too easily, discontent with a retrenching Western European welfare state blames \gls{CEEC} ``social dumping'' %source
or otherwise slides into poorly disguised economic nationalism. 
It must not: 
whoever thinks that Western and Northern, otherwise intact mixed economies now whither away because of competition from the East and the South, misunderstands, or --- more likely --- misconstrues the issue.

Of course, the newly opened, low-productivity and low-capital markets in the poorer East and South have much lower, often proportional taxation, especially on capital incomes. 
They \emph{cannot} afford high, or progressive taxes on capital, if they are to remain competitive, let alone converge\footnote{
	For example, L\'{a}szl\'{o} Kov\'{a}cs (2004), then Commissioner for Taxation and Customs asked that tax rates must be allowed to differ 6-8\% simply to make up for the remote location of some markets.} 
Had Romania the capital income taxation of pre-reform Sweden, the net returns on any investment would fall tremendously. 
Given the low labor and total factor productivity in Romania, an investment might not be profitable at all, and capital might instead stay in high-tax, but also high-productivity Sweden. 
Romania would not be able to attract much foreign capital, which it so direly needs to leapfrog into convergence.
	
By contrast, the high-productivity, high-capital markets in the West and North, \emph{could} afford higher, and more progressive taxes, if only they did not face tax competition from all other \gls{MS}. 
Had Sweden the capital income taxation of post-accession Romania, or even the 0\% \gls{CIT} of non-member Moldova \citep{Piatkowski2008}, it would have to abandon much of its mixed economy. 
Forced to finance its large public sector only out of labor, or other immobile, proportional sources, the tax wedge on the lower and middle strata of society would become unbearable. 
%add \href

At first sight, there appear to be only two policy responses to this conundrum:
\begin{enumerate}
	\item \emph{High European-Level Taxation.} \gls{MS} might agree to enforce union-wide, uniformly high and progressive taxes on mobile bases, that would save welfare in the rich \gls{MS}, but at the cost of laggard growth in the poor capital-deprived \gls{MS}. 
	Such tax cooperation would save the welfare state, but sacrifice fast, or even any, convergence.
	\item \emph{Tax Competition.} \gls{MS} might continue with the status quo ante, set tax rates individually and let competition whither away levels and progressivity. 
	Such tax competition would sacrifice the welfare state, but bring some convergence ``through the back door'' as rich \gls{MS} capital will to rush to poor \gls{MS}: 
	poorer (and smaller) countries win under tax harmonization, as evidenced by their opposition to even nascent \gls{EU} tax cooperation (\citealt{Kellermann2009}: 138)\footnote{
		Capital inflows will be particularly effective in relatively poor economies. 
		As these economies, supposedly, still lie far below the golden rule of saving \citep{Solow1956}, returns on capital will be much higher than in richer economies where further (near-Solowian) capital deepening faces diminishing returns (e.g. \citealt{Barro1995} or \emph{ibid.} 1992 as cited in \citealt{Beckfield2009}: 3).}.
\end{enumerate}

Such are the alternatives that politicians might peddle, pitting poor members against rich, convergence between them against welfare within. 
These are, as so often in defunct mixed economies, impossible choices to make, seemingly forcing European democracies to play a game of zero-sum.

Alas, regional economic integration, as all trade, is not zero-sum game: 
there are gains to be had, if not accruing to everyone equally. 
Sweden is better of for having a prospering, converging Romania as a trading partner, with different absolute or comparative advantage, capital endowments and areas of specialization. 
Out of the proceeds of this positive-sum game, there is a third option:
\begin{enumerate}
	\setcounter{enumi}{2}
	\item \emph{High European-Level Taxation With Side Payments.} \gls{MS} agree to enforce union-wide, uniformly high and progressive taxes on mobile bases, but rich \gls{MS} recompense poor \gls{MS} for their ensuing competitive disadvantage. 
	Absent different taxes rates on mobile factors, the union as a whole is able to choose freely amongst the tradeoffs of a mixed economy, including a well-funded welfare state. 
	Still facing low labor productivities, poor \gls{MS} receive fiscal transfers that they can use to build capital stock, educate their workforce, upgrade infrastructure, or even subsidize investment. 
	For example, poor \gls{MS} might lower or even abandon taxation of some or all immobile factors, such as low-productivity labor. 
	Facing a small or no tax wedge, maybe even receiving subsidized or free health care, or some other \gls{NIT} of sorts, the low-productivity workers in poor \gls{MS} would be able to accept yet lower gross market wages, and become competitive again, even as investors have to pay high taxes\footnote{
		To remind readers that this is not, in fact, what the acquis currently stipulates: 
		The overall structural and cohesion funds for 2007--2013 amount to no more than \texteuro 347 billion, less than 3 times the budget of the city of Berlin (\texteuro 21 billion in 2009), or about 0.005 \% of the budget of Germany (\texteuro 1,164,000 billion in 2011). 
		In addition, \gls{EU} budget negotiations are still marred by \emph{juste retoure} attitudes, with \gls{MS} wanting to get paid out what they have received --- the very opposite of a redistributive regime (e.g. \citealt{Begg2008a})}.
	
	Side payments need to be initially large enough but decrease over time, and they must be used wisely by poor \gls{MS}. 
	As in any subsidy or  protection, it will be difficult to avoid long-term dependence and rent-seeking.
	
	If we get it right, though, high european-level taxation with side payments will save the mixed economy, enable a potent welfare state, clear factor markets, converge living standards in the union and keep us all on the long-term growth path. 
\end{enumerate}

%add game theory here. 
%Is there a game to be played? %create a table about the options in the catch-up game.

We need not trade off openness and convergence \emph{between} \gls{MS} with efficiency and equity \emph{within} them. 
In fact, they are the same issue: 
behind the pressures on rich welfare states always lurks the question of convergence: 
just how fast, and by which means, do we want the poor members to catch up?

If we opt for policy response \begin{inparaenum} 
	\item high european-level taxation, convergence will be minimal and through the market only. 
	If \item we choose tax competition, convergence will be faster, and still only market-led. 
	If we \item adopt high european-level taxation with side payments, convergence can progress at an arbitrary speed, with a substantial state component. 
	\end{inparaenum}

Deciding on the speed of the catching up may just be the deeper question of european integration that many politicians of all colors, but especially those tending to economic nationalism, may seek to avoid by misrepresenting european choice in dichotomous terms of more or less protection, more or less integration. 
Convergence supported by side payments \emph{does}, of course, include a zero-sum component, as any redistribution by definition does. 

But it is not, as it is easily made out to be, redistribution \emph{between} one economy and another, it is instead, redistribution \emph{within} one, European economy. 
Whenever the markets expand beyond borders, so, too, the abstractions of the mixed economy rule at a higher level, and they apply not just to tax --- the clearest case that I have here illustrated --- but to all command interventions prone to arbitrage: 
as the first good, service, or capital equipment crosses the border, the two trading partners have already adopted \emph{one} of the three regimes of economic integration, and, related, have made their however implicit choice of just how, and how fast, the poorer party gets to catch up.

The elephant in the room of european economic integration is this: 
if we are to save the mixed economy, we have to explicitly decide, not just agree by default, how fast we want to converge \footnote{
	It too, is neither new nor revolutionary: 
	internally heterogenous mixed economies have always, with varying results, dealt with this question. 
	In the US, much of the controversy about the role of the federal government boils down to the question of how much rich Texans should dole out to poor Louisianans. 
	In Italy, the rift is between North and South (e.g. \citealt{PutnamLeonardi-1993-aa}) and in Germany, initially between industrial and rural states, since 1990 between old and new \emph{Laender}.}.

\newpage

\section{Who Dunnit?: Second-Order Theory of Negative Integration} \label{sec:who_dunnit}

So, a reader may ask, what does any of this have to do with who, or what  ``built welfare states in post-communist \gls{CEEC}'', or alters them elsewhere in the \gls{EU}? 

The defunct mixed economy of the \gls{EU} has everything to do with the continents welfare states, not just because in the East, the \gls{EU} (and \glspl{IFI}) might have --- and still do --- dictate economic policy to any country who might ever wish to join the union through convergence criteria, accession negotiations and conditionality (\citealt{Bonker2006}: 55), but because, much, much more fundamentally, any economic or social policy adopted in \glspl{CEEC} after 1990, or elsewhere, is greatly constrained by the European economic regime in place. 
\gls{EU} regional economic integration, to be sure, only dictates what cannot be done: 
no progressive or high taxation, no independent monetary policy, and, sometimes, no effective regulation. 
%add hrefs
European integration, in \citeauthor{Scharpf1997}'s influential formulation, is \emph{negative integration}, but nonetheless consequential \citep{Scharpf1997}. 
If there is anything to the otherwise muddled empire thesis \citep{BeckGrande-2007-aa} it is this: 
cripple your mixed economy and settle for the promises of neoliberalism and consumerism --- or perish in autarky.

I do not know, nor claim to explain here, who committed this ``perfect crime'' \citep{Galbraith2002a} --- but murder of the mixed economy, it was, and that is the deed that any second-order theory of social change has to explain.

\subsection[Literature]{The Testimony: Critique of the Literature} \label{sec:Literature}

\begin{quote}
	%\emph{``Abstraktion, das Denken in Systemen wurde durch das jeweils Unmittelbare, das Pragmatische ersetzt. 
	%Damit haben sich die sozialdemokratischen Parteien Zug um Zug desavouirt.''}\\
	\emph{``Abstractions, thinking in systems has been replaced by the immediate, the pragmatic. 
	This is how the social democratic parties have disgraced themselves.''}\\
	--- Herbert \citeauthor{Schui2009} (\citeyear{Schui2009}: 41)
\end{quote}

On that count, some of the literature falls short: 
it fails to bring the \citeauthor{Mills-1959-aa}ian sociological imagination and \citeauthor{Keynes1936}ian economic abstraction to bear on the nitty-gritty empirical data of \gls{CEEC} and other welfare reform, to even discern  the offense to the post-war social contract. 
%add footnotes to authors.

This blithe ignorance comes in at least four different flavors, that I here summarize as follows:
\begin{enumerate}
	\item \emph{TINA}-flavored literature holds that the mixed economy expired unavoidably and therefore, requires no further investigation. 
	For TINAs, the mixed economy is not murdered, but dies a natural death.
	\item \emph{Pangloss}-flavored literature accepts possible alternatives to death, but assumes that, axiomatically, it was the best possible course of action, and therefore acquits any possible suspect. 
	For modern-day Panglosses, the defunct mixed economy is collateral damage to a worthy cause.
	\item \emph{Newsspeak}-flavored literature shrouds itself in impermeable and misleading language. 
	For newsspeakers, the mixed economy is not dying, but just ``restructuring'' into less complex organic compounds (known as ``decay'' in oldspeak).
	\item \emph{Bystander}-flavored literature abstains from judgment, or even conclusion. 
	For bystanders, it may or may not be murder, but it's none of their business. %add hrefs to below sections
\end{enumerate}

\subsubsection{TINA}

\begin{quote}
	\emph{``There [\ldots] is no alternative.''}\\*
	--- Margaret Thatcher (London, 1980)
\end{quote}

TINA is hardly observed directly in the literature, if only because, as a political strategy, it aims to ``veil the essentially political character of political decisions'' (\citealt{Bluhdorn-2007-aa}: 314), and therefore, will not even explicate itself. 
The purported lack of alternatives, mostly, asserts itself not by negating existing alternatives, but by ignoring them.  

Evidence of TINA, therefore, lies mostly in conspicuous omissions, the most glaring of which is that no literature I am aware of even acknowledges a hypothetical scenario in which \gls{EU} trade liberalization co-occurs with strong redistributive and stimulus policies, and how such wide \emph{and} deep integration might have played out. 
Such a mode of concurrent economic and political integration even has a recent historical precedence of integrating a \gls{CEEC}: 
German reunification --- however flawed in its implementation, or disappointing of its promises --- not only introduced economic and currency union in 1990, but, simultaneously, politically integrated the new L\"ander and launched a massive fiscal expansion, including an expressly redistributive ``solidarity surcharge'' tax. 

TINA is also elusive, because it has become so successfully omnipresent, that we hardly recognize it anymore, where it lingers. 
For example, when \citeauthor{Kovasc} speculates that ``Eastern Europe may be unlucky (...) if it is confined to imitation'' of Western European welfare states, because in the future ``these regimes will probably not produce the same performance levels as they do today'' he silently implies that such doom would be inevitable, which clearly, to a social scientist, it must not be, for it would not require any social science if it were, in fact, akin to a law of physics.  

Proponents of Western welfare states can also be conspicuously silent on some true alternatives, and tradeoffs. 
Social-democratic \cite{Scharpf1997} accepts that ``the economically less developed \gls{MS} simply could not afford [\ldots] the same level of welfare [\ldots] [as] the highly developed \gls{MS}'' but, oddly, ignores its alternative: 
transnational side-payments alternative in a european transfer union (\citeyear{Scharpf1997}: 26). 
This silence might have been (mis)read by \citeauthor{Moravcsik-2002-aa} (\citeyear{Moravcsik-2002-aa}: 
619) as \citeauthor{Scharpf1997}'s supposed preference for domestic over transnational redistribution. 
Surely, \citeauthor{Scharpf1997} is not an economic nationalist, but he does seem to throw up his hands as he concludes that \emph{either} rich welfare states or emerging economies must perish, and that, in any event, harmonization would be ``probably impossible'' (\citeyear{Scharpf1997}: 26). 
That is the kind of despair that TINA always inspires, because it renders invisible the magnificent ability of intact mixed economies to smooth over even such epic transformations. 
It is strange and confusing that \citeauthor{Scharpf1997} --- who talks about (fiscally subsidized) German reunification (\emph{ibid.}) ---would not even mention the tax-cum-sidepayment solution to european integration, and this slight unnecessarily impairs his otherwise clear analysis. 
Ominously, \citeauthor{Scharpf1997} seems to have made amends with TINA when he commands ``\emph{normative} (sic!) political theory as well as political practice [to] come to grips with the conditions (sic!) of `democracy without omnipotence' '' (\citeyear{Scharpf1997}: 29)\footnote{
	As happens when you fraternize with TINA, (some of) the policy prescriptions fall short. 
\citeauthor{Scharpf1997}, for example, advocates wage earner funds (pay in equity) to counteract the shifting terms of trade between capital and labor, a mostly notional innovation that might still imply real wage cuts and push more risk on possibly unwilling or unable workers. 
He also promotes a shift to consumption taxes in the face of crumbling income and corporate tax revenues, a move that, if those taxes are prepaid, and therefore regressive will burden workers, low- and middle-income earners with the load of welfare state financing. 
Most surprisingly --- and nonsensically --- he also seems to ask for a privatization and voucher model of health care (and other risks!), that, aside from all the possible market failures and distributive vagaries, will not in itself right the welfare state. 
Finally, \citeauthor{Scharpf1997} even falls for the old social-democratic smoke grenade of ``parity'' contributions to social insurance, nominally shared by employees and employers, but whose incidence really always falls entirely on labor (all of the above \citeyear{Scharpf1997}: 30-34)}.

Sometimes, the TINA fallacy reveals itself with no shame, for example when \cite{Grow2005}, wondering how to strengthen the social safety net, ask that countries maintain programs ``within the available resource envelope'' (\emph{ibid.}: 39). 
Not only is this quite tautological, but worse, it seems to suggest that --- as the flight envelope of a plane --- welfare states would meet quasi-physical boundaries in their ability to fight (in this case) poverty, which clearly, cannot be a satisfactory perspective for a social scientist. 
Poverty on Planet Earth, and certainly in Europe is not a matter of absolute scarcity, but of distribution, and therefore, requiring of a social, not physical explanation.

Nitpicking such shortcomings from isolated arguments is exactly what critical social science should do, but readers might here get the wrong impression that I am questioning the seriousness of purpose or integrity of the aforementioned, and other deserved scientists, or worse, that I would harbor some kind of conspiracy theory. 
To avoid such false impressions, I further illustrate the fallacies of TINA in just \emph{one} area of \gls{CEEC} and Western welfare reform: 
pensions. 
%add footnote on footnotes.

\paragraph{Pensions} \phantomsection \label{sec:pensions}
Providing for old age is everywhere considered a key goal of welfare states. 
The abstractions of pensions are actually very easy. 

 \begin{table}[htbp]
	\centering
	\includegraphics[width=1\linewidth]{./img/pension_design}  
	\caption{Pension Design}
	\label{tab:pension_design}
\end{table}%might add family as a third column here.

As summarized in table \ref{tab:pension_design}, all possible pension designs are exhaustively described by tabulating four simple choices:

\begin{description}
	\item[Who Should Manage it?] The surplus production saved for old age can either be managed publicly by the state, including its quasi-fiscal institutions, or it can be managed privately by capital markets (see the the middle two columns in table \ref{tab:pension_design}). 
	In both cases, savers acquire some kind of ownership claim to the accumulated surplus: 
	if under public management, savers own their savings as \emph{entitlements}, governed by public or administrative law, if under private management, savers own their savings as \emph{property}, governed by private law\footnote{
		In \citeauthor{Barr2005a}'s precise formulation, ``PAYGO and funding are both financial mechanisms for organizing claims on that (future) output. 
		(...) 
		Funded schemes are based on accumulations of financial assets, PAYGO schemes on promises'' (\citeyear{Barr2005a}: 157).} \footnote{
		In addition, the right of current savers to future pensions can take the form of \emph{equity} or \emph{debt}. 
		As equity, in either private stock or public economy-wide growth, savers partake fully in both the upside and downside risks: 
		if either the stock, or the economy as a whole grows or falls, so do their pensions. 
		For example, mutual funds share risks under private management and a strict defined-contribution PAYGO-systems share risks under public management. 
		As debt, in either private bond markets or entitlements to future economy-wide incomes, savers are insulated from all but the risk of default. 
		For example, private life insurance includes only a risk of default, and a strict defined-benefit PAYGO-system --- at least nominally --- carries only the risk of sovereign default. \\
		In publicly managed pension schemes, the distinction between the risk profile of equity and debt is often muddled, as future, \emph{nominally} defined-benefit returns are sometimes partly indexed to demographic change, labor incomes, inflation, economic growth or other, risky macroeconomic factors. 
		If and to the extent that such pension schemes turn out to burden future pensioners, instead of future taxpayers or ``social insurants'' with the shortfall, they become de-facto equity investments. \\
		The distinction between equity and debt risk portfolios to publicly managed pension systems carries only so far: 
		in contrast to equity in corporations, equity in sovereigns --- at least ideally --- is always ``non-voting stock''. 
		Conversely, sovereign defaults are different from corporate defaults: 
		in democracies, bondholders \emph{cannot} turn into shareholders and ``take over'' as they would in an insolvent private corporation. 
		The default risk of sovereign bonds is qualitatively different: 
		if a publicly managed pension scheme defaults on its obligations, pensioners can change policy only in their capacity as voters, but, aside from basic rule of law protection, not in any \emph{additional} capacity as bondholders. 
		The bottom line is: 
		if push comes to shove, in publicly managed pension schemes, whether you are a shareholder or a bondholder, your power to change outcomes are mostly those of ordinary voters.}.
	\item[When to Save?] Pension schemes can either be pre- or post-funded. \\
	When they are pre-funded, present surplus production is coagulated into capital \emph{before} future consumption exceeds future incomes in old age. 
	On introduction, current workers bear the initial incidence. 
	Pre-funded pension schemes can be, for example, managed by the state as \glspl{SWF}, or provided by markets as life insurances or other annuities.\\
	When they are post-funded, future income-exceeding consumption in old age is paid out of \emph{future} surplus production of other, younger people. 
	On introduction, future workers bear the initial incidence. 
	Post-funded pension schemes can be, for example, managed by the state as PAYGO, or provided privately in families, or as corporate pension funds\footnote{
		While practically useful, the distinction between pre- and post-funded is mostly nominal, and does not always, in fact, correspond to real savings quotas of an economy. 
		\\
		If pre-funded capital is nominally invested in asset bubbles or other overvalued junk, much of the hard-earned surplus production may actually go to waste and coagulate little in the way form of capital that will actually be valuable, or generate a return in the future. 
		\\
		Conversely, if people are free from such post-funded obligations, but instead use their surplus production to have more children, educate them better, built a home or a company, they \emph{have}, the economy as a whole, does in fact accumulate capital. 
		\\
		In this sense, the current support for pre-funded schemes is based on a false sense of certainty: 
		whether pre-funded pension schemes \emph{actually} stash away enough for old age depends on how good the investments are. 
		The next question, of \emph{How to Grow} is therefore a more meaningful choice than the tiresome pre-funded vs. post-funded controversy.}.\\ 
	\item[How to Grow?] \emph{Any} pension scheme needs to coagulate surplus production somewhere, somehow. 
	Excluding exogenous growth, the long-run output of an economy is determined by the size of its workforce, and the productivity of its workers, which is in turn determined by the different forms of coagulated surplus production they command as human or physical capital, or endogenous technology. 
	Creating both these drivers of output --- people and their productivity --- is costly. 
	The hyper-materialist connotations notwithstanding, creating people or the basis for their productivity are also alternative, competing uses for the surplus of an economy. 
	People and economies can use their surplus production to feed a second baby, or to build a room for the first baby (physical capital), or to better educate the first baby (human capital). \\
	Different pension regimes suggest, and \emph{face} different allocations to generating new workers, and more capital. 
	On the public side, a pension regime may invest surplus production --- pre-funded \emph{or} post-funded --- into better education, or a \gls{SWF}, hoping that either of those will pay off in increased output in the future. 
	Alternatively, a state may try to encourage, and/or individualize the costs and returns of rearing children in its family policy, to increase, or --- better yet --- stabilize future workforces. \\
	On the private side, a private investment may accumulate into physical capital, or family may pour resources into one prodigious child, hoping that both will increase future productivity. 
	Alternatively, a family may decide to have \emph{more} children, hoping that they together, will support the parents in old age.
	
	\item[Where to Invest?] Lastly, pension schemes can invest their surplus production either abroad and at home under an open economy regime, or, if under autarky, they can invest only at home. 
	Crucially, for private markets, the mix between foreign and domestic investments will be determined by expected returns. 
	Domestic investment can be enforced only if the economy in question foregoes capital mobility and confines itself to --- at least some --- autarky. \\
	These investment choices reflect in macroeconomic movements. 
	A pension scheme investing heavily abroad will first bring a trade surplus, and later, a trade deficit, both set off by respective flows in the capital account. 
	A pension scheme investing only at home will not alter trade balances, but will first build positive savings rates, and later negative savings rates, as the pensions are paid out and consumed away.
\end{description}

%add somewhere a good visualization about pensions, to illustrate the trivial difference between PAYGO and funded. 
%Go back to Barr to figure this out, he's got some good ideas. 
%There's a box in Barr with the economics of pensions that I should look into.

Each of these choices has to be considered very carefully. 
To name just a few of such vexing considerations, private management might help capital markets mature, and contribute to growth (e.g. \citealt{Barr2005a}: 155) --- \emph{or} it might help inflate bubbles, and expose individuals to undue risk. 
Nativist policy may be considered illiberal, \emph{or} raising children may be considered a positive externality for which parents should be compensated. 
Investing pensions abroad may be thought to spur growth and convergence, \emph{or} we might find find the colossal --- if ideally only temporary --- trade imbalances unacceptable\footnote{
	Private, pre-funded pension schemes in rich, open economies may cause much capital to flow abroad into emerging economies, where they promise to generate higher marginal returns. 
	In the future, these formerly trade surplus, aging economies such as Germany would then run colossal trade deficits with emerging economies such as Brasil. 
	While many advocate such a scenario (e.g. \citealt{Borsch-Supan2003}: 176), I have not read any one remarking on whether such a scenario in which, for example, young Brazilians do much of the producing, and old Germans do much of the consuming would even be conceivable, let alone desirable.}.

The devil, here, as always in policy, dwells in the details, and must be engaged. 
I allow myself the rather superficial summary in table \ref{tab:pension_design}, because I want to draw attention to the fundamental equivalences of these seemingly dramatic choices: 
however they answer these questions, pension regimes, as all policy, cannot escape the constraints of Haig-Simons, summarized in figure \ref{fig:haig-simons_individual_collective}. 
If, other things equal, you have fewer children and lower the production of future workers, as Germany presently does, but wish to maintain the standard of living of a future, older Germany and future, older citizens, you \emph{must} invest in either human or physical capital, in some form, by some mean. 
If,  other things equal, you cut public pension contributions, you have to increase privately managed saving.

The Haig-Simon identity and, as one of its terms, demographic change\footnote{
	The current, second demographic transition (\citealt{Davis1945}, restated by \citealt{Caldwell-1976-aa}) delivered low, often below-replacement level \glspl{TFR} --- the average number of children a woman would have by age 50 based on the current age-specific fertility rates --- and low mortality, concisely measured as life expectancy at birth (after 6 months). \\ 
	US 2009 estimated TFR: 2.05 \citep{CIA2009}, Germany 2009 estimated TFR: 1.41 \citep{CIA2009}, EU-25 2002 TFR: 1.37 (\citealt{Demeny-2003-aa}: 2). 
	Life expectancy at birth for EU-25 is 69 years for males and 78 years for females (\citealt{Demeny-2003-aa}: 2).}, 
are unforgiving and inevitable strictures, akin to the law of conservation of matter. 
Population aging\footnote{
	Falling fertility and falling mortality lead to two interacting effects: 
	the population \emph{shrinks} and \emph{ages}. 
	Pure shrinking alone could ideally be welfare neutral on a per-capita basis. 
	Such pure shrinking with no ageing component would, however, require \emph{rising} mortality as long as TFR is below-replacement level to offset older, larger cohorts.\\
	Ageing, or more specifically, a change in the dependency ratio between very young and very old transfer recipients and everyone else, however, is an unavoidable welfare loss. 
	Fewer people are available to produce for the consumption of more, older people.},
as other real dissavings, harbor unavoidable losses in future welfare (e.g. \citealt{Borsch-Supan2003}: 152). 
They are also both self-reinforcing: 
dissaved capital not only earns no interest, but also no compound interest, unborn babies not only cannot support their parents, but they also will not have babies, themselves\footnote{
	Demographic shocks echo on for generations and generations, as for example, 1950s baby boomers procreate in the 1970s, and baby-baby boomers procreate in the 1990s, and so on. 
	Population dynamics are so damningly powerful because, as \cite{Malthus1798} knew, it grows geometrically, that is, has an incredibly high ``interest'', and therefore compound interest rate.}.

To these strictures, there really are \emph{no alternatives}, no matter the pension design. 
There are, however, very real policy choices of whom to burden with the inevitable welfare loss: 
\begin{inparaenum} \item whether, and to which extent to burden current or future workers, \item whether, and to which extent to individualize the material costs and rewards of rearing children, \item whether and to which extent to tie individual contributions to individual benefits, and, most importantly, \item whether and to which extent to alter the incidence of the pension design through redistributive intervention, that is, taxation. 
\end{inparaenum}

TINA in the literature on building pension schemes in \gls{CEEC} or reforming them elsewhere, takes a peculiar form. 
It weighs alternative pension scheme designs, where --- with the exception of devilish details --- there really are no meaningful choices, and neglects those very real choices of burden-sharing that democratic polities have. 

\citeauthor{Cerami2009a}, for example, though ever critical of ``neoliberal reforms'', describes extensively the addition of compulsory or voluntary private pensions to \gls{CEEC} regimes and cites demographic change and --- unspecified --- ``economic and financial pressures'' (sic!) as partial reasons (\citeyear{Cerami2009a}: 336). 
He seems to forget that the intertemporal Haig-Simons identity of an economy is hardly affected by a privatization of pensions: 
sure enough, the incidence changes, but no old age crisis can be averted by privatization. 
\cite{Barr2005a}, by contrast, are one of the few authors in the field who --- at least implicitly --- endorse Haig-Simons, and exemplarily, reveals the policy options that TINA would have us ignore: 
\begin{quote}
	\emph{``An alternative approach [to parametric adjustments] seeks to finance higher future pension spending by reducing other expenditure. 
	One way is to reduce public debt now; thus governments in the future would spend less on interest repayments, freeing resources for PAYG[O] pensions''}\\
	--- \citeauthor{Barr2005a} (\citeyear{Barr2005a}: 152)
\end{quote}

\citeauthor{Cerami2009a} asks for a ``new politics of aging'', involving institutions, ideas and power as a new second-order theory of social change (\citeyear{Cerami2009a}: 338), but, before that, rightly stresses a first-order question: 
whether, indeed, ``funded'' (by which he means privately managed, pre-funded) pensions really resolve the (unspecified) ``demographic, economic and financial (sic!) pressures'' supposedly arising under ``PAYGO'' (by which he means publicly managed, post-funded) (\emph{ibid.}: 339). 
There are at least three TINAs that \citeauthor{Cerami2009a}'s account of the first-order social choice falls for:

\begin{enumerate}
	\item \citeauthor{Cerami2009a} (and \citealt{Bastian1998}, too) seemingly accept ``funded'' vs. ``PAYGO'' as a demographically meaningful alternatives, even though they are clearly not. 
	Funded, or more precisely, privately managed, pre-funded pensions may change the \emph{incidence} of aging, by burdening current workers, but they not by virtue of being privately managed alter the economy-wide or even individual, intertemporal Haig-Simons identity (for a detailed model see \citealt{Borsch-Supan2003}: 170). 
	Privately managed, pre-funded, just as publicly managed pre- or post-funded regimes may, or may not save enough for the future, and may, or may not invest such surplus production wisely to make up for the dissaving in offspring, and growing life expectancy. 
	``Pre-funded'' conveys a false sense of austere security that social scientists should not buy into.
	
	\item Likewise, \citeauthor{Cerami2009a} cites, without reproach, the arguments for ``pre-funded'' schemes, including supposed higher returns of private investments and breaking of the ``vicious cycle'' of PAYGO. 
	Both, again, assume alternatives where there truly are none. 
	
	Privately managed funds may, or may not generate higher returns than equivalent \glspl{SWF}. 
	Any supposed superior performance of one or the other management requires a specific economic theory to explain it, and empirical evidence to support it, non of which are mentioned here\footnote{
		There may be good, theory-driven, empirically supported reasons for favoring private or public management of pre-funded schemes, but that is part of the messy detail that neither \citeauthor{Cerami2009a} nor I can, or need to discuss here.}.
	To just mention such supposed superiority of privately-managed funds without explaining why that would be so, is to buy into the unquestioned promises of neoliberal ideology.
	
	There is also no such thing as a ``vicious cycle'' of PAYGO, where, in \citeauthor{Cerami2009a}'s reading of its opponents, ``current workers are forced to pay for current pensioners'' (339). 
	True enough, a move from nominally post-funded to nominally pre-funded regimes alters the nominal incidence of demographic change, but that is simply a zero-sum redistribution of the hardship that some group, at some point, has to endure. 
	There is nothing particularly self-reinforcing about (nominally post-funded) PAYGO, that would make it ``vicious''. 
	
	A similar notion comes from \cite{Bastian1998}, who, with alarm, reports the rising share of public pension outlays in \gls{CEEC} GDP (\citeyear{Bastian1998}: 69), and cites PAYGO as the ``main reason'' of the fiscal malaise (\emph{ibid.}: 71). 
	That is of course, rather tautologically, true: in a PAYGO system, all other things equal, the public purse absorbs demographic change. 
	However, he seems to be unaware that the percentage of pensions, or, equivalently, consumption of elderly people, is entirely \emph{invariant} to the pension design. 
	If PAYGO is transformed into a ``funded'' scheme, the same number of old people will, all other things equal, still consume the same percentage of economic output. 
	
	\item Conversely, \citeauthor{Cerami2009a} also glosses over the arguments presented against pre-funded reform. 
	
	He reports caution about the supposed demographic cure-all of pre-funded pensions (339), but again, fails to explain why --- as is in fact correct --- the public or private management, or even nominal pre- or post-funding do not alter the Haig-Simons dissaving at all.
	
	\citeauthor{Cerami2009a} also cites risks associated with unstable markets (339) as evidence for the prosecution of pre-funded schemes, but apparently relying more on leftist gut-feeling than critical reasoning, fails to explain what the theory and evidence of theses risks would be. 
	To be sure, pension schemes should probably spread and thereby minimize risk, but as these messy details go, they are no matter to be mentioned in passing. \citeauthor{Cerami2009a} here assumes a possibly non-existing alternative by suggesting that \emph{only} privately managed, pre-funded pension schemes are risky.
	Clearly, publicly-managed, even post-funded pensions also include --- albeit probably smaller\footnote{
		\ldots or not, according to \citealt{Borsch-Supan2003}: 178.} 
	--- risks: 
	an \gls{SWF} can make poor investments, and even a social-security PAYGO system by only betting on one class of investment (labor productivity) in one market (the domestic economy) takes on risks\footnote{
		For example, \citeauthor{Cerami2009a} presents  2009 \gls{OECD} reports of 20\% losses in private pension funds as evidence against private management. 
		That need not be so: 
		\begin{inparaenum} 
			\item The losses may so far be evident only in balance sheets, and need not ever --- but well may --- materialize in diminished cash flows. 
			The depreciated assets may bounce back. 
			Well-managed funds will insulate individual pensioners from such short- and medium-term fluctuations. 
			\item These same losses, might, absent a private pension scheme, have materialized elsewhere in the economy. 
			For example, workers might have invested the windfall in other risky assets. 
			Bubbles and crashes are inter temporal bumps in the Haig-Simons identity, and they will always hurt the economy --- as \citeauthor{Cerami2009a} (340) concedes ---, no matter their nominal incidence.\end{inparaenum}}. 
	Pensions --- as all savings --- always entail some risk: 
	whoever manages this pre- or even nominally post-funded surplus production has to decide where it will generate the highest return at an essentially uncertain future point in time. 
	As \citeauthor{Barr2005a}, again lucidly, point out: 
	``PAYG[O] and funding are both financial mechanisms for organizing claims on that [future] output [\ldots] [and] fare similarly in the face of output shocks.'' (\citeyear{Barr2005a}: 156). 
	
	But \citeauthor{Cerami2009a} here also ignores an alternative, that in fact may exist: 
	well- (or better-)regulated markets that efficiently spread, and thereby minimize risks. 
	Especially to a critical social scientists there must be a very good reason to assume that financial markets are \emph{always} unable to spread risk, or, absent such compelling (economic, first-order) reason, the social scientist must suggest (sociological, second-order) reasons why the institutions of financial capitalism underperform in this specific way. 
	By not even alerting us to this issue, but by dogmatically assuming financial market failure, \citeauthor{Cerami2009a} --- surely unintentionally --- feeds a particularly backhanded TINA of truly neoliberal hegemony: 
	that financial markets --- be they good or bad --- cannot be altered, or, that their failure or success is inevitable, and needs no social scientific explanation\footnote{
		For a fully-fledged Gramscian account of european integration, see \cite{Bieler2002,Bieler2003,Bieler2005}.}
	%make sure to mention risk in the above.
	
	Lastly, as an inverse argument to the ``vicious cycle''-critique of PAYGO, \citeauthor{Cerami2009a} cites the ``double payment'' of current workers as they are transitioned to a pre-funded regime\footnote{
		As a pre-funded, privately managed component replaces, or is added to a nominally post-funded, public managed pension, current workers have to pay twice: 
		once into a private account for their old age, and once into a public account fur currently old, PAYGO recipients.}. 
	As the ``vicious cycle'', the ``double payment'' argument is mis-construed: 
	there is \emph{nothing} particularly bad, or ``double'' about pre-funded regimes, just as there is nothing ``vicious'' about post-funded regimes. 
	The difference between the two is a trivial, zero-sum redistribution of the incidence of demographic change. 
	Under PAYGO, depending on parametric configurations, the young and/or the old pay for the dissaved future workforce. 
	When pre-funded regimes are added to the mix, but all other things remain equal, only the young pay for demographic change. 
	The sum paid, in any event, does not change. 
	By framing pension reform as a struggle between the currently young and the currently old, \citeauthor{Cerami2009a} falls for an old ruse: 
	\emph{divide et impera}, divide and conquer. 
	If we obsess about the mystically ``vicious'' or ``double'', but truly trivial incidence of pre-funded vs. nominally post-funded pensions, we loose sight of the bigger redistributive choices of aging societies: 
	whether to burden the rich, or the poor, to burden current, or future generations.
\end{enumerate}

It may, again, seem nitpicky, to relentlessly criticize authors as critical as \cite{Cerami2009a} himself, who, after all, only reports arguments frequently presented. 
Still, I (nit-)pick on \citeauthor{Cerami2009a}, precisely because he is so far left of neoliberalism, but, I would maintain, not persuasively so. 
He rightfully alerts us to the ideological import in pension reform debates (\emph{ibid.}: 340), but penetrates not nearly deep enough into the economic abstractions of pension-design to fully expose the hegemony he correctly suspects. 
Whatever the merits of his second-order \emph{explanans} of a new politics of pension reform, he has the first-order \emph{explanandum} wrong: 
he assumes economic alternatives where there are none (pre-funded vs. post-funded), and neglects other, more relevant choices (financial market reform). 

TINA plagues not only the right, but, more deviously, the left, too. 
When critical social scientists, as \cite{Cerami2009a}, present only the dogmatic, but shorthanded arguments against neoliberalism (``private pensions are too risky''), their important dissent remains superficial, and will easily brushed aside by more assiduous, if duplicitous, neoliberals. 

TINA operates not by straightforwardly denying leftists ``possible, better worlds''. 
Rather, TINA is neoliberal and conservative in a roundabout way: 
it obfuscates the abstractions through which any such hypothetical, preferable worlds may be glimpsed. 
And so, even critics as \citeauthor{Cerami2009a}, inadvertently serve TINA, when they neglect the deep economic abstractions, maybe confusing their neoclassical language and hard choices for the neoliberal doctrine that has so successfully appropriated them. 
In pension design, as elsewhere in public policy, a Haig-Simons understanding of the economy helps us to sift through the epiphenomenal debates (``funded'' vs. PAYGO), to relegate the complex details (financial markets) to appropriate theory and data, and advance to those choices that our scarce, constrained and material world leaves us to take: 
how much we should save for future generations and in which form, and who of us, rich or poor, should contribute how much. 
\emph{These} are the real first-order alternatives that a second-order theory as \citeauthor{Cerami2009a}'s must take as a starting point. 
All policy and all pension designs, underneath the complex detail and nominal casuistry, make these choices. 
My hunch is: 
many  current designs and their marginal reforms greatly burden future generations, and present non-rich people, a peculiar choice, that may not even enjoy popular support. 


\citeauthor{Schui2009} is right to point out that minimizing the problems of old age insurance to demographic change is latently affirmative: 
it distracts us from the possibility to redistribute the fruits of growth and the pains of decline (\citeyear{Schui2009}: 147)\footnote{
	\citeauthor{Schui2009}, the orthodox leftist, is of course otherwise hardly moved by the strictures of Haig-Simons. 
	Ever the die-hard Keynesian --- or Marxist? ---, he always and everywhere assumes a crisis of underconsumption, or, equivalently, excessive overall profits, and seems not to allow even the possibility of material constraints at some exogenously given, only slowly expanding steady state.}.

The enormity of these alternatives can hardly be overstated, especially for \gls{CEEC}, where hard-working people often spent old age in abject poverty. 
As even the aging economies of Europe \emph{are still growing} over the medium- and long-term on a per-capita basis, we should be able to build a pension regime that, somehow\footnote{
	Progressive taxation comes to mind.} \footnote{
	A pension scheme including, or supplemented by progressive taxation, does not preclude actuarial components in a pension formula. 
	As \cite{Barr2005a} helpfully reminds us, working longer may not be such an undue thing to ask, if people live longer, too. 
	If people some people like to retire earlier, and others are willing to work longer, we might want to punish and reward them accordingly, while still asking the rich to chip in more for any actuarial increment earned. 
	Alternatively, and probably more transparently, the redistributive component can also be organized exclusively through the tax code, with pensions remaining cleanly actuarial.\\
	Evidently from the summary of the mixed economy in table \ref{tab:ends_mixed_economy} (p. \pageref{tab:ends_mixed_economy}), if, regrettably, not from all real existing self-ascribed social market economies, asymmetrically known \hyperref{sec:risk}[risk] should be covered by compulsory or state insurance. 
	This also includes disabilities that may occur more often in old age, or in some occupations. 
	Presently, some pension regimes --- clumsily --- include some form of old-age disability insurance, and debates on extending the age of retirement inevitably bring up the plight of the old construction worker. 
	In all of this, I assume that there is extensive, compulsory or state-covered disability insurance. 
	When I argue for actuarial pensions, and/or later retirement, I assume that whoever cannot, because of occupational or other disability, work into her older age, should receive benefits out of the disability insurance until she reaches the statutory retirement age. 
	If, as seems likely, disability clusters in risky or hard jobs --- such as teaching or construction --- it may even make sense to charge a premium for insuring these jobs, so as to raise the costs of such hazardous labor, and to make it safer or rarer.}
collects this still increasing economic output, saves enough for our children and disburses enough to our seniors. 

If we do not, such criminal neglect of the mixed economy certainly deserves a fair trial, and begs a social scientific explanation.

%barr, for pensions is the benchmark \cite{Barr2005a}

\subsubsection{Pangloss} \label{sec:Pangloss}

\begin{quote}
	\emph{``It is demonstrable'' said he, ``that things cannot be otherwise than they are; for as all things have been created for some end, they must necessarily be created for the best end. 
	Observe, for instance, the nose is formed for spectacles; therefore we wear spectacles. 
	The legs are visibly designed for stockings; accordingly we wear stockings. 
	Stones were made to be hewn and to construct castles; therefore my lord has a magnificent castle; for the greatest baron in the province ought to be the best lodged. 
	Swine were intended to be eaten; therefore we eat pork all year round. 
	And they who assert that everything is right, do not express themselves correctly; they should say that everything is best.''}\\*
	--- Fictional Dr. Pangloss in Voltaire's novella \emph{Candide} (\citeyear{Voltaire1759}: loc. 125).
\end{quote}

\paragraph{Second Best.} Today, maybe one of the clearest Panglossian pronouncements comes under the imposing heading of a \emph{Theory of Second Best}, originally formulated by \cite{Lancaster1956}. 
As so many great economic ideas, this one has strayed far from its original form, and interbred with ideology to father many illegitimate --- and sometimes  deformed --- offspring. 

In its initial, rigorous formulation, the Theory of Second Best showed formally that if --- as seems likely --- some conditions for the pareto optimality of markets cannot be meet, it might enhance efficiency to allow additional, possibly offsetting deviations from perfect competition elsewhere in the economy. 
Rather than to fight all market failures everywhere in isolation (``piecemeal welfare economics'', \emph{ibid.}: 11), \citeauthor{Lancaster1956} suggested that from a general equilibrium view, there may be less demanding \emph{necessary} conditions that could pareto improve the economy, in addition to the often implausible, \emph{sufficient} conditions of perfect competition (\emph{ibid.}: 17). 
 \cite{TheEconomist2007} explains it beautifully thus: 
 if your optimal cookie recipe requires chocolate chips and coconut flakes, but you cannot find the chocolate chips, your (second) best bet may be to bake gingersnap, rather than chocolate chip cookies without chocolate chips. 
 This is the kind of devilish complexity that I allow myself to ignore here, but that policy makers have to consider: 
 if, for example, research and development are so prohibitively expensive that we absolutely cannot profitably have more than one manufacturer of wide-body aircraft, our (second) best policy may indeed be to stray further from neoclassical doctrine, and to keep subsidizing \emph{The Boeing Company} and \emph{Airbus SES}, so that we can have at least have ourselves a decent, somewhat competitive, duopoly. 
 There is nothing Panglossian about such hard choices because, it is, in fact, \emph{materially} impossible to develop dozens of competing wide-body designs. 
 Because for the social sciences --- as for moral philosophy --- \emph{ought implies can}, the second-best of the subsidized Boeing/Airbus duopoly also needs no social scientific explanation. 
 This \emph{really} is collateral damage to a worthy cause.

After \citeyear{Lancaster1956}, however, the Theory of Second Best as slowly morphed into a general skepticism of state intervention, it is now name-dropping ``proof'' of its \emph{ipso-facto} futility and serves as welcome absolution for the demise of the mixed economy. 
\citeauthor{Wolf1987}(\citeyear{Wolf1987,} \citeyear{Wolf1979}), for example, argued that because governments fail, just as markets do, the second-best response to such market failures may be to just let them be. 
You can take this argument to merely logical extremes, and throw out government and democracy altogether. 
\cite{Leeson2009}, for instance, wonders whether in some (developing economies) settings, anarchy may not be preferably to predatory statehood, whether, in other words, no state would not be second-better than an inevitably failing government. 
\cite{Caplan2007}, in an otherwise thoughtful book, seems to suggest that because voters are so invariably rationally irrational, markets may be second-better than democracy altogether.

This has very little to do with the rigorous argument presented by \citeauthor{Lancaster1956}: 
he did, at least in \citeyear{Lancaster1956}, never consider a constrained government, let alone an incapacitated democracy to be grounds for ``second-besting'', but, instead even seemed to hope for a government and people powerful and wise enough to heed his call. 

This is Pangloss at his finest: 
if you assume, as modern-day second-besters do, that the very \emph{means} to deliberately get to a better world --- government and democracy --- are inevitably flawed, you can show, with almost hermetic logic, that whatever world we find ourselves in, must be the best of all \emph{possible} worlds. 

In that word --- possible --- lies the catch. 
Modern-day second-besters assume that, akin to markets and evolution, government and democracy are \emph{aimless} processes that merely aggregate pre-social, more- or less rational self-interest. 
If government and democracy are aimless, it follows --- as it does, in fact, follow for markets and evolution --- that any positive results of government and democracy are beyond reproach, and beyond improvement. 
Government failures, as monopoly pricing or an appendicitis, are just \emph{facts}. 

Enlightenment, the mother of modern science, did not consider democracy, a merely \emph{positive} affair, but a normative prescription. 
\cite{Kant1785} asked us not to ``follow your incentives'', but to ``act only on that maxim through which you can at the same time will that it should become a universal law'' (\emph{ibid.}: Chapter 11). 
The US Constitutional Convention in 1787, did not just decide to try out this new \gls{FPTP} way to aggregate preferences, but ``We the People'' were to do so ``in Order to form a more perfect Union''.
In a free society, social scientists need not believe in these, or any other ideas, but if they reject them all, they deserve not the air of scientific sophistication in which they cloak their utter cynicism. 
As mere accountants of the status quo, their work is anathema to Enlightenment, and they ought to be disowned of the emancipatory heritage that the social sciences were endowed with.

But ignoring, for now, the enlightened humanism that comes part and parcel with the social sciences, the logic of modern-day second-besters is also simply fallacious. 
Even if government and democracy turn out to inevitably disappoint, such flaws are \emph{not}, to the social scientist, quasi-material constraints. 
Because government and democracy \emph{are} the subject matter for the social scientist, she must not presume, but has to \emph{explain} them. 
If we do, as the second-besters, exclude from the first-order alternatives to be explained by second-order theory, those alternatives that the political process \emph{may} corrupt, we have thereby conflated first and second-order theory. 
Whatever second-order theory might have to tell us about a corrupted political process, we would never learn, if we did not test for the absence of first-best solutions. 
This Frankenstein variant of the Theory of Second Best confuses, as \cite{Brubaker-2002-aa} succinctly criticized the literature about ethnic conflict, the ``empirical data'' with our ``analytical toolkit'': 
government failure is ``what we want to explain, not what we want to explain things \emph{with}'' (\emph{ibid.}: 165, emphasis in original).
%add caveat to this, excusing the guy on development second best that indeed, government failure may be endemic under some constellations.

Panglossian bastards of the Theory of Second Best also roam the literature on European and \gls{CEEC} welfare states. 
I will here only cite one model student of Pangloss', and eminent social scientist, \citeauthor{Moravcsik-2002-aa}, who refutes a supposed neoliberal bias of european integration thus:
\begin{quote}
	\emph{``No responsible analyst believes that current individual social welfare entitlements can be maintained in the face of these [postindustrial, demographic, etc.] shifts. 
	In this context, the neo-liberal bias of the \gls{EU}, if it exists, is justified by the social welfarist bias of current national policies [\ldots].''}\\
	--- \citeauthor{Moravcsik-2002-aa} (\citeyear{Moravcsik-2002-aa}: 618)
\end{quote} %add here.
In other words, even if European integration constrained national, democratically governed mixed economies, that would be for the better because these welfare states are too spendthrift to begin with. 
Also, according to \citeauthor{Moravcsik-2002-aa}, to suggest that mixed economies might --- using efficient fiscal, regulatory and monetary tools --- alter market allocations any which way they want, is \emph{irresponsible}. 
Pangloss would probably applaud how elegantly \citeauthor{Moravcsik-2002-aa} defines away all redistributive considerations, and, for good measure, adds an ad hominem. 

\paragraph{En- or Retrenchment.} Todays Pangloss has grown more sophisticated since the times of Enlightened Absolutism, when he was easy game for satirist Voltaire. 
As any influential teacher, he has learned to lead on his students by making them ask the questions that suit his lesson plan. 
The lesson relevant here is that on European and \gls{CEEC} welfare states. 
It is remarkable just what kind of feeble questions our ever-affirmative teacher Pangloss has gotten us to ask.

\cite{Beckfield2006}, for instance, contents himself to ask how much of \gls{EU}-15 income inequality can be explained by regional political integration as opposed to (economic) globalization, and, alarmed, finds that nearly half of it can be explained thus. 
He lists the ways in which  regional political integration constrains the welfare state: 
through \begin{inparaenum} 
	\item policy feedback such as austerity-enforcing nominal convergence criteria, 
	\item diffused classical-liberal policy scripts
	\item possible blame avoidance and
	\item by tying \gls{MS} economic fortunes to one another.
\end{inparaenum} This is rigorous empirical work, albeit, lacking a natural experiment, and plagued by questionable external validity and --- one fears --- intricate multicollinearity, it will always remain vulnerable to methodological critique. 
More important, though, is what \citeauthor{Beckfield2006}, in his quest to tell apart the inequality of globalization and political integration, \emph{does not ask}: 
how much of the rising income equality could an intact, european mixed economy have enforced, and why did it not do so? 
The bigger question, I would maintain, is what \emph{kind} of political integration could have curbed inequality. 
\citeauthor{Beckfield2006}, again, surely is one of the authors rightfully critical of globalization and the current mode of \gls{EU} integration. 
Still, Pangloss, one imagines, sympathetically smiles at this diligent student, hardly moved in his affirmation of the status quo by such timid and naive questions. 
Pangloss can rest assured, as long as \citeauthor{Beckfield2006} and others busy themselves with the nitty-gritty of welfare state retrenchment, nicely playing globalization off against regional integration, which are really two sides of the same coin. 
Ever affirmative, if pressed for an explanation, Pangloss can still wring his hands at the inequality, shrug his shoulders and point to the gains from trade. 
He has already won this game, as \citeauthor{Galbraith2002a} wryly observes: 
``So what are the facts [of inequality]? 
Has globalization hurt or helped? 
Oddly, researchers do not know; mostly they do not ask.'' (\citeyear{Galbraith2002a}: 11, also \citealt{Crouch2004}: 158).

If sceptics as \citeauthor{Beckfield2006} are the slightly recalcitrant, but still manageable students in Pangloss' classroom, the naysayers of retrenchment as \cite{Swank-2005-aa} are his overzealous disciples. 
\citeauthor{Swank-2005-aa} argues that globalization did not force welfare states to retrench, because, evidently, income replacement rates in unemployment, health and pension insurance remain high. 
Pangloss would surely applaud in delight: 
``excellent work, all is well!''. 
What did \citeauthor{Swank-2005-aa} \emph{not} ask, that would so endear himself to Pangloss? 
Plenty. 
\begin{enumerate}
	\item Obviously, and at minimum, \citeauthor{Swank-2005-aa} should look at public debt and other \hyperref[sec:smoke_n_mirrors]{smokes and mirrors} of the mixed economy to make sure that these sustained income replacement rates are not built on a base of sand, long washed out by tax competition (p. \pageref{sec:smoke_n_mirrors}). 
	He is certainly right that there will be substantial political pressure to maintain welfare states, but their victories may be pyrrhic if bought at the price of sovereign default, or heightened \glspl{DWL}.
	\item More fundamentally, the causal rejection (!) of welfare entrenchment through globalization as put forward by the optimists needs careful positivist research design. 
	 This would require a hypothetical, namely a sufficiently large, prosperous and closed economy. 
	 This does not exist in the OECD world, and may not exist at all in reality, as there is a well acknowledged trade-off between liberalization and prosperity. 
	 This real world constraint notwithstanding, empirical arguments have to take this methodological problem into account to answer the question whether welfare states can \emph{sustainably} continue to operate under international economic liberalization.
	\item Thirdly, lastly, and most fundamentally, the optimists seem to be not so much optimistic about what a welfare state can do, as they seem to be minimal about what it should do. 
	In part, this neoliberal bias follows from an epistemological concentration on the status quo. 
	The research question that \citeauthor{Swank-2005-aa}, for example answers, is not whether globalization challenges the welfare state. 
	Instead, he argues that it may not affect the status quo of \emph{income replacement}. 
	This is an unexplicated, negative and minimal definition of the welfare. 
	It assumes that the legislator’s social agenda is limited to income replacement, without regard to broader redistributive issues, particularly the question of who should bear the costs of income replacement. 
	More generally put, this amounts to what the public management literature calls an output perspective on how much money is spend on welfare, in this case.
\end{enumerate}
 
%check this again! Not sure about Leibfried!
%!marginpar!
\marginpar{The remainder of this section is still quite confused. 
It needs a better structure and I need to double-check the references. 
Read with care.}

But Pangloss is pleased with other students, too, including such eminent voices as \cite{Leibfried-2005-aa} who seems to consider anti-discrimination policy as an instance of social policy proper. 
Pangloss might silently rejoicing at the neoliberal bias he has so successfully instilled in his student. 
Anti-discrimination --- however extensive --- already by name, is negatively defined, and diminishes social policy to be concerned with the abolition of negative constraints, of things we cannot do, of market distortions and at the same time clouds a positive definition of social policy as furthering our freedom \emph{to} do something, not freedom \emph{from} something.
 
Welfare regimes are, as \cite{Esping-Andersen-1990-aa} has said, systems of stratification in themselves, and they have been, from the very beginning socialist demands in the 19th century more than epiphenomenal income replacement. 
They are tools to socialize the costs of painful, but welfare-enhancing \citeauthor{SchumpeterSwedberg-1942-aa}ian economic transformations as well as individual hardship and serve as engines to redistribute this very welfare as the legislator wishes. 
The indeed quintessential and highly political question for globalization and welfare entrenchment is then, whether the state can \emph{still} (or ever could) socialize and redistribute as it wants, with no limitations resulting from the behavior of other states. 
Welfare state sustainability then has to be pitted not against past or current performance, but against a hypothetically desired welfare regime under global trade with global redistribution, both within and between countries. 
In game theoretic terms, to gauge how badly welfare-depressing defection is, you first have to calculate the payoff for mutual cooperation.
In the global context, this certainly requires quite a bit of political imagination, something that political scientists, it appears, like to shy away from. 
Even leaving normative considerations aside, in this case, academic rigor alone requires such exercise.

If welfare states can be well or poorly designed mixed economies, achieving different outcomes, we should also judge its prospects by comparing actual or evolving regimes to such hypothetical, but possible and desirable configurations. 
That is a very different question than testing whether welfare states en- or retrench, let alone its spurious correlates of income replacement \citep{Swank-2005-aa} or even spending (\citealt{Kleinman2002}: 24), and a question that would deeply unsettle Pangloss. 
As \citeauthor{Offe2003} reminds us:
	\begin{quote}
		``The mode in which welfare state institutions change can be explicit reform and retrenchment. 
		But it can also be inconspicuous and gradual decay. 
		For instance, people may defect from public health and pension systems, trade unions see themselves forced into single-employer concessions bargaining, workers resort to unprotected forms of pseudo self-employment in order to avoid social security dues, if not to illegal (“black”) forms of employment.'' (\citeyear{Offe2003}: 364)
	\end{quote}
\citeauthor{Genschel2005}, too, reminds us of what Pangloss would rather have us forget: 
``The effect [of globalization] is not so much to force change upon the tax [and thereby, welfare] state as to reduce its freedom to change." (\citeyear{Genschel2005}: 53). 
This is  why political scientists such as \cite{Pierson2002,Pierson1996} that assume institutional constraints (e.g. veto points, \citealt{Tsebelis-2002-aa}) to only work to \emph{prevent} retrenchment, are wrong: 
the very same constraints that may prevent or delay nominal cuts will also make it harder for welfare states to \emph{adapt} to, rather than just to recede in the face of changing economic and demographic circumstance. 
Nondecision does \emph{not} ``generally favor the welfare state'' (\citealt{Pierson1996}: 174). While the welfare state might nominally have retrenched only ``cautiously'' (\emph{ibid.}: 174), the ground underneath it has shifted, leading to a much graver de-facto change of positions: 
it can no longer expand or react, relies on unsustainable deficits or real dissavings and must make do with growing inequality, and sometimes, structural unemployment, non of which \citeauthor{Pierson1996} even mentions.

If we do not inquire about that freedom to change, we will --- as \citeauthor{Kleinman2002} (\citeyear{Kleinman2002}: 99) --- eat up the Commission's diagnosis of the european economic malaise: 
``the main explanation for the poor unemployment performance of the Community over the past two decades is to be found in the constraints that unresolved distributional conflicts and insufficient structural adjustment placed on macroeconomic policies.''

%In the lesson on welfare state en- or retrenchment we should demand an answer for this question: not only, whether income replacement, or even spending, has declined or stayed stable, but whether mixed economies can still today, as they ideally should, choose arbitrary and efficient mixes of state and market.

To shake off Pangloss, and see clearly the demise of the welfare state, we must ask three new questions, that so far, much of the retrenchment literature has shirked:
\begin{enumerate}
	\item What, \emph{given} a hypothetical, intact mixed economy, would welfare states be capable of, if their democratic sovereigns wanted it? 
	This is a question that, absent natural experiments on the matter, cannot be subjected to straightforward positive test, because precisely such a hypothetical, intact mixed economy does not exist. 
	Still, we must compare actual to hypothetical regimes, to find out just how constrained actual welfare states may, or may not be, by whichever second-order process we subsequently propose.
	
	\item What is the highest possible tradeoff between equity and efficiency that an intact mixed economy can offer their democratic sovereigns? 
	As I have argued in the above, better-designed mixed economies face less harsh --- or even no --- tradeoffs between equity and efficiency than worse-designed mixed economies. 
	The price for an additional increment of equity in efficiency (or vice versa) not only varies at the margin\footnote{
		It seems reasonable to assume that, akin to a production function with diminishing returns to scale, the relationship will be convex-curvilinear. 
		At very low levels of equity, initial improvements in equity will be quite cheap in efficiency. 
		At very high levels of equity, further increments in equity might be quite expensive in efficiency. 
		At full equity, of course, the free price system ceases to operate.}, 
	but it also varies depending on the set-up of the mixed economy. 
	For example, a highly progressive tax on consumption may allow same or greater equity at a much lower price than a compressed tax on labor income\footnote{
		\citeauthor{Offe2003} reminds us that there is no ``hyper-rational'' answer of a \emph{best} balance between efficiency and equity (\citeyear{Offe2003}: 445): 
		that is an essentially political question, and should be decided by democratic sovereigns. 
		There are, however, objectively \emph{better} or {worse} institutional designs under which these tradeoffs are made. 
		For example, experts cannot know an optimal progressivity in a tax code, but they may well show that whichever progressivity the democratic sovereign desires will cost less in efficiency under a consumption than an income tax (e.g. \citealt{McCaffery2005}, \citealt{Frank2005})}.
	More broadly, \citeauthor{Ganßmann2010} has proclaimed the Scandinavian welfare state as the ``winner'' to achieve higher levels of \emph{both} equity and efficiency (\citeyear{Ganßmann2010}: 343). 
	
	Amongst these higher tradeoffs between equity and efficiency, there may, additionally, be local optima of equity-efficiency mixes, complemented by quite distinct institutional constraints and --- somewhat related --- path dependencies, ranging as wide as educational systems and industrial relations. 
	I have here mostly ignored these \emph{Varieties of Capitalism} \citep{HallSoskice-2001-aa}\footnote{
		At first sight, \glspl{CME} might be considered to be in violation of ordoliberal economic policy, and might therefore be considered incompatible with the largely neoclassical model of the mixed economy I have sketched here. 
		This need not be so. 
		
		Straightforwardly, as \citeauthor{HallSoskice-2001-aa} suggest, \glspl{CME} might just have a competitive advantage in the production of incremental improvement, and what appears as deviations from atomistic competition is de-facto firm organization at a higher, economy-wide level --- much as the notion of the ``Deutschland AG'' (Germany Inc.) suggests. 
		Akin to the framework suggested by \cite{Hart1990}, \glspl{CME} may simply be an economies way, to minimize the transaction costs involved in the production of their specialties by ``insourcing'' important counter parties by institutional design, if not formal merger. 
		
		Alternatively, I suspect, many of the institutional peculiarities of \glspl{CME} might be explained --- as I have done here for the welfare state --- as remedies to particular market failures. 
		Industry-level wage bargaining, for example, might effectively balance the playing field between monopsony employers and labor cartels (unions). 
		Employment protection legislation, conversely, might work to smooth the business cycle (albeit imperfectly) or might insure workers against the risk of specialized education, they might otherwise be to risk-adverse to take (e.g. \citealt{Offe2003}: 444).}. 
	They, too, are important detail. 
	They strongly suggest that there may not be \emph{one} universal mixed economy design, but that quite different designs might coexist and specialize. 
	Still, also within \emph{each} of these varieties, there are again different tradeoffs between equity and efficiency. 
	Interestingly, the institutions that \citeauthor{HallSoskice-2001-aa} identified as markers of \glspl{CME} and \glspl{LME} do \emph{not} mention variants of tax, social insurance or any of the other key welfare state institutions. 
	While \glspl{CME} may correlate with, and are often conflated with Bismarckian welfare states, there is nothing about \glspl{CME} that would make them necessarily rely on, for example, labor income-backed social insurance. 
	Of course, the variant of capitalism will be reflected in the nitty-gritty of welfare state institutions: 
	for example, the statuses originally maintained by Bismarck are, arguably, related to the categorical groups that \gls{CME} educational systems create, or \gls{CME} industrial relations are organized around. 
	This institutional spill-over notwithstanding, I would hypothesize, that \glspl{CME} and \glspl{LME} might be able to achieve equally high tradeoffs between equity and efficiency, even if the institutional implementation may vary: 
	for example, \glspl{CME} might continue to sport extensive job protection, while \glspl{LME} will allow quick ``hire and fire'', potentially complemented by generous unemployment benefits (as in ``flexicurity''). Allocative results, either way, may be very similar, which is my point here.
	
	Even carefully crafted positive research into changing welfare states currently looks, at best, at cross-sectional or longitudinal variation in social transfers as a percentage of output (e.g. \citealt{Ravenhill2005}: 
249). This is much better scholarship than the naysayers who like to look only at absolute spending or income replacement, but still, it does not tell us how good a tradeoff we are getting.
	
	To gauge the \emph{level} of tradeoff between equity and efficiency available to democratic sovereigns, the retrenchment debate has to look at allocative \emph{results}, not at transfer flows, that is, at inequality (e.g. Gini-coefficients) and growth (preferably measures more comprehensive than \gls{GDP}).
	
	Out of logical necessity, if nothing else, welfare state retrenchment, inequality and growth are \emph{one question}. 
	The compartmentalization of these into different academic areas allows not, as one would hope, greater theoretical clarity but instead confuses and waters-down concepts. 
	If ``welfare'' is to mean anything, surely, it must be the ability of states to alter allocative \emph{results}, and to do so at a minimum, or democratically acceptable price in efficiency.
	
	Consider the alternative research designs, that currently predominate. 
	If income replacement stays the same \citep{Swank-2005-aa}, but, realistically, incomes become more unequal and states more indebted, is that evidence of a non-retrenched welfare state? 
	If transfer volumes rise absolutely, or stay constant relative to output \citep{Ravenhill2005}, but, realistically, ever more people rely on ever smaller transfers, all paid for the labor-incomes of an already squeezed middle class, is that evidence of a non-retrenched welfare state? 
	Surely, just external validity requires more extensive operationalizations. 
	The best, theory-driven operationalization of a non-retrenched, welfare state is the intact mixed economy.

	\item How well does the welfare state work as an entire system of production and distribution, that is, as a mixed economy? 
	This is a very different question from those based on a traditional, more limited definition of the welfare. 
	
	\citeauthor{Offe2003}, for instance, takes pains to remind readers that welfare states have nothing to do ``with equality of outcomes', neither normatively nor positively'', that ``the guiding principle of principle (...) is the security and protection of workers, not equality'' (\citeyear{Offe2003}: 450). 
	This is a historically accurate definition, but it is no longer an externally valid conceptualization of welfare states, if the term is to be more than an empty hull devoid of positive reality and economic possibility. 
	``Welfare states as worker protection'' is not \gls{MECE} any more. 
	By this definition, an economy, or rather, sectors thereof would be classified as a ``welfare state'', in which poorly qualified workers are nominally protected, but either structurally unemployed because their gross wages are higher than their productivity, or live in working poverty, ever unable to participate in the riches of the wider economy
	To the people working in cleaning or security in Germany today, such a definition would not have a lot of face validity. 
	The labor market for poorly qualified workers in Germany is then, at the same time, evidence of a welfare state and evidence of a non-welfare state. 
	Conversely, by the traditional definition, an economy, or rather a sector thereof with no nominal protection, but generally high compensation and little economic hardship, would be classified as a non-welfare state. 
	For example, freelancers in management consulting, earning handsome but unsteady labor (!) incomes, but equipped with enough assets to weather rainy days, surely do not enjoy welfare state protection. 
	Still, at face validity, they also do not exactly suffer from Manchester style laissez-faire. 
	Under the traditional definition, they reality is neither welfare-state, nor non-welfare state.
	
	\citeauthor{Offe2003} might, asked about the plight of German cleaners, point to a leak in the ``Keynesian'' roof of full employment, protecting the lower storeys of welfare states. 
	Today, full employment is only a necessary, but not a sufficient condition for an intact roof. 
	In fact, full employment might always have been merely necessary, and we were just lucky that in the past, all other necessary conditions were mostly met. 
	To stay in \citeauthor{Offe2003}'s elegant metaphor, the welfare state house is facing much harsher weather. 
	For example, severe crosswinds of rising income inequality (e.g. winner-take-all, efficiency wages) diverging factor returns (e.g. Stolper-Samuelson trade), threaten to further drive apart the different economic strata making up the house, threatening to tilt the building. 
	In addition, international tax competition, but also home-made dysfunctions are eating away at some of the load-bearing walls, putting enormous stress on the few remaining walls and the (already struggling) people making it up. 
	If all we care about in this house is whether the roof is still tight against cyclical unemployment, the structure will not stand much longer. 
	If the house of the welfare state is to survive the throes of economic transformation, it needs strong cross-beams, to re-balance the load of its stories. 
	These cross-beams are progressive redistribution, and we measure their solidity by looking at overall inequality. 
	Today, if not always, the ability of a mixed economy to efficiently curb runaway inequality is the \emph{sine qua non} of welfare states, too. 
	
	This is not new to \cite{Offe2003}, who also includes ``monetary, fiscal, trade and economic policies'' in the roof (\citeyear{Offe2003}: 543). 
	However, he seems to neglect that consequently, the different stories of welfare protection \emph{cannot} be organized (financed) irrespective of overall inequality: 
	if, for example, the fiscal shingles are to remain intact, the protection schemes must charge those most who can best afford it and in a way that will least affect them\footnote{
		This is also why, at least for welfare state researchers, or those concerned about social integration should not look only at relative, let alone \emph{absolute} poverty (as \citealt{Grow2005}: 1, and many others) do: 
		that research makes invisible the broader context that created that inequality in the first place, and hides who is paying for the poor relief.}. 
	
	Surely, Pangloss would already despair over \cite{Offe2003}'s insistence on a full-employment protecting roof. 
	But with inequality, we can and should ask him an even harder question that might reveal his unreasonable optimism in starker colors.
	
	In addition to these functional reasons, there are normative and empirical reasons to demand of welfare states worthy of the label to, at least, be able to curb inequality. 
	Normatively, it seems questionable to constrain the surely emancipative agenda that once endowed welfare states to worker protection. 
	That's quite little to ask of Pangloss. 
	Empirically, we know that people care about \emph{relative} differences in access to resources \citep{Frank2005}, that they suffer from \emph{relative} inequality \citep{Pickett-2009-kx}. 
	If we are welfare state researchers and, as humanists, care about the human outcomes of institutions, maybe more than evident at Bismarck's time, today inequality \emph{is} that relevant outcome, even if and to the extent that absolute material security is achieved.
\end{enumerate}

\subsubsection[Newspeak]{Newspeak} \label{sec:newspeak}

\begin{quote}
	\emph{``When the general atmosphere is bad, language must suffer. 
	[\ldots] But if thought corrupts language, language can also corrupt thought. 
	A bad usage can spread by tradition and imitation even among people who should and do know better.''}\\
	--- George \citealt{Orwell1946}
\end{quote}

Some of the literature on European and \gls{CEEC} welfare states suffers, plainly, from bad language, especially when social scientists uncritically adopt the language of policy makers as, again, ``to explain things \emph{with}'' rather than as the thing to be explained, as they should \citep{Brubaker-2002-aa}. 
At best, this results in terminology and arguments that are devoid of meaning. 
At worst --- if not equivalently --- this results in science signing on to the ideology they are meant to disguise.

Only the mildest from of sloppy language is when social scientists use \gls{EU} terminology, without questioning whether they deliver as advertised. 
\citeauthor{Dehey2003}, for example, in an otherwise insightful article, seems to imply that \emph{actually}, union structural and cohesion funds would drive convergence (\citeyear{Dehey2003}: 566), when clearly these currently paltry funds offer merely cosmetic redistribution. 
Similarly, \citeauthor{Sipos2005} report in a chapter for the World Bank that ``after a period of transition confined more narrowly to poverty relief (...) accession to the \gls{EU} \emph{facilitates} restoration of broader and more active instruments of social inclusion [in \gls{CEEC}].'' (\citeyear{Sipos2005}: 89, emphasis added). 
Surely, whichever standard \citeauthor{Sipos2005} of social safety net have in mind here must be quite low, as, compared against the homestead of social policy --- the national mixed economy --- there is nothing much to speak of at the \gls{EU} level.

Likewise, \gls{EU} mumbo-jumbo such as ``cohesion'', ``inclusion'' and ``anti-discrimination'' should always be used with care, and quotation marks. 
As \citeauthor{Offe2003} reminds us, these are rhetorical devices, not necessarily actual policy goals (\citeyear{Offe2003}: 461). 
What is worse, they cannot be empirically falsified, nor normatively denied: 
how do you ever fail ``cohesion'', who could ever be in favor of ``discrimination''? 
These, are, in short, ideological terms, that hermetically seal off any actual policy (or lack thereof) thus labeled from political contestation.

Another favorite in this vein is ``social dialogue'' (e.g. \citealt{Durr2009}), exuding harmony, agreeability and general fuzziness. 
The choice of dialogue, maybe not just to me, seems to imply that, \emph{really}, if employers and employees --- maybe rich and poor, too? --- would only sit around a table and \emph{talk}, everything would be fine. 
And who could ever be opposed to talking? 
The economics of industrial relations, alas, are very different: 
they contain --- dare I say it? --- a \emph{zero}-sum element, they know winners and losers. 
This is not to say that employers and employees may not \emph{also} find themselves in positive-sum games that they \emph{may} solve to everyone's content. 
But if such a cooperation problem is encountered and/or solved, that needs to be explained, and must not be ex-ante assumed, by choice of words.

Terminology familiar to economists --- or rather, \gls{IFI} economists --- can also corrupt language. 
``Structural adjustment'' (e.g. \citealt{Begg2008}: 19) is such a false friend. 
It sounds like an effortless, mechanical process, one that \emph{wants} to happen. 
Surely, an unsuspecting reader might not expect that this usually means mass unemployment, pay cuts, and --- absent an insulating welfare state --- material hardship for many people. 
There is to the neoclassical economist, nothing wrong right with such transformations, and maybe they are right. 
But we ought to, at least once in every article using the term, explicate clearly what it means: 
that some economic activity will no longer occur, or not at the same price or wage as before, that many people will have to retrain, or make do with less.

Again, I want to avoid sounding like a ranting conspiracy-monger, and will therefore illustrate my gripe with Newspeak language on two examples in more depth:

\paragraph{\gls{OMC} \& Governance.} Both the \gls{OMC} and governance are heavily en-vogue with social scientists in European and \gls{CEEC} welfare states, and, as similarly hyped post-isms, are mostly defined in the negative. 
The \gls{OMC} is, apparently, \emph{not} closed --- but, one is assured, still methodical --- and somehow, \emph{not} hierarchical. 
Governance, too, is mostly \emph{not} state and \emph{not} market. 
Whoever came up with these terms might have taken a lesson out of a marketers playbook: 
against these new-new things, old-fashioned hierarchy, negotiation or government look instantly passe. 

Problems arise, when this newspeak creeps into social science. 

Governance for once, as \citeauthor{Jachtenfuchs2001} guardedly criticizes, may be a ``\emph{problematique}'' or an empirical phenomenon, but it is not a theory (\citeyear{Jachtenfuchs2001}: 259)
For the social scientists, \emph{that} is problematic. 
Say about state command and market exchange what you will, but at least we have well-articulated theories about how they operate, grounded in somewhat plausible models of human nature. 
Critique, in the social sciences, implies theory. 
State and market, because we have (competing) theories on them, we \emph{can} criticize: 
we can argue about their functions and dysfunctions, about when best to use them, and how to remedy their respective shortcomings. 
As for governance, we do not know. 
Caution dictates, that, as long as no plausible theory of such a third mode of rule, production and distribution has been explicated, we might better consider these empirical problematiques as deficient states or markets, or, possibly, hybrids thereof. 
Using, for now, these old-fashioned theories does not preclude a later, third, genuinely theorized mode nor, for that matter, necessarily implies that all such deficients or hybrids would deliver bad or no results.

The \gls{OMC}, too, is a smoke grenade. 
For starters, as \citeauthor{Offe2003} acutely notes, it is a misnomer: 
if anything, it ought to be an open method of \emph{cooperation} (\citeyear{Offe2003}: 467). 
Coordination --- as in a \emph{battle of the sexes} game --- implies that the payoff of successfully aligned strategies (far) exceeds the divergent interests that players may have in alternative strategies. 
In the paradigmatic couple story, the man prefers sports, the wife likes opera, but both want to spend the evening together. 
In an extreme case of a coordination game, such as the decision to drive on the right or left side of the road, there is no divergent interest and the game could be solved, if only players agreed on a \emph{focal} point. 
Integrating economies, building welfare states and, most clearly, harmonizing tax or other market interventions are, emphatically, no mere coordination problems. 
Here, the divergent interests of alternative strategies, when considered \emph{individually rational}, outweigh the benefits of cooperation. 
Overcoming something akin to a prisoner's dilemma of individually set tax rates or social policies would be small accomplishment. 
Still, the \gls{OMC} blithely assumes it to be a fait-accompli. 
To be sure, there \emph{are} non-hierarchical ways to solve \glspl{PD} (originally \citealt{Axelrod1980}), and if we are lucky, \gls{OMC} consultations produce such happy resolutions. 
But if, in fact, they do, it would be nice to learn just how it happened, rather than to presume it would. 

Consider, for example, the thoroughly unsatisfactory conclusion that \citeauthor{Theobald2009} draw, after 19 pages of --- supposed --- theorizing elder care systems in \gls{CEEC}:
\begin{quote}
	\emph{``Although the EU has undoubtedly gained in importance with regard to social policy in its member states, essential decisions are still made at nation­ state level and determined by actor constellations within the member states. 
	Nonetheless, national policy changes are strongly affected by external factors, including foreign models, transnational expert networks, international organizations and the European Union.''}\\
	--- \citeyear{Theobald2009}: 163
\end{quote}
Well, that is good to know. 
Who would have guessed?

Good language is not just an intellectual nicety. 
Newspeak takes the conflict, material, and politics out of policy, where, for now, we must still suspect it. 
``Governance'', in \citeauthor{Jachtenfuchs2001}'s minced words, ``almost completely ignores questions of political power and rule'' (\citeyear{Jachtenfuchs2001}: 258). 
The ``\gls{OMC}'' by its suggestion of Potemkin harmonization, is not so different from the ``political'' science that \cite{Agnoli-1989-aa} berated for ``praising a state under careful circumvention of its economic underbelly [\ldots]'' (\citeyear{Agnoli-1989-aa}: 195)\footnote{
	In the german original: 
	``Lobpreisung einer Staatsform unter sorgf\"{a}ltiger Umgehung ihres \"{o}konomischen Unterleibes [\ldots]'' (\emph{ibid.}).}.

If --- as deliberative democracy (e.g. \citealt{Elster-1998-aa}) --- these concepts imply normatively, that policy-making \emph{should} shed such unhappy remnants, they should come out and say it. 
If --- as, maybe, behavioral economics (e.g. \citealt{Tomasello2009}) --- these concepts indicate positively, that policy-making \emph{does} sometimes occur without the selfish demons of our nature that state and market are out to civilize, they should come out and prove it. 

What social scientists must not do, is to add these concepts to the ontology, \emph{under} the radar of contestation. 
However much it may pain us post-ideologues to speak of \citeauthor{Agnoli-1989-aa}an underbellies of power and material inequality, that is where ---not just figuratively speaking --- society reproduces itself, and where therefore, we must look.

\subsubsection{Bystanders}

\begin{quote}
	\emph{``If there is a hard, high wall and an egg that breaks against it, no matter how right the wall or how wrong the egg, I will stand on the side of the egg. \\
	Why? 
	Because each of us is an egg, a unique soul enclosed in a fragile egg. 
	Each of us is confronting a high wall. 
	The high wall is the system which forces us to do the things we would not ordinarily see fit to do as individuals. 
	[\ldots] 
	We are all human beings, individuals, fragile eggs
	We have no hope against the wall: 
	it's too high, too dark, too cold. \\
	To fight the wall, we must join our souls together for warmth, strength. 
	We must not let the system control us --- create who we are. It is we who created the system.''}\\
	--- Haruki Murakami (Jerusalem, 2009)
\end{quote}

To me, the most outrageous attitude to take on European integration and \gls{CEEC} welfare state is no attitude at all. 
Being merely a bystander, for many social scientists, appears not only to be acceptable, but the professional thing to do. 
As ``normative'' has become a dirty word in many social science departments, and prescriptions something to snicker at, ``disinterestedness'' now appears the lone, unchallenged criteria for good science. 
Somewhere along the way \emph{a} political science (politische Wissenschaft) has turned into \emph{apolitical} science (Politikwissenschaft).

This is evident in much of the literature on European and \gls{CEEC} welfare state, that, for the most part, refrain from judgment. 
Even eminent scholars in this field now fetishize disinterestedness: 
Scharpf asked me in a 2011 e-mail whether ``as a policy researcher, I wanted to work on (a however likely, initiated by whomever) revolution [sic!] or contribute to \emph{real} political decision processes in \emph{real} existing polities'' (emphasis added). 
\marginpar{Do not quote or circulate the Scharpf quote. 
I have not cleared it with Scharpf.} %!marginpar
Hemerijck, after a presentation on an alternative tax regime opined that he considered history smarter than himself --- or myself, for that matter.

This science turned ignorant is neither humanist, nor, I would maintain, a worthy heir to the spirit of enlightenment that bore its methods 250 years ago.

The operative metaphors here are that of a bystander or a witness. 
Of course, witnesses must not embellish their accounts to suit their ideology, or theory: 
as far as possible, they should report only the facts. 
Also, witnesses must not take the law in their own hands: 
passing judgment is up to the judges. 
This code holds for social scientists, too: 
they should not cook the data to suit their theory, or, present facts that cannot be falsified. 
Social scientists, too, are not philosopher kings, or guardians: 
the democratic sovereigns, not the professors, call the shots.

Still, a witness bears a responsibility: 
to alert others to the injustice she has observed. 
Social scientists, are, whether they like it or not, often the lone expert witnesses on a scene. 
In our modern, complex world, there will, increasingly, be no one else to sound the alarm. 
Take the demise of \gls{CEEC} welfare states for an example: 
who, if not the social scientists should alert the wider public that \emph{other welfare states} are possible? 
Only social scientists and economists can conceive of better designed mixed economies and blow the whistle: 
``\emph{no}'', they should shout, the hardship and instability suffered by the people of \gls{CEEC} is not materially inevitable, it is a particular choice made by particular people. 


Pointing to the alternatives to the status quo does not, as many seem to fear, taint all empirical research. 
There are still plenty of entirely positive questions, for example, how to best organize a public service such as waste disposal. 
Social scientists as witnesses need not and should not have a position on this: 
it is \emph{not} a normative, but a positive question. 
They can test which kind of observed, real existing and hypothetical, reasonably imagined way of waste collection is more efficient or equitable. 
Once that work is done, social scientists can compare reality to what they know is possible, and inform people about the difference and why it exists. 
To simply note that, apparently, waste disposal was privatized because city budgets were tight, however, is to not bear witness, but to be a bystander.

Because so many of the other people on the crime scene have never learned what is positively possible, and what not, they will often be unsure about whether a particular event was man-made or materially inevitable. 
For that reason, they need first, economists to clarify what is possible in a scarce world, and second, social scientists to explain why and how the man-made events were made.

This is not a new thing to ask of social scientists. 
Proponents of public sociology (Gans 1988), and, before that, \citeauthor{Mills-1959-aa}'s call for sociological imagination meant exactly this: 
to, true to their emancipatory heritage as science, \emph{enlighten} fellow human beings about the abstractions and alternatives facing the modern world.

It is particularly important in the realm of welfare state design in Europe, and the suffering \gls{CEEC}, where the relevant abstractions and alternatives are vanishing into oblivion, and are being replaced by ideology-infused newspeak, ever-optimist Panglosses and depoliticized TINAs. 
To do meaningful second-order, sociological or political science work on the welfare state, you first have to do some first-order, economic work, if nothing else, because it lets us talk about welfare as if people mattered:
\begin{quote}
	\emph{``In the first instance, we are interested in the welfare state because we are interested in human welfare.''}\\
	--- \citeauthor{Haggard2009} (\citeyear{Haggard2009}: 236) 
\end{quote}

\subsection[Suspects]{Suspects: Hunching Second-Order Theory}

\begin{quote}
	\emph{Only from the margins can you see well.}\\
	--- Michel \cite{Foucault-1972-aa} 
\end{quote}

I have argued that the existing literature on the second-order theory of the welfare state is pretty bad, at least because it does not engage the relevant first-order alternatives. 

But what do I suggest instead? 
Not much, mostly, because I have no data to build, let alone test or support any theory. 
Bearing witness, here, as always, entails positive questions. 
This second-order question of \emph{why}, by the possible and preferable hypothetical of an intact mixed economy, European regional integration is so tilted, and its welfare states in such disarray, is an entirely positive question.

The \gls{CEEC} periphery of the \gls{EU} is probably a good place to gauge the second-order theory of european integration. 
It is, according to \cite{Foucault-1972-aa}, at those (continental) margins where you can see clearest the hegemonic discourse, and probably other non-structuralist dynamics, too.

I can only present some hunches.

\subsubsection[No Enlightened Understanding]{No Enlightened Understanding}

\begin{quote}
	%\emph{``Europa ist ein sch\"oner Kontinent. 
	%Es ist freundlich und sehr sauber. 
	%Europa hat mehrere sch\"one M\"adchen.''}\\
	\emph{``Europa is a nice continent. 
	It is friendly and very clean. 
	Europe has a couple of pretty girls.''}\\
	--- Andrei, 17, Romania (in \href{http://eurolektionen.de}{Eurolektionen} \citeyear{DeRuffray2010})
\end{quote}

\begin{quote}
	%\emph{`Mit leerem Kopf nickt es sich leichter.'} (Sprichwort aus Deutschland)\\
	\emph{``An empty head makes it easy to nod.''}\\
	--- From Germany
\end{quote}

The first, immediate hunch at a second-order theory is that we --- including our social sciences --- have so thoroughly, misunderstood the mixed economy, that the european polities are at the mercy of the few, interested and educated enough to make informed, but self-interested choices. 
European democracy would then violate the criterion of ``enlightened understanding'' \citep{Dahl-1989-aa}. 
The alternatives presented to the sovereign, and the choices which she assumes have so far detached from the actual abstractions of the mixed economy that any remaining political choice is, in the best case, spurious and, in the worst case, a rigged game.

\subsubsection[False Consciousness]{False Consciousness}

\begin{quote}
	\emph{``We did this to ourselves.''}\\
	--- Cailin, 17, Romania (in \href{http://eurolektionen.de}{Eurolektionen} \citeyear{DeRuffray2010})
\end{quote}

A variant of this first hunch assumes, that maybe, the children of the revolutions in \glspl{CEEC} hold systematic misunderstandings of the mixed economy. 
As \citeauthor{Bonker2006} reminds us, the two constituent systems were indistinguishable under state socialism with its ``tight coupling of the economic and political sphere'' (\citeyear{Bonker2006}: 35). 
It is, maybe, little surprise that the people who courageously threw out state socialism, but were still not used to ``clear demarcation lines'' between institutions as distinct as the ``state budget, state-owned enterprises and the central bank'' (\emph{ibid.}: 36), did not turn into expert policy makers of mixed economies over night.

Vividly remembering the horrible inadequacies of the ancien regime (e.g. \citealt{Szikra2009}, \citealt{Millard1992}), one might assume that people, moreover, held a specifically  ``anti state ethos and a rejection of many social functions of the allegedly 'over-protective state' '' (\citealt{Millard1992}: 130). 
40 years of socialist mismanagement might well have given the state a bad name, and thereby jinxed any effort to build well-balanced mixed economies. 
Along these lines, \citeauthor{Inglot2008} reports that ``in companson with continental Western Europe, East Central European countries have developed a spectrum of political forces that has clearly shifted in favor of market-liberal, nationalist, and conservative ideologies and policies of the political right'' (\citeyear{Inglot2008}: 212, on labor \citealt{Crowley2002}, more broadly \citealt{OrenOuto2001}).

This provides a plausible, partial explanation for the defunct mixed economies of \gls{CEEC}, but, to be clear, it is still history first as tragedy, then as farce \citep{Marx1852}. 
That, plagued by decades of state socialism, their citizens would subsequently have to endure an impotent welfare state when they most needed it in times of transition must surely be another instance of a revolution devouring its own children.

\section{Growth and Solidarity} \label{sec:growth_solidarity}

\begin{quote}
	\emph{``The Congress shall have Power [...] To regulate Commerce with foreign Nations, and among the several States, and with the Indian tribes''}\\
	--- Constitution of the United States: Article I, Section 8, Clause 3\footnote{
		Known as the commerce clause.} 
	(Philadelphia, 1787)
\end{quote}

Let me now return to the first order perspective: 
what, from here, is to be done about european integration and its failing welfare states?

Crucially, we must, again, understand that economic integration must always beget more economic and political integration, that, production at great economic scale implies solidarity at the same scale.

\emph{That} is the story of economic integration and material progress. 
Living on a scarce and harsh planet, for which, alone, we are ill-equipped, we escaped the \citeauthor{Malthus1798}ian curse of overpopulation and starvation by scaling up our production. 
That is, to this day, how we remake an arithmetically growing material world, to feed our geometrically growing hunger: 
we join forces to bend upward the production curve, and wrest from it above-linear returns. 
To reap such economies of scale, in agriculture \citep{Diamond1997}, in the production of violence \citep{Tilly-1985-aa}, or in europe-wide automotive engineering \citep{Krugman-1980-aa}, we have to master the feat of cooperation in the face of atomistic incentives. 
Akin to ever-increasing entropy, the arrow of pre-historic organic life, and history of human life on earth progresses by increasing complexity, specialization, but also, always, mutual dependence. 
\citeauthor{Wright1994} thus describes the transformation of individual cells into higher organisms: 
they become richer, more resilient, but they must also sacrifice, and will do so only to the extent that they have successfully merged their genetic instructions (\citeyear{Wright1994}: Chapter 7, 8). 
%better double-check this stuff. 
%This isn't quite right yet.

And so it is with european, or other regional economic integration: 
opening up commerce to one common market makes everyone richer by the magic of comparative advantage --- if not necessarily by the same amount. 
But, if this association is to remain stable, some other elements of human organization have to follow to that higher level, if they are not to perish and destabilize. 
In the \gls{EU}, continent-wide commerce also requires continent-wide taxation, regulation and monetary policy, if the formerly stable and re-productive system of a mixed economy is to remain operative.

The genius of the otherwise hyper-federal, and hyper-liberal US Constitution is that it foresees this functional necessity in its commerce clause\footnote{
	\ldots the reach of which was only recently discussed in the 2012 US Supreme Court ruling on the Affordable Care Act.}.
It stipulates that, precisely as commerce traverses the otherwise autonomous states, the federal polity asserts itself. 
The German constitution knows a similar norm:

\begin{quote}
	\emph{``The Federation shall have the right to legislate on matters [\ldots] if and to the extent that the establishment of equivalent living conditions throughout the federal territory or the maintenance of legal or economic unity renders federal regulation necessary in the national interest.''}\footnote{
		In the German original:
			\begin{quote}
				\emph{``Auf den Gebieten [\ldots] hat der Bund das Gesetzgebungsrecht, wenn und soweit die Herstellung gleichwertiger Lebensverhältnisse im Bundesgebiet oder die Wahrung der Rechts- oder Wirtschaftseinheit im gesamtstaatlichen Interesse eine bundesgesetzliche Regelung erforderlich macht.''}\\
				--- Grundgesetz der Bundesrepublik Deutschland: Artikel 72, Absatz 2 (Bonn, 1949)
			\end{quote}}\\
	--- Basic Law for the Federal Republic of Germany: Article 72, Paragraph 2 (Bonn, 1949)
\end{quote}

It too, displays the same genial insight into the relationship between economic and political integration: 
as the commerce clause, it does not compel any particular social or other policy, but, if economic integration has occurred, endows the democratic sovereign to rule at the newly emerged, higher level of societal organization.

%add more from the Papier debate here?

What the \gls{EU} needs, is not more stringent subsidiarity, but it's inverse: 
a commerce clause.

\subsection[Difference]{Difference} \label{sec:ID-Difference} 

\begin{quote}
	\emph{``The boundary is not a spatial fact with sociological consequences, but a sociological fact that forms itself spatially.''} \\
	--- Georg \citeauthor{Simmel1903} (\citeyear{Simmel1903}: 142)
\end{quote}

To this cosmopolitan prescription, people often reply with a positive and normative critique of difference, or --- the dreaded I-word --- identity. 

Positive critics of would-be United States of Europe point out that ``empirical conditions'' for democracy and statehood are not given: 
there is, supposedly, no european people, no european public sphere and no european party system\footnote{
	This according to former Hans-J\"{u}rgen Papier, former president of the German Constitutional Court at a panel discussion in Berlin on October 17, 2013.}.
Similarly, \citeauthor{Scharpf1997} more carefully 	speaks of ``a pre-existing sense of community [\ldots], of common history or common destiny, and of common identity'' on which democracy depends, but ``which cannot be created by mere fiat'' (\citeyear{Scharpf1997}: 20).
Such skepticism about european democracy seems to be founded in the assumption of all hitherto, modern statehood as ``socioculturally homogenous'' (e.g. \citealt{BeckGrande-2007-aa}: 93) --- a notion that \citeauthor{Scharpf1997}, somewhat illogically, rejects (\citeyear{Scharpf1997}: 20). 
Historically and conceptually, that is not true: 
(democratic) states made nations at least as much as nations made (democratic) states (\citealt{Gellner-1983-aa}, recently \citealt{Schmitter1999}: 934). 
But even aside the history, methodological nationalism is also epistemologically fallacious. 
Again, \citeauthor{Brubaker-2002-aa}'s dictum applies: 
if, indeed, we have no European democracy because people feel as citizens of its \gls{MS}, than that is ``what we want to explain, not what we want to explain things \emph{with} (ibid.: 165, emphasis in original), or as \citeauthor{Simmel1903} said, a merely spatially formed, but sociological fact.

Normative critics of would-be United States of Europe --- maybe more honestly --- argue for smaller polities to preserve diversity. 
Surely diversity of identities and lifestyles is something worth protecting, but this argument perilously confuses what Jonas \cite{Marx2012} has recently identified as \emph{political} and \emph{ontological} identity. 
Ontological identity, for \citeauthor{Marx2012}, arises only as actual people get to know each other, their character, and preferences. 
Ontological identity, therefore, exclusively resides within small groups or institutions, such as a circle of friends, family or a small firm: 
it cannot be scaled up\footnote{
	\cite{Marx2012}'s ontological identity is similar to \cite{Brubaker-2002-aa}'s definition of groups (as opposed to groupisms):
	they, too, are mutually interacting, as friends and family are, and not merely based on categories, such as ``Greek'' or ``German''.}.
Political identity, by contrast, can be scaled to arbitrary dimensions, as it arises wherever people form a democratic polity, as their lives become more intertwined by, for example, economic integration. 
Maybe inspired by the proto-deliberative \cite{Ahrendt1958}, \citeauthor{Marx2012} takes pains to remind readers that political identity is \emph{not} merely an assortment of particular, material interest\footnote{
	\citeauthor{Marx2012} points out that etymologically, ``interest'' refers to that between and beyond individuals, and their needs.}, 
but, by contrast, that one takes on a political identity purely as a choice, if and to the extent that particular, material \emph{needs} are fulfilled. 

Both the traditional conflation of ``nation (=) state'' and the difference critique of European integration confuse ontological with political identity. 
The nation state assumes \emph{ontological} homogeneity, a dangerous fiction that has gone horribly awry more than once in the 19th and 20th century. By this rather sorry standard --- political integration through homogenization --- European integration really does look quite worrying. 
If not only all Germans must be punctual, Sauerkraut-eating and Goethe-reading, but all Europeans must share the same ``judeo-christian values'', the ``Occident'' really is in trouble.

Pluralist democracy and its supposed home, the nation state, harbor another ideological fiction: 
that polities are, or ought to be, merely households, or even families writ-large. 
The first-order, material reality of Europe, or any other modern society, is of course truly that of a household, \emph{oikos}, or economy, ruled by interests, or even needs. 
But democracy is, and ought to be, more: 
it \emph{is} the one, purposeful (not aimless!) process by which societies come up with their second-order answers, and, thus, must ideally be \emph{isolated} from material interest, or need. 
Pluralist democracy, however, extends the incentive logic to government: 
here, too, as in a household, people have pre-social interests, including material needs, that are simply aggregated into decisions. 
Akin to markets, pluralist democracy knows only procedural rules of ``level playing fields'', but allows no substantive standards. 
Preferences are a black box, beyond reproach or qualification. 
The nation state, likewise --- already etymologically (\emph{natio}, that which has been born) --- presumes (incorrectly) common descent, and therefore suggests that nationals share some common, material interest. 
By this standard, too --- politics as interest aggregation, or agglomeration --- further European integration seems doomed, if not dangerous. 
If not only in the Bundestag, MPs are supposed to vote their constituents' (or, more malignly, their lobbyists') pocket books, but in the European Parliament, too, MPs vote their countries --- much different --- welfare, the union may fray.

There are more hopeful alternatives to nation state democracy, and therefore, hope for European democracy. 


First, there is no reason why further European political integration would require any ontological homogeneity: 
just as people drive much the same cars in Portugal as in Sweden, they can be regulated by the same emission's trading and taxed by the same code, without anyone having to give up what they eat, how they dress or to which (or none) god they pray. 
In truth, this homogeneity has not existed in the \gls{MS}, except as nationalist ideology or xenophobia-baiting. 
A rural Bavarian may not --- other than her passport and other institutional vestiges --- have much more in common with a Hamburger, than with a Parisian, or an Athenian. 
They do not even really speak the same language, as is so readily assumed \citep{Kymlicka-2001-aa}: 
talking about the economic abstractions of the Euro crisis, a \emph{Financial Times Deutschland}-reading German will have almost as much trouble understanding a \emph{BILD}-reading compatriot as a \emph{Daily Mail}-subscriber from the UK. 
Surely, such absence of \emph{logos} is dramatic for democracy, if it is ever to allow \emph{action} \citep{Ahrendt1958}, but it is not much more of a challenge to establish it between, than within \gls{MS}.

Second, democracy must, and can be more than ``two wolves and a sheep voting on what to have for dinner'', as Benjamin Franklin remarked. 
The economic abstractions and material view of the mixed economy I have presented here may be misunderstood to suggest that a European intact mixed economy, too, is just another, more efficient way to organize a polity as a large oikos, as \citeauthor{Ahrendt1958} might have deplored. 
That is not so: 
european democracy is not instrumental to resolving the \glspl{PD} and other inefficiencies and inequities of currently defunct mixed economies, but the other way around, an intact mixed economy is merely instrumental to restore the freedom from need to \emph{act} democratically as citizens in a newly potent, European polity.

%\subsection{Postwar}

%\begin{quote}
%	\emph{``Meanwhile [in 1989], across the Leitha and Danube rivers just a few kilometres to the east, there lay the ‘other' Europe of bleak poverty and secret policemen. 
%The distance separating the two was nicely encapsulated in the contrast between Vienna's thrusting, energetic Westbahnhof, whence businessmen and vacationers boarded sleek modern expresses for Munich or Zurich or Paris; and the city's grim, uninviting S\"{u}dbahnhof: 
%a shabby, dingy, faintly menacing hangout of penurious foreigners descending filthy old trains from Budapest or Belgrade.''}\\
%	--- Tony \citeauthor{Judt2006} (\citeyear{Judt2006}: 5)
%\end{quote}

%!marginpar
%\marginpar{Because this section is so small and I have so little to say on it, it might better go. 
%I just love the Judt quote.}

%In his epic Postwar tome, \citeauthor{Judt2006} chronicles the divergent paths that drove apart eastern and western Europe, as the post-war settlement inflicted on them the historical accident of Cold War division. 
%The wounds of this continental tear have not healed, in fact, until and unless we install democratically governed, intact mixed economy on the continent, it will continue to pain and undermine the union, only superficiously covered by the paltry Band-Aid of the common market.

%If there is a Postwar history to building welfare states in \gls{CEEC} is that so far, they will only succeed if we finally close the tear of 40 years of state socialism, and agree on how fast to have the East catch up, and how to make it happen. 

\subsection[Weimar]{Weimar}

\begin{quote}
	\emph{``There’s a storm coming, Mr. Wayne. 
	You and your friends better batten down the hatches. 
	Because when it hits you’re all going to wonder how you ever thought you could live so large and leave so little for the rest of us.''}\\
	--- Selina Kyle (``Catwoman''), Dark Knight Rises (2012)
\end{quote}

\begin{quote}
	\emph{``But who can say how much is endurable, or in what direction men will seek at last to escape from their misfortunes''}\\
	--- John Maynard \cite{Keynes1936}
\end{quote}

Optimists, not just of the Panglossian kind, are always inclined to see glasses half-full, and so many argue that any, even imperfect or negative European integration is better than none. 
That is not so much optimistic, as it is hopelessly naive. 
The metaphor is flawed. 
 ocieties are not just glasses half-full or half-empty, they are dynamic systems, residing within, and reproducing the walls of this glass: 
 half-\emph{built} glasses, as the \gls{EU} defunct mixed economies, are not merely waiting to be filled, they can also drain of all remaining water, or topple and fall over.

The democratically governed mixed economy is \emph{the} institutional achievement of the century, and as a social compact, it is fragile. 
As \citeauthor{Offe2003} points out, ``it is much more likely that a European-styled [mixed economy] capitalism transformed itself into a liberal model'' than the other way around, or, appropriating Lech Walesa, ``it is easier to make fish soup out of an aquarium than the other way around'', because it --- like the mixed economy --- depends upon ``supportive dispositions of a cognitive as well as moral kind'' (\citeyear{Offe2003}: 446). 
A mixed economy, as the democracy that governs it, is ``easily lost, but never finally won'', as the first African-American federal judge William H. Hastie ominously warned. 

We have already lost much of the mixed economy, and we are now risking to harm or loose democracy, too: 
not just democratic rule \emph{of} the \gls{EU}, but democracy \emph{in} Europe, too (\citealt{Scharpf1997}: 19).

Democracy needs not just a legitimacy of inputs, but also of outputs (on Europe see \citealt{SchaGove1999}). 
In \citeauthor{Dahl-1994-ab}'s (\citeyear{Dahl-1994-ab}) terms, democracies need to be ``system effective'', in \citeauthor{Zurn-2000-aa}'s \citeyearpar{Zurn-2000-aa} apt words, democracies need to be ``output congruent'': 
people must be able to choose any (liberal) policy they want to govern a given polity. 
The \gls{EU}, currently violates output congruency: 
with an impotent, largely defunct mixed economy, the people of Europe cannot have all materially possible policies to improve their currently grim life chances, including crushing trade imbalances, demoralizing structural unemployment, wild economic cycles, public squalor and rampant inequality. 
In the half-built, supposedly \emph{sui generis} glass of Europe, there is a wide mismatch between the two walls: 
the exchange components of the mixed economy roam the continent, while most of the command components are confined to the constrained nation state. 
As a result, the glass is heavily leaking water, both in efficiency and equity.

Over time, the hardship that this arrangement inflicts on people, especially in the poorer, more austere \gls{MS} (Greece), and when the imbalances and crises unload (Spain, Ireland), will corrode the inputs of popular rule, too, and fray the social contract. 
\citeauthor{Offe2003} has already seen the grave-diggers of liberal democracy, capitalizing on the resulting popular discontent of such ``permanent austerity'' \citep{Streeck2010c}: 
\begin{quote}
	\emph{``Also, a third voice, luckily with much less resonance, is making itself increasingly heard in European politics, a voice which claims that the social security of workers (as well as the protection of citizens from violent crime), on the one hand, and efficiency of production and competitiveness, on the other, can only be reconciled if national borders are sealed to the influx of foreign people, foreign workers, foreign goods, and those praying to `foreign' gods. 
	Since the mid-1990s, integrating Europe has seen the sometimes sudden and spectacular rise to electoral success of figures such as Pia Kjaersgaard (Denmark), Umberto Bossi and Gianfranco Fini (Italy), Pim Fortuyn (Netherlands), with Jean Marie Le Pen (France), J\"{o}rg Haider (Austria) and Carl Hagen (Norway) being among the pioneers of this new field of populist political entrepreneurship. 
	Le Pen has described himself in the 2002 French electoral campaign as being a leftist in social affairs, a rightist in economic affairs, and a nationalist for everything else. 
	This formula, which is designed to resolve the tension between liberal market freedom and welfare state status rights by ethno-nationalist, xenophobic, and anti-European appeals, is applied by his rightist populist colleagues as well."}\\
	--- \citeauthor{Offe2003} (\citeyear{Offe2003}: 454)
\end{quote}

The ground of popular resentment will only grow more fertile as the Euro crises goes on. 
The crippling economic depressions of the south, and the ever-insufficient, multi-billion bail-outs from the north are political dynamite, waiting to be ignited by fringe populists. 


If the extreme right, or extreme left, and/or nationalists start playing with this fire, the moderate voices of social democracy, liberalism and conservativism will have little to douse the flames, because in fact, both the polities in the north and south \emph{are} faced with thoroughly unattractive alternatives.

In the now often wretched south, policy makers must either accept the conditional straightjacket imposed by the \gls{EU} and its rich sponsors, or jump the cliff of secession and sovereign default, all but ensuring all-out economic collapse under autarky. 
In the still largely isolated north, policy makers must either continue to periodically support often (but not always) failing governments in the south to alleviate the gravest imbalances, or splinter the union and forego the years, maybe decades of economic growth that wide integration brought.

European democracies thus are also no longer \emph{input congruent} \citep{Zurn-2000-aa}, in addition to the already grave, but homemade democratic deficit of a heavily intergovernmentally-biased institutional setup. 
Faced with such non-alternatives, the people of Europe are no longer subjected only to decisions in the making of which they had a say. 
For the past two years, every couple of months, German or Greece executives, legislators and voters are asked to choose between bail-out or break-up, austerity or abyss, always at gunpoint, with the entire european project held hostage (e.g. \emph{Grexit}, followed by Portugal, Spain, Italy, followed by doomsday). 
If the people of Europe merely get to choose between a rock and a hard place, popular sovereignty becomes a farce.

\citeauthor{Offe1998}'s \citeyear{Offe1998} intuition is, tragically, fully borne out:	
\begin{quote}
	\emph{[\ldots] that every interim solutions between the extremes of intact national sovereignty on the hand, and complete european supranationalty of a European Federation will, inevitably, violate both the reference point of welfare state protection and that of democratic legitimacy.}\footnote{\label{fn:Offe_regress}
		In the German original:\\
		\begin{quote}
			\emph{``Meine These ist, dass jede Zwischenl\"{o}sung, die zwischen diesen beiden Extrempolen intakter nationalstaatlicher Souver\"{a}nit\"{a}t einerseits, einer kom-plettierten europ\"{a}ischen Supranationalität in der Gestalt eines föderalen eu-ropäischen Staates andererseits gefunden wird, zwangsl\"{a}ufig beide Be-zugswerte verletzt, den des wohlfahrtsstaatlichen Schutzes ebenso wie den der demokratischen Legitimation. 
			Demnach k\"{o}nnte man im Blick auf die europäische Integration einen Abstieg auf jener Leiter vermuten, die T. H. \cite{Marshall-1950-aa} sich als Modell für den Prozess der europ\"{a}ischen politischen Modernisierung vorgestellt hat. 
			Die drei Stufen dieser Leiter sind bekanntlich die kumulative Durchsetzung liberaler, demokratischer und sozialstaatlicher Rechte. 
			Die Frage ist, ob im Prozess der europäischen Integration die demokratische und die sozialstaatliche Stufe in r\"{u}ckw\"{a}rtiger Richtung passiert werden und im Ergebnis der Euro-B\"{u}rger allein mit der Rechtsausstattung eines (neo-)liberalen Marktteilnehmers dastehen wird.''}\\
			--- \citeauthor{Offe1998} (\citeyear{Offe1998}: 41)\\
		\end{quote}}
	--- \citeauthor{Offe1998} (\citeyear{Offe1998}: 41)
\end{quote}

The political fringes will prosper, as they exploit these violations, both from the extreme left and the extreme right. 
The extreme left will fan the flames of societal disintegration by pitting liberal freedoms --- especially those of property and contract! --- against welfare protection and democratic sovereignty. 
Merely a change in tone, the extreme right will push societal regress by pitting all cosmopolitan integration and solidarity against always national welfare and sovereign rule. 
This new game is rigged against liberal democracy and the moderate voices (or cross-cutting cleavages, \citealt{LipsetRokkan-1967-aa}) on which it so thoroughly depends. 

What the glass-half-full-optimists forget is that societal, political and economic integration can reverse course, too. 
The self-reinforcing dynamics of integration operate in \emph{both} directions: 
more integration begets more integration (as Monnet had hoped, according to \citealt{Schmitter1999}: 948), but regress also begets more regress. 
As a matter of fact, the neo-functionalists are right, if one-sided (Tranholm-Mikkelsen 1991 as cited in \citealt{Bieler2003}: 1)

The course of economic disintegration is quite clear: 
literally the milli-second that an over-indebted \gls{MS} is thought to leave the \gls{EMU}, anticipating massive devaluation, all holders of cash in, say, Greek institutions, will take their money out\footnote{
	In fact, the run on Greece is already underway, with the country having lost almost a third of its domestic bank deposits by May 2012, according to \cite{TheEconomist2012}.}. 
	
Faced with such an economy-wide bank run, the Greek government will be forced to install capital controls, effectively leaving the common market. 
Without international finance, the remaining trade, too, will be greatly constrained. 
Just the credible rumor of \emph{Grexit} could, thereby, almost in an instant, unravel 21 years of integration since the country joined in 1981.

We tread the course of political disintegration if --- outflanked by populists --- our \gls{MS} democracies increasingly re-embrace the supposed trade-off between national interest and political unification, and again entertain the nationalist politics that the union was built to overcome, in a scenario of decay similar to that painted by \citeauthor{BeckGrande-2007-aa} (\citeauthor{BeckGrande-2007-aa}: 339ff, or \citealt{Schmitter1999}: 947).

Writ large on society, this is the disintegration that \citeauthor{Offe1998} presciently feared in \citeyear{Offe1998}:
\begin{quote}
	\emph{[\ldots] one could suspect a downward descent on the ladder that T. H. \cite{Marshall-1950-aa} suggested as a model for european political modernization. 
	The three rungs on this latter are the cumulative expansion of liberal, democratic and social rights. 
	The question is, whether during the course of european integration, we will pass the democratic and welfare state stages on our way down, and, as a result, european citizens will be equipped merely with the rights of a (neo-)liberal market participant.}\footnote{
		For the german original, see footnote \ref{fn:Offe_regress}.}\\
	--- \citeauthor{Offe1998} (\citeyear{Offe1998}: 41)
\end{quote}

We have been here before: 
1918. 
Torn apart by nationalist antagonisms and frayed by beggar-thy-neighbor responses to depression, economies \emph{can} disintegrate, as the havoc-wreaking de-globalization of the interwar years showed. 
Strangled by economic depression, convulsed by trade and monetary imbalance, enslaved by material hardship and disheartened by a political system that would not and could not offer remedy, liberal democracies \emph{can} crumble into such mob rules as fascism or stalinism, as people, in their despair, turn to the easiest answers.

It does not take another genius to, as \cite{Keynes1936} did in 1918, anticipate the economic consequences of this particular peace, and to recognize how fragile this mode of one-sided, one-legged, market-only European integration is. 
It is both the greatest strength and greatest weakness that democracy, to thrive and to persist, must be able to complement market production and distribution with a plan. 
Whether in the German Empire, New Deal America or  Postwar West Germany, when under --- or just forestalling --- popular rule, the societies of Bismarck, FDR/LBJ and Adenauer/Erhard always matched competitive markets with strong states, each for their own reasons: 
to ward off democracy \citep{Leibfried}, to protect it from the robber barons \citep{Wapshott2011}, or to compete with a socialist neighbor \citep{Judt2006}.

To this day, not a single developed, liberal democracy has withstood the test of time, that did not stand firmly on \emph{two} legs, one of market exchange, and one of state command. 
Those who relied merely on command disgraced themselves in corruption, waste and totalitarianism. 
Those who relied merely on exchange crumbled under the assault of extremists, and inequality. 

Compared with even the most market-liberal (merely electoral) democracy, the United States, the European Union is a one-legged cripple of a mixed economy (e.g. \citealt{Bordo2011}). 
It does not even have a commerce clause, or union-wide bonds.

We have seen before, where one-legged polities can stumble: 
in the two-fold 31-year war that ravaged the world after 1914 and the singular catastrophe of Hitler Germany.

This is not alarmism, this what the precautionary principle demands of us. 
If everything is at stake, you do not run the risk of even half-full, half-built glasses.

That, especially, is the burden borne by Germans of every generation after 1945, who, 
\begin{quote}
	\emph{``Conscious of their responsibility before god and men, inspired by the determination to promote world peace as an equal partner in a united Europe [\ldots]''} \\
	--- Preamble of the Basic Law for the Federal Republic of Germany (Bonn, 1949)
\end{quote}
must never again leave an impaired polity to be preyed on by the vultures of liberal democracy, always lurking at the fringes. 
Germans, above all, should cherish the mixed economy, insist that european integration be thus righted and show the solidarity with other \gls{MS} on which their post-1945 economic and democratic miracle thrived and, to this day, utterly depends.

That, too, is the spirit to which all peace loving nations have subscribed, 
\begin{quote}
	\emph{``[\ldots] to save succeeding generations from the scourge of war, which twice in our livetimes has brought untold sorrow to mankind [\ldots]''}\\
	--- Preamble of the Charter of the United Nations (San Francisco, 1945)
\end{quote}
and that now, must be re-kindled: 
that everywhere and always, a social contract to preserve peace, liberty and prosperity, is a fragile achievement, that must not be thoughtlessly abandoned.

Negative regional integration always puts in peril that social contract. 
If a united Europe is to be the exception to its own history of war and terror, it must not stop at the half-way point. 
If it is to stay, it must complete full, positive integration, re-grow the limbs of an effective mixed economy and take to heart the lessons of Weimar: 
never again must we stay stranded on one leg, half of the way, for we may fall back and loose it all.

%===Notizen Papier

%Europäische Einigung leidet nicht an zu viel, sondern zu wenig europäischen Kompetenzen, vor allem in der Steuer-, Sozial- und Strukturpolitik. 
%Nicht, oder mindestens nicht nur mangelnde Subsidiarität ist das Problem, sondern gefährliche Arbitrage. 
%Wo ökonomische Aktivität (dankenswerter- und wohlstandsbringenderweise) europa- und weltweit organisiert ist, wird es für Nationalstaaten zunehmend schwieriger, und unmöglich eben diese Marktaktivitäten zu regulieren und zu besteuern, ohne dass Verlagerung droht. 
%übrigens: 
%die oft kritisierte Brüsseler Regelungswut zeugt nicht selten von sachlicher Unkenntnis. 
%In Ordnungspolitik, wie anderswo, steckt der Teufel oft im Detail, und eben dort GIBT es gute Gründe für EU-Regelung, etwa von der Krümmung von Gurken (ist einfacher zu verpacken).
%Europäische Demokratien -- nicht nur EU Demokratie -- wird zunehmend defizitär weil sie sich nicht mehr, wie etwas das Deutsche GG vorschreibt, beliebige Sozialpolitiken geben können. 
%Das "soziale" unter den Staatsstrukturprinzipein ist im GG -- aus sehr guten Gründen, wie uns Prof. Papier erinnert -- nicht in der Verfassung geregelt, sondern dem Gesetzgeber überlassen. 
%Das heisst aber auch: 
%Der Gesetzgeber müsste in der Lage sein, im Rahmen von Freiheitsrechten und Eigentumsschutz beliebige Balancen zwischen Markt und Staat, Effizienz und Gerechtigkeit, usw. einzugehen. 
%Diese Möglichkeit ist dem Souverän gegenwärtig genommen: 
%er ist seiner zentralen Instrumente, der Steuer und der Regulierung, beraubt.
%trotzdem: 
%hat natürlich europäische Demokratie eigene Defizite. 
%Die EU ist zu intergouvernmental, zu wenig direkt repräsentativ-demokratisch. 
%Das liesse sich aber ändern: 
%durch ein starkes Parlament.
%Gleichzeitig hat europäische wirtschaftliche Einigung enormen Wohlstand geschaffen, und tut das weiter: 
%Handel und offene Grenzen sind (fast) immer gut. 
%Es sollte also kein zurück zu geschlossenen Grenzen geben.
%Europäische Demokratie KANN es geben. 
%"Empirische Argumente": 
%Kein europäisches Staatsvolk, keine europäische Medienöffentlichkeit, keine europäische Parteienlandschaft. 
%empirisch fragwürdig: 
%historisch sind Nationalstaaten durchaus auch anders herum gewachsen
%nimmt zu unkritisch die Kategorien des Nationalstaates an: was teile ich mit einem Bayern oder was unterscheidet mich von ihm, was mich nicht auch von einem Portugiesen unterscheidet? 
%praktisch unmöglich:
%ökonomischer Integration muss IMMER politische Integration folgen, sonst wird der Sozialvertrag impotent.
%unpolitisch: In Politik geht es normatives, nicht um empirisches. 
%Die politische Frage ist nicht: gibt es einen europäischen Demos, was immer das sei, sondern: wie kommen wir dahin?
%Und deshalb: wir müssen das den Unionsbürgern ERKLÄREN: das segensreiche wirtschaftliche Einigung immer politische Einigung, und besonders, Solidarität folgen muss. 
%Das hat im Nationalstaat das erste mal, seit 1949, einigermassen erfolgreich, freiheitlich und mit breitem Wohlstand, geklappt. 
%Das geht auch anderswo.
%Es MUSS auch auf europäischer Ebene gehen: es ist zu riskant, den Zusammenhang zwischen wirtschaftlicher Integration und politischer Solidarität aufzugeben. 
%Kein politisches System, erst recht keine Demokratie hält extreme Ungleichheit, Arbeitslosigkeit und periodische Krisen auf die Dauer aus. 
%Das sollten wir, "Im Bewusstsein unserer Verantwortung vor Gott und den Menschen", aus Weimar unter anderem gelernt haben.

%2005 Zahnärztetag: Zur Zukunft des Sozialstaates. 

%GG sieht da nichts viel vor, ist unspezifisch, lässt aber dem Gesetzgeber viel Freiraum. 
%Sie wiesen darauf hin das grundgesetzliche Sozialgarantien, etwa Recht auf Arbeit auch nicht wünschenswert wären, weil das, nur folgerichtig, auch einen entsprechende Eingriffs- und Zwangsrechte nicht unähnlich einer sozialistischen Wirtschaftsordnung erfordert. 
%So weit so gut.
%Aber was macht das BVerfG, wenn es beobachten kann, dass der Gesetzgeber keinen Regelungsspielraum mehr hat, etwa durch Steuerwettbewerb und europäische wie globale Regelungsarbitrage?
%Völlig richtig: Sozialstaat braucht Abgabenstaat. 
%Ist es dann nicht ein Problem, wenn Abgaben nicht mehr organisiert werden können?

%Meine Position: Demokratie krankt in der EU nicht hauptsächlich, oder jedenfalls nicht nur an zuviel Brüssel, sonder an bei weitem zu wenig, vor allem in der Sozial- und Steuerpolitik. 
%Nicht Subsidiarität ist das Problem, sondern Arbitrage. 
%Die zentralen Instrumente des sozialstaatlichen Sozialvertrages sind impotent: Steuer-, Struktur- und Sozialpolitik.
%Ich meine damit nicht: der Staat MÜSSE diese Aufgaben erfüllen, vielmehr steht es dem Staat völlig frei sie zu erfüllen -- oder eben nicht. 
%Aber ein Demokratiedefizit liegt vor, wenn er das nicht mehr kann.

%Zum Argument des Demokratiedefizits:
%ja, die europäischen Institutionen sind zu intergouvernmental, zu indirekte Demokratie. 
%Also: Stärkung des europäischen Parlaments, Stärkung des EUGh
%übrigens: das heischen gegen Brüsseler Bürokratie zeugt häufig von sachlicher Unkenntnis. 
%Der Teufel steckt im Detail: es GIBT einen Grund für die europäische Bananenkrümmung und Gurkenkrümmung.
%aber, das tiefere Argument: es gibt kein "europäisches Staatsvolk", "europäische Medienöffentlichkeit" und "europäische Parteienlandschaft" das ist Quatsch, das gibt es National auch nicht, und hat es nie gegeben. 
%Ist historisch nie so entstanden, kann auch konzeptionell so nicht entstehen. 
%Identität gibt es nur zwischen Freunden.
%Solidarität gibt, und MUSS es auf europäischer Ebene geben: wir wollen mehr wirtschaftliche Integration, das macht uns reich. 
%Wirtschaftlicher Integration MUSS aber immer auch mehr politische, besonders sozialstaatliche Integration folgen, sonst gibt es eine schieflage. 
%Deshalb, im Widerspruch zu Ihnen: DOCH, es MUSS um einen permanenten Ausbau der Union gehen, nicht in allem, aber in ziemlich vielem.
%Natürlich kann es in Vielfalt geeint gehen, aber eben nur bei manchen Sachen (Folklore); nicht bei anderen.
%Verweis auf Jonas Text, nochmal lesen. 
%Wenn wirtschaftliche Integration und sozialstaatliche Regelung auseinanderfallen gibt es Ungerechtigkeit, Arbeitslosigkeit, Ungleichgewichte, die explosiv sind.
%Das müssen wir den Unionbürgern erklären.
%Nein, es geht nicht um Schulden in erster Linie: Überschuldung ist auch das Ergebnis von der Unmöglichkeit anständige Steuern zu erheben. 
%Ein demokratisches Gemeinwesen -- wie Papier schreibt, besonders deines auf dem Boden des GG -- ist immer in der Lage sich beliebige Mischungen aus Staat und Markt zu geben. 
%Warum vergessen Sie das, wo sie doch sonst immer auf die Verbindung zwischen Sozialstaat und Abgabenstaat hinweisen?
%Wenn etwas von Populisten ausgeschlachtet wird, dann muss man sich gegen die Populisten stellen.
%Aber es ist auch wichtig zu erkennen: kein politisches System kann auf die Dauer große Arbeitslosigkeit oder andere materielle Not überdauern.


\subsection[Finality]{Finality} %formerly known as: finality of progress

\begin{quote}
	\emph{``Europe is not a place, but an idea.''} \\ %find better quote.
	--- Bernhard-Henry L\'{e}vy
\end{quote}

It seems a little ungrateful to criticize European integration these days, where supposedly, it is making such big, historic strides to a closer union, including such carrots as the \gls{ESM} (a permanent, conditional bail-out fund) and sticks as the \gls{EFC} (a beefed-up \gls{SGP}).
But these, alas, make not yet a more perfect union, but might even further tilt the glass. 

The \gls{ESM} may, as bail-outs do, kill some of the pain and tacitly socialize some of the costs of macroeconomic imbalances, but will, in itself, do little to prevent those imbalances from arising in the first place. 
If it does not kill the messenger (of possibly imperfect financial markets), it at least drowns out its message (of possibly unsustainable public finances or banks) by placing massive, reassuring bets on whichever economy is in trouble. 
This is, first of all, a fiscal transfer through the back door, without democratic legitimacy but effectively  shrouded in technical detail. 
It is, secondly, also not getting at the root cause of sovereign debt or bank crises in Europe, but merely providing a band-aid.

The \gls{EFC} and other means for budgetary discipline, too, do not make for a fiscal union, as advertised. 
The \gls{EFC}, in a further bout of negative integration, only limits the \emph{spending} of \gls{MS}, but does not harmonize the (competitively lowered) taxes in the union. 
Going into unsustainable sovereign debt is always a siren call for democratic polities, and it is a scourge on future generations that must be curbed. 
But in this apocryphal version of Odysseus travel, Odysseus is not only tied to the mast of fiscal discipline --- as he should be, to steer clear of the sirens --- but he is also forbidden from raising the sail of taxation. 
We should not be surprised that Odysseus might somehow free himself of the ties, or find other ways to heed the sirens' call if we prevent him from sailing out of their waters.

It also seems a little bit unfair to blame it all on Europe. 
\emph{Negative} is the current mode of worldwide economic integration, not just in Europe. 
The \gls{EU} and its \gls{MS} play the same \gls{PD} games not just on this continent, but on higher levels with other countries. 
The contradictions of a liberalized world economy without a world government are not merely european problems.

But Europe must and always has been more than just the continent, the free trade or the political union, but instead the \emph{idea} of an alternative, better path to postindustrial modernity. 
Lead by enlightenment values and rationality, we must find this path by, yet again, embracing the revolutionary values of freedom, equality and fraternity, that almost 250 years inspired Friedrich Schiller to celebrate the brotherhood and unity of all mankind, that, to this day, lives on in the symphonic poem of Ludwig van Beethoven as the anthem of the European Union.

\paragraph{Ode to the Mixed Economy.} And there is, indeed, reason to rejoice! If we rebuild an intact mixed economy with harmonized taxation and fiscal transfers, we can get it all: 
a European Union that does not just grow wide, but wide and deep, not just ``united in diversity'', but, really, ``integrated in solidarity''. 
We must insist, and move on with the only attractive model of human civilization that modernity has ever bred, amongst the catastrophical brethren's of fascism, socialism and (more benignly) neoliberalism: 
the liberal social democracy of an efficient, equitable welfare state.

If anywhere, it is in the singular institutional achievement of the mixed economy, that ``your [its] magics bind again / what custom's [modernity's] sword as strictly parted / all men become brothers / under the sway of thy [its] gentle wings'':
\begin{verse}
	\emph{Deine Zauber binden wieder,}\\
	\emph{Was die Mode streng geteilt,}\\
	\emph{Alle Menschen werden Br\"{u}der,}\\
	\emph{Wo Dein sanfter Fl\"{u}gel weilt.}\\
	--- Ode an die Freude, Friedrich Schiller (1785)\\
	--- 9th Symphonie, Ludwig van Beethoven (1824)
\end{verse}

\footnote{
	Advises Schiller the naysayers: 
	``And whoever never could achieve this, / Let him steal away crying from this gathering!'':
	\begin{verse}
		\emph{Und wer's nie gekonnt, der stehle}\\
		\emph{Weinend sich aus diesem Bund.}\\
	\end{verse}\\
	And to the skeptics: 
	``Pleasure was given to the worm'':
	\begin{verse}
		\emph{Wollust ward dem Wurm gegeben}\\
		--- Ode an die Freude, Friedrich Schiller (1785) 
	\end{verse}}

\pagebreak

\appendix
\chapter{\appendixname}

%!backmatter
\glsaddall
\renewcommand*{\glspostdescription}{}
\printglossaries

\pagebreak

\bibliography{./tex/library} % Bibliography database file, moga.bib 
	\bibliographystyle{apsr} % Bibliography style file, unsrt.bst

\end{document}
