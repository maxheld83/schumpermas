Pace, S. (2004). A grounded theory of the flow experiences of Web users. International Journal of Human-Computer Studies, 60(3), 327-363. doi: 10.1016/j.ijhcs.2003.08.005.

329: Flow is a state of consciousness that is sometimes experienced bypeo ple who are
deeply involved in an enjoyable activity. The experience is characterized by some
common elements: a balance between the challenges of an activityand the skills
required to meet those challenges; clear goals and feedback; concentration on the
task at hand; a sense of control; a merging of action and awareness; a loss of selfconsciousness;
a distorted sense of time; and the autotelic experience




Expose 1: Scale-Free Education Outcomes
What makes for guassian distributions?
-headwinds
- no network FX
- independent raws

Also: we're an egregious species. We like hanging out together, doing things togehter. Power Laws, then, are alo good. We want to do things in a network.


Expose 2: Brain Drain in New Accession Countries

or maybe: write about gaussian in education or mandelbrotian! Zess!

school holiday length in OECD coutnries compare between gender, class, how much do people unlearn.

or: romania, accesson countries, eating the children.


The distinction b/w continental and liberal isn't clear at all: the former may have more networks, and the liberal may be more excessively unequal.
Think spatial segretation not so much in Germany.

At the end: conclusions about how the welfare state should be.
Tja also wenn, dann sollte er sich mal des European social survey
anschauen -  aber ich denke nicht dass da was zu interethniscen
Kontakten drin ist.
Ah ja dann gibt es natürlich den TIES Datensatz, der sicher solche
Variablen enthält, nur wahrscheinlich ist der bisher nicht öffentlich.

Gruß,


Merlin


Think about the language of "needs". I think this is a wrong language, it's minimal in its idea of the welfare state, and that is plain wrong.



Steward & Langer 2007
I am not sure I agree that horizontal and vertical inequality is, in fact, different things. Maybe vertical inequality is the same, it's just that horizontal inequality is tightly grouped (think network theory)
Nice definition: inequality between groups is then the consequence of inequality in asset ownership between grups and inequalities in the returns to these assets. Assets include land, financial assets, education, public infrastructure and social capital (5)
Reasons why inequality persists:
1) cumulative forces: people who have money find it easier to get more money, to find jobs, borrow, invest et al.
2) the returns to different kinds of capital (financial, social, human) interact with the level of other capital. You can make more off a given level of financial capital given higehr social capita.
3) persistent assymetries in social capital, which cause unequal returns on other types of capital. Poor people tend to know more poor people. (also known as: the network effect)
4) discontinuities in reutnrs to diferent types of capital. Low levels of capital trap you righ thre. This could be due to the type of interaction discussed on the above, OR because of indivisibilities in that capital.
5) overt or implicit discrimination
keep in mind as they write: (9) present inequality could be interpreted as a result of past discrimination
it need not be, as Tilly and brown seem to think, hoarding. Just a network effect will do.
again, the counterfactual of inequality is wrong. it's not: absence of horizontal inequality between groups. that's bad. the correct one is: everyone according to his ability. Blog this.
write this: it's not about marxism or relative deprivation. It's just a condition of postindustrial production, that's all.
contrast my idea of reinforcing inequality to ideas of comparative advantage.
the very idea of neo-classical economics, that ienqualities would balance out because of diminishing returns to capital, appears to be wrong (29). and here is why: networks. 

Cozzi & Privileggi 2009
Basic assumption: growth and polarization are correlated
they argue that what leads to the gross inequalities is in fact, paradoxically, social mobility (and NOT Marxs classes!)
this is the fractal nature they mean: over time.
 show the self-similarity on page 19. this is great for an intuition

Keller 2005
"the power-law connectivity distribution seen in scale-free networks seems to emerge as one of the very few universal mathematical laws of life" )wolff et al as cited in Keller 2005: 1061




Off-Topic
Note how knowledge gap hypothesis (Severin) is really similar to what I want to do in the educaiton piece.

