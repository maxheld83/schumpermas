\documentclass{article}\usepackage[]{graphicx}\usepackage[]{color}
%% maxwidth is the original width if it is less than linewidth
%% otherwise use linewidth (to make sure the graphics do not exceed the margin)
\makeatletter
\def\maxwidth{ %
  \ifdim\Gin@nat@width>\linewidth
    \linewidth
  \else
    \Gin@nat@width
  \fi
}
\makeatother

\definecolor{fgcolor}{rgb}{0.345, 0.345, 0.345}
\newcommand{\hlnum}[1]{\textcolor[rgb]{0.686,0.059,0.569}{#1}}%
\newcommand{\hlstr}[1]{\textcolor[rgb]{0.192,0.494,0.8}{#1}}%
\newcommand{\hlcom}[1]{\textcolor[rgb]{0.678,0.584,0.686}{\textit{#1}}}%
\newcommand{\hlopt}[1]{\textcolor[rgb]{0,0,0}{#1}}%
\newcommand{\hlstd}[1]{\textcolor[rgb]{0.345,0.345,0.345}{#1}}%
\newcommand{\hlkwa}[1]{\textcolor[rgb]{0.161,0.373,0.58}{\textbf{#1}}}%
\newcommand{\hlkwb}[1]{\textcolor[rgb]{0.69,0.353,0.396}{#1}}%
\newcommand{\hlkwc}[1]{\textcolor[rgb]{0.333,0.667,0.333}{#1}}%
\newcommand{\hlkwd}[1]{\textcolor[rgb]{0.737,0.353,0.396}{\textbf{#1}}}%

\usepackage{framed}
\makeatletter
\newenvironment{kframe}{%
 \def\at@end@of@kframe{}%
 \ifinner\ifhmode%
  \def\at@end@of@kframe{\end{minipage}}%
  \begin{minipage}{\columnwidth}%
 \fi\fi%
 \def\FrameCommand##1{\hskip\@totalleftmargin \hskip-\fboxsep
 \colorbox{shadecolor}{##1}\hskip-\fboxsep
     % There is no \\@totalrightmargin, so:
     \hskip-\linewidth \hskip-\@totalleftmargin \hskip\columnwidth}%
 \MakeFramed {\advance\hsize-\width
   \@totalleftmargin\z@ \linewidth\hsize
   \@setminipage}}%
 {\par\unskip\endMakeFramed%
 \at@end@of@kframe}
\makeatother

\definecolor{shadecolor}{rgb}{.97, .97, .97}
\definecolor{messagecolor}{rgb}{0, 0, 0}
\definecolor{warningcolor}{rgb}{1, 0, 1}
\definecolor{errorcolor}{rgb}{1, 0, 0}
\newenvironment{knitrout}{}{} % an empty environment to be redefined in TeX

\usepackage{alltt}
\IfFileExists{upquote.sty}{\usepackage{upquote}}{}
\begin{document}

% notice: there may be stuff in the data that suggests how many factors make sense (loc 2287)
% notice: I am definetely doing exploratory factor analysis; no ex-ante reason to expect one or the other result
% make a correlation matrix, just for descriptives, look at which items are most highly correlated
% same thing for q sorts
% check kline 1994 and Harman 1976 in the above on PCA vs CFA
% Harman 1976 says they will mostly be the same!
% what we're loosing when then R2 increases is, by definition, some residuals.
	% interesting idea: maybe look at what kind of residuals was lost? could we tell a story about that?

%consider jonathan haidt, joshua greene on deep pragmatism and slow/fast morality

  %via qlist Since Mary Furnari decided to rely on varimax rotation, Watts & Stenner (2012), for instance, would actually recommend PCA. (But note that Watts & Stenner erroneously think that varimax maximizes the amount of explained variance. To clarify: Varimax searches a simple structure solution characterized by a maximal number of either high or near zero loadings for every factor which is arrived at by maximizing the variance of the factors' loadings. The amount of explained variance is not affected by rotation.)
  %In my view, the meaningfulness of certain quantitative coefficients in Q, like so-called 'significance' of factor loadings is often overrated. So as if observing coefficient based rules could provide mathematical-statistical  proof for the soundness of the researcher's decisions and conclusions. Without referring to the wisdom of inferential statistics (which is not applicable in Q, IMO), however, I would dare to bet that given 60% explained variance of the 1st factor and 4% and less for the following, that Mary Furnari won't be able to assemble groups of sorts (to be flagged on different factors) that represent distinct = uncorrelated views. But that's just a bet which I possibly can lose. So nothing is lost by just trying out varimax solutions with 2, 3, 4 ... factors. Two simple (but not 100% unambiguous) rules for accepting a factor solution: (1) At least 2, better 3, defining sorts (load strongly on the respective factor only). (2) Intercorrelations of factor scores at a moderate level (possible choice of critical level: not higher than the size of the loading accepted for a defining sort).

%factor extraction
% just use factanal normal factor
% or normal princcop as opposed to psych
http://www.statmethods.net/advstats/factor.html
http://cran.r-project.org/web/packages/nFactors/index.html
http://cran.r-project.org/web/packages/FactoMineR/index.html
psych also appears to do parallel analysis by fa.parallel

\begin{knitrout}
\definecolor{shadecolor}{rgb}{0.969, 0.969, 0.969}\color{fgcolor}\begin{kframe}
\begin{alltt}
\hlkwd{setwd}\hlstd{(}\hlstr{"/Users/Max/Github/schumpermas"}\hlstd{)}
\hlkwd{install.packages}\hlstd{(}\hlkwd{c}\hlstd{(}\hlstr{"qmethod"}\hlstd{),} \hlkwc{repos} \hlstd{=} \hlkwa{NULL}\hlstd{,} \hlkwc{type}\hlstd{=}\hlstr{"source"}\hlstd{)}
\hlkwd{library}\hlstd{(qmethod)}
\end{alltt}


{\ttfamily\noindent\itshape\color{messagecolor}{\#\# Loading required package: ggplot2\\\#\# Loading required package: psych\\\#\# \\\#\# Attaching package: 'psych'\\\#\# \\\#\# The following object is masked from 'package:ggplot2':\\\#\# \\\#\#\ \ \ \  \%+\%}}

{\ttfamily\noindent\color{warningcolor}{\#\# Warning: replacing previous import by 'ggplot2::\%+\%' when loading 'qmethod'}}\begin{alltt}
\hlkwd{library}\hlstd{(ggplot2)}
\hlkwd{library}\hlstd{(stringr)}
\end{alltt}
\end{kframe}
\end{knitrout}

\begin{knitrout}
\definecolor{shadecolor}{rgb}{0.969, 0.969, 0.969}\color{fgcolor}\begin{kframe}
\begin{alltt}
\hlkwd{source}\hlstd{(}\hlstr{"r/imports.r"}\hlstd{)}
\end{alltt}


{\ttfamily\noindent\color{warningcolor}{\#\# Warning in file(filename, "{}r"{}, encoding = encoding): cannot open file 'r/imports.r': No such file or directory}}

{\ttfamily\noindent\bfseries\color{errorcolor}{\#\# Error in file(filename, "{}r"{}, encoding = encoding): cannot open the connection}}\end{kframe}
\end{knitrout}

\section{Descriptives}

\begin{knitrout}
\definecolor{shadecolor}{rgb}{0.969, 0.969, 0.969}\color{fgcolor}\begin{kframe}
\begin{alltt}
\hlstd{item.mean} \hlkwb{<-} \hlkwd{apply}\hlstd{(q.sorts,} \hlkwd{c}\hlstd{(}\hlnum{1}\hlstd{,}\hlnum{3}\hlstd{), mean,} \hlkwc{na.rm} \hlstd{=} \hlnum{TRUE}\hlstd{)}
\end{alltt}


{\ttfamily\noindent\bfseries\color{errorcolor}{\#\# Error in apply(q.sorts, c(1, 3), mean, na.rm = TRUE): object 'q.sorts' not found}}\begin{alltt}
\hlstd{item.mean} \hlkwb{<-} \hlkwd{cbind}\hlstd{(item.mean, item.mean[,}\hlstr{"after"}\hlstd{]} \hlopt{-} \hlstd{item.mean[,}\hlstr{"before"}\hlstd{])}
\end{alltt}


{\ttfamily\noindent\bfseries\color{errorcolor}{\#\# Error in cbind(item.mean, item.mean[, "{}after"{}] - item.mean[, "{}before"{}]): object 'item.mean' not found}}\begin{alltt}
\hlkwd{colnames}\hlstd{(item.mean)[}\hlnum{3}\hlstd{]} \hlkwb{<-} \hlstr{"change"}
\end{alltt}


{\ttfamily\noindent\bfseries\color{errorcolor}{\#\# Error in colnames(item.mean)[3] <- "{}change"{}: object 'item.mean' not found}}\begin{alltt}
\hlstd{item.mean}
\end{alltt}


{\ttfamily\noindent\bfseries\color{errorcolor}{\#\# Error in eval(expr, envir, enclos): object 'item.mean' not found}}\begin{alltt}
\hlstd{item.mean[}\hlkwd{order}\hlstd{(item.mean[,}\hlstr{"change"}\hlstd{]),]} \hlcom{#  sort by change}
\end{alltt}


{\ttfamily\noindent\bfseries\color{errorcolor}{\#\# Error in eval(expr, envir, enclos): object 'item.mean' not found}}\begin{alltt}
\hlstd{item.sd} \hlkwb{<-} \hlkwd{apply}\hlstd{(q.sorts,} \hlkwd{c}\hlstd{(}\hlnum{1}\hlstd{,}\hlnum{3}\hlstd{), sd,} \hlkwc{na.rm} \hlstd{=} \hlnum{TRUE}\hlstd{)}
\end{alltt}


{\ttfamily\noindent\bfseries\color{errorcolor}{\#\# Error in apply(q.sorts, c(1, 3), sd, na.rm = TRUE): object 'q.sorts' not found}}\begin{alltt}
\hlstd{item.sd} \hlkwb{<-} \hlkwd{cbind}\hlstd{(item.sd, item.sd[,}\hlstr{"after"}\hlstd{]} \hlopt{-} \hlstd{item.sd[,}\hlstr{"before"}\hlstd{])}
\end{alltt}


{\ttfamily\noindent\bfseries\color{errorcolor}{\#\# Error in cbind(item.sd, item.sd[, "{}after"{}] - item.sd[, "{}before"{}]): object 'item.sd' not found}}\begin{alltt}
\hlkwd{colnames}\hlstd{(item.sd)[}\hlnum{3}\hlstd{]} \hlkwb{<-} \hlstr{"change"}
\end{alltt}


{\ttfamily\noindent\bfseries\color{errorcolor}{\#\# Error in colnames(item.sd)[3] <- "{}change"{}: object 'item.sd' not found}}\begin{alltt}
\hlstd{item.sd[}\hlkwd{order}\hlstd{(item.sd[,}\hlstr{"change"}\hlstd{]),]} \hlcom{#  sort}
\end{alltt}


{\ttfamily\noindent\bfseries\color{errorcolor}{\#\# Error in eval(expr, envir, enclos): object 'item.sd' not found}}\begin{alltt}
\hlstd{item.sd}
\end{alltt}


{\ttfamily\noindent\bfseries\color{errorcolor}{\#\# Error in eval(expr, envir, enclos): object 'item.sd' not found}}\begin{alltt}
\hlstd{person.cor} \hlkwb{<-} \hlkwd{cor}\hlstd{(q.sorts[,,}\hlstr{"before"}\hlstd{])}
\end{alltt}


{\ttfamily\noindent\bfseries\color{errorcolor}{\#\# Error in is.data.frame(x): object 'q.sorts' not found}}\begin{alltt}
\hlstd{person.cor[}\hlkwd{which.min}\hlstd{(}\hlkwd{abs}\hlstd{(person.cor))]}
\end{alltt}


{\ttfamily\noindent\bfseries\color{errorcolor}{\#\# Error in eval(expr, envir, enclos): object 'person.cor' not found}}\begin{alltt}
\hlstd{person.cor}
\end{alltt}


{\ttfamily\noindent\bfseries\color{errorcolor}{\#\# Error in eval(expr, envir, enclos): object 'person.cor' not found}}\begin{alltt}
\hlkwd{str}\hlstd{(person.cor)}
\end{alltt}


{\ttfamily\noindent\bfseries\color{errorcolor}{\#\# Error in str(person.cor): object 'person.cor' not found}}\begin{alltt}
\hlkwd{help}\hlstd{(which.min)}
\end{alltt}
\end{kframe}
\end{knitrout}

\subsection{Factor Extraction}

\cite[6]{Exel2005} recommends 4--5 people per viewpoint, which given 17 participants yields 3--4 factors, or viewpoints.

\subsection{Which Data Reduction Technique?}

Factor analysis (which?)
or
PCA (which?)
PCA (which?)

\cite{Wittenborn} says explicitly what the problem is: 132, namely that there is no test as to whether there are other people who would sort like this.

agree that in fact, as Kampen and Tamas point out, the representativeness of the q sample to the universe of statements is a weak, weak spot.

Important: as cuppen in  kampen and tamas writes 3112 it's not the number of supporters that count, just the perspectives.

Consider the discussion of q methodology and validities oin kampen tamas 3112

do cluster analysis instead of factor; according to kampen and tamas. This might be interesting.


\begin{knitrout}
\definecolor{shadecolor}{rgb}{0.969, 0.969, 0.969}\color{fgcolor}\begin{kframe}
\begin{alltt}
\hlkwd{library}\hlstd{(paran)}
\end{alltt}


{\ttfamily\noindent\itshape\color{messagecolor}{\#\# Loading required package: MASS}}\begin{alltt}
\hlstd{hornvectors} \hlkwb{<-} \hlkwd{paran}\hlstd{(}\hlkwd{t}\hlstd{(q.sorts[,,}\hlstr{"after"}\hlstd{]),}
                        \hlkwc{iterations} \hlstd{=} \hlnum{1000}\hlstd{,}
                        \hlkwc{centile} \hlstd{=} \hlnum{95}\hlstd{,}
                        \hlkwc{quietly} \hlstd{=} \hlnum{FALSE}\hlstd{,}
                        \hlkwc{status} \hlstd{=} \hlnum{TRUE}\hlstd{,}
                        \hlkwc{graph} \hlstd{=} \hlnum{TRUE}\hlstd{,}
                        \hlkwc{color} \hlstd{=} \hlnum{TRUE}\hlstd{)}
\end{alltt}


{\ttfamily\noindent\bfseries\color{errorcolor}{\#\# Error in t(q.sorts[, , "{}after"{}]): object 'q.sorts' not found}}\end{kframe}
\end{knitrout}

%read kline 1994 on wattss/stenner on PCA vs CFA, and they will be similar as per harman1976
% check whether the two flagging criteria make sense, aren't they too restrictive?

\section{Analysis}

\begin{knitrout}
\definecolor{shadecolor}{rgb}{0.969, 0.969, 0.969}\color{fgcolor}\begin{kframe}
\begin{alltt}
\hlstd{keyneson} \hlkwb{<-} \hlkwd{list}\hlstd{(}\hlstr{"before"}\hlstd{=}\hlkwd{c}\hlstd{(),} \hlstr{"after"}\hlstd{=}\hlkwd{c}\hlstd{())}
\hlstd{keyneson}\hlopt{$}\hlstd{before} \hlkwb{<-} \hlkwd{qmethod}\hlstd{(}
  \hlkwc{dataset} \hlstd{= q.sorts[,,}\hlstr{"before"}\hlstd{],}
  \hlkwc{nfactors} \hlstd{=} \hlnum{3}\hlstd{,}
  \hlkwc{rotation} \hlstd{=} \hlstr{"varimax"}\hlstd{,}
  \hlkwc{forced} \hlstd{=} \hlnum{TRUE}
\hlstd{)}
\end{alltt}


{\ttfamily\noindent\bfseries\color{errorcolor}{\#\# Error in nrow(dataset): object 'q.sorts' not found}}\begin{alltt}
\hlstd{keyneson}\hlopt{$}\hlstd{after} \hlkwb{<-} \hlkwd{qmethod}\hlstd{(}
  \hlkwc{dataset} \hlstd{= q.sorts[,,}\hlstr{"after"}\hlstd{],}
  \hlkwc{nfactors} \hlstd{=} \hlnum{3}\hlstd{,}
  \hlkwc{rotation} \hlstd{=} \hlstr{"varimax"}\hlstd{,}
  \hlkwc{forced} \hlstd{=} \hlnum{TRUE}
\hlstd{)}
\end{alltt}


{\ttfamily\noindent\bfseries\color{errorcolor}{\#\# Error in nrow(dataset): object 'q.sorts' not found}}\end{kframe}
\end{knitrout}

\begin{knitrout}
\definecolor{shadecolor}{rgb}{0.969, 0.969, 0.969}\color{fgcolor}\begin{kframe}
\begin{alltt}
\hlstd{arrayviz} \hlkwb{<-} \hlkwd{array.viz}\hlstd{(}
  \hlkwc{QmethodRes} \hlstd{= keyneson}\hlopt{$}\hlstd{before}
  \hlstd{,}\hlkwc{f.names} \hlstd{=} \hlkwd{c}\hlstd{(}\hlstr{"resentment"}\hlstd{,}\hlstr{"critical"}\hlstd{,}\hlstr{"moderate"}\hlstd{)}
  \hlstd{,}\hlkwc{incl.qdc} \hlstd{=} \hlnum{TRUE}
  \hlstd{,}\hlkwc{color.scheme} \hlstd{=} \hlstr{"Set1"}
  \hlstd{,}\hlkwc{extreme.labels} \hlstd{=} \hlkwd{c}\hlstd{(}\hlstr{"very much disagree"}\hlstd{,}\hlstr{"very much agree"}\hlstd{)}
\hlstd{)}
\end{alltt}


{\ttfamily\noindent\bfseries\color{errorcolor}{\#\# Error in if (length(f.names) != QmethodRes\$brief\$nfactors) \{: argument is of length zero}}\begin{alltt}
\hlstd{arrayviz[}\hlnum{3}\hlstd{]}
\end{alltt}


{\ttfamily\noindent\bfseries\color{errorcolor}{\#\# Error in eval(expr, envir, enclos): object 'arrayviz' not found}}\end{kframe}
\end{knitrout}

%stephenson on number of factors, via verena
% OS-9-3-Stephenson.pdf S. 89 “Second, a little simple factor analysis is
% all that the operations demand: It will be the end
% of work in this domain if anyone thinks that its be- all and end-all is factor analysis. The less of it, the better. Three or four factors are all that most well planned studies require; there's something loose in the works if anything like ten or so factors are carved out for interpretation. The key to sound work, i.e., to make discoveries, is what one puts into Q method as abduction, not what factor analysis turns out deductively.”

% use multiplot for several
% # add title
% # get rid of y axis
% # change tick marks
% # shorten axes
% # worry about bipolar factors
% # notice that I look at sd of pop, not sample, because it's not r stats
% # but still, dispersion is interesting - those are the loose lego blocks
% # is there maybe a need to also look at the loose lego blocks *overall*? And what are those? What are the loosest blocks?


%Frank destroys the SD of pro-socialism

\begin{knitrout}
\definecolor{shadecolor}{rgb}{0.969, 0.969, 0.969}\color{fgcolor}\begin{kframe}
\begin{alltt}
\hlstd{q.feedback[}\hlstr{"pro-socialism"}\hlstd{,}\hlstr{"Frank"}\hlstd{,}\hlstr{"after"}\hlstd{]}
\end{alltt}


{\ttfamily\noindent\bfseries\color{errorcolor}{\#\# Error in eval(expr, envir, enclos): object 'q.feedback' not found}}\end{kframe}
\end{knitrout}

\begin{knitrout}
\definecolor{shadecolor}{rgb}{0.969, 0.969, 0.969}\color{fgcolor}\begin{kframe}
\begin{alltt}
\hlstd{array.viz.data} \hlkwb{<-} \hlkwd{merge}\hlstd{(}  \hlcom{# add type of item}
  \hlkwc{x} \hlstd{= array.viz.data}
  \hlstd{,}\hlkwc{y} \hlstd{= q.sampling.structure}  \hlcom{# that is where the types are from}
  \hlstd{,}\hlkwc{by.x} \hlstd{=} \hlnum{0}  \hlcom{# these are rownames}
  \hlstd{,}\hlkwc{by.y} \hlstd{=} \hlstr{"handle"}  \hlcom{# that is how they are called}
  \hlstd{,}\hlkwc{all} \hlstd{=} \hlnum{TRUE}
\hlstd{)}
\end{alltt}


{\ttfamily\noindent\bfseries\color{errorcolor}{\#\# Error in merge(x = array.viz.data, y = q.sampling.structure, by.x = 0, : object 'array.viz.data' not found}}\begin{alltt}
\hlkwd{rownames}\hlstd{(array.viz.data)} \hlkwb{<-} \hlstd{array.viz.data}\hlopt{$}\hlstd{Row.names}  \hlcom{# restore rownames}
\end{alltt}


{\ttfamily\noindent\bfseries\color{errorcolor}{\#\# Error in eval(expr, envir, enclos): object 'array.viz.data' not found}}\begin{alltt}
\hlstd{g} \hlkwb{<-} \hlstd{g} \hlopt{+} \hlkwd{geom_text}\hlstd{(}
  \hlkwd{aes}\hlstd{(}
    \hlstd{,}\hlkwc{fontface}\hlstd{=}\hlkwd{c}\hlstd{(}\hlstr{"plain"}\hlstd{,}\hlstr{"bold"}\hlstd{,}\hlstr{"italic"}\hlstd{)[metaconsensus]}
  \hlstd{)}
  \hlstd{,}\hlkwc{size} \hlstd{=} \hlnum{3.5}
\hlstd{)}
\end{alltt}


{\ttfamily\noindent\bfseries\color{errorcolor}{\#\# Error in eval(expr, envir, enclos): object 'g' not found}}\begin{alltt}
\hlcom{#g <- g + scale_family_manual(c("serif","sans","mono"))}
\end{alltt}
\end{kframe}
\end{knitrout}
\end{document}
