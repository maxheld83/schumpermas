%!TEX root=../tax-democracy-held.tex

\section*{Abstract}

As capitalism matures, commerce globalizes and inequalities widen, the mixed economies of the \gls{OECD} face steeper tradeoffs between efficiency, equity and sustainability in taxation. 
Suboptimal taxation of (some) incomes and postpaid consumption undermines state-market balances, retrenches welfare and, ultimately, constrains democratic rule of the economy.
Progressive taxation of consumption defined as income minus net savings (\gls{PCT}) or rent capture on natural resources and unimproved land value (\gls{LVT}) appear to offer higher tradeoffs, but remain hypothetical and are hardly discussed in science or the public.
Similarly, as complexity increases, special interests concentrate and traditional allegiances wane, the liberal democracies of the \gls{OECD} face steeper tradeoffs between mass participation, political equality and enlightened understanding.
Representative, pluralist democracy suffers from resultant malaggregations and ill-formed preferences, and, ultimately, impairs political equality in the face of economic inequality. %better wording
Collective decisions by ordinary citizens, based on mutual respect, balanced information and reason-giving promise to better uphold liberal democracy, but such deliberation has hardly been tried on abstract and complex choices.

I here propose to investigate whether, and how, people can deliberate an issue as complex and pervasive as tax, and, conversely, whether, and how people change their beliefs and preferences about taxation as a result of such deliberation.
To test this, I invite a diverse sample of around 80 Bremen-area citizens to participate in a daylong \gls{DP} in which they discuss basic choices in tax base and schedule in moderated small-group and plenary sessions.
Participants will receive a balanced briefing book and have access to a diverse panel of experts.
I will quantitatively compare participants' responses to closed-ended questionnaires handed out before and after the event.
In addition, I will qualitatively compare selectively transcribed recordings of participants' deliberations with arguments and concepts broadly consensual among economists. 
I may further subject some groups to short learning interventions, targeted at improving participant understanding of select concepts and abstractions deemed relevant for an informed choice of base and schedule.

If ordinary citizens can be shown to competently deliberate an issue as demanding as taxation, deliberative democrats will have more reason to be confident in their procedures and standards.
If ordinary citizens can be shown to prefer different tax bases and schedules as a function of greater knowledge and deliberation, welfare state research and political economy will have to explain the absence of such a morally relevant hypothetical.

%abouts
	%200-500 words long
	%max 2 paragraphs
	%4 Ws, via Art Woodward
		% W1 Why was the research done?
		% W2 What was done?
		% W3 What was discovered?
		% W4 Why is the discovery important?
	%Checklist
		%is the topic clear from reading the title?
		% is there a bridge statement that a) shows the research has not been done before and b) links specific previous research to the purposes of your study?
		%is all terminology defined for the general non-specialist reader?
		% is the definition of the sample clear? 
		% are independent and dependent variable clearly defined?
		% are the methods explained so that the reader knows what was done? (which things were manipulated experimentally, which things were controlled statistically ...)
		% are the discoveries linked clearly to the goals stated earlier?
		% was the prove it or remove it rule applied?
	%Goals
		% big picture problem or topic widely debated in your field
		% gap in the literature on this topic
		% your project filling the gap
		% your original argument
		% a strong concluding sentence

%This is context (says art):
%I wonder: why, in the richest of countries, in the most enlightened of times (current day \gls{OECD}-world), are our welfare states so constrained and inefficient, our democracies so confused and distorted? 
	%do I mean something specific/scientific by these adjectives?
	%you should cite some previous author, literature on here (you'll pick your reviewer!, also, you really need to give credit! says Art).
	%say what you're problem is with the mainstream literature. Don't say criticize.
	%So far, this question has been answered by looking at welfare state spending and revenue. %However, ...
	%However, this work would benefit from alternative configurations of welfare states and democracies, and for failing to explain their absence. %put something equally concrete as spending and revenue, examine the tax law itself.
	%explain welfare state retrenchment
	%include the whole progression thing earlier
	%stress that this is an experiment, which isn't so common otherwise (this would make it sexy) (include control groups)
	%maybe include the questionnaire, do pre/post, 4-5 days, 
	%absolutely stress the sense of validity
%I ask: if a hypothetical, superior political process (deliberative democracy) ruled our polity, would we think better and fairer about the institutions of the mixed economy and would we opt for a hypothetical, superior tax regime (\gls{PCT}, \gls{LVT} and \gls{NIT})? 

%And so I test: if given the possibility to deliberate well-informed, fairly and thoroughly (a \gls{DP}), will randomly selected, ordinary voters understand the mixed economy differently (better) and prefer a different tax (\gls{PCT}, \gls{LVT} and \gls{NIT})? %control group
		
	%We hope that people will understand and prefer [use same concepts as above, link] a more progressive and efficient tax
			
%If, in fact, they do, welfare state research will have a lot more to explain, and deliberative democracy will have shown its stripes.

%\begin{figure}[htbp]
%	\centering
%	\includegraphics[width=1\linewidth]{diss-expertise}  
%	\caption{Academic Fields and Advisors for this Dissertation}
%	\label{fig:diss-expertise}
%\end{figure}
%
%\begin{landscape}
%\begin{figure}[htbp]
%	\centering
%	\includegraphics[width=1\linewidth]{diss-mindmap}  
%	\caption{Mindmap of this Dissertation}
%	\label{fig:diss-mindmap}
%\end{figure} 
%\end{landscape}