%!TEX root=../tax-democracy-held.tex

%\chapter[Proposal]{Proposal} \label{chap:proposal-phd}
\footnote{
	The structure follows \citeauthor{Schmitter2002}'s suggestions for a good PhD proposal \citeyearpar{Schmitter2002}.
}

%should be 40k characters tops, or 12-15 pages, or 20 pages at BIGSSS

%what are we going to learn as the result of the proposed project that we do not know now?
%Why is worth knowing?
%how will we know that the conclusions are valid?

\begin{quote}
    \emph{``The spirit of a people, its cultural level, its social structure, the deeds its polity may prepare --- all this and more is written in its fiscal history, stripped of all phrases.
    He who knows how to listen to its message here discerns the thunder of world history more clearly than anywhere else.''}\\
    --- Joseph \citet[6]{Schumpeter}
    %add better source
\end{quote}

\begin{quote}
    \emph{``The unforced force of the better argument'':}\\
	%\cite[305]{Habermas1996}
    \emph{``The speaker must choose a comprehensible expression so that speaker and hearer can understand one another.''}\\
    %\cite[2f]{Habermas1976}
    \emph{``[A]nyone acting communicatively must, in performing any speech act, raise universal validity claims and suppose that they can be vindicated.''}\\
	%\cite[2]{Habermas1979}
    --- Jürgen Habermas (\citeyear[305]{Habermas1996}, \citeyear[2f]{Habermas1976}, \citeyear[2]{Habermas1979}, respectively)
\end{quote}

%must look at q methodology, noted in \cite{Niemeyer2007}, referencing brown 1980

\section{Introduction}
%write one short sentence, follow with several paragraphs to explain (schmitter)
%1. The idea: what do you want to study?
%2. The reason: why do you want to study it? 

If world history is still written in tax, as \citeauthor{Schumpeter} believed, for democratic rule to be legitimized by rational argument, as \citeauthor{Habermas-1984} demanded, the citizens of our democracy, too, must be well-versed in \emph{fiscalese}.
I ask how people think and speak differently about tax, if they participate in a democratic process \emph{closer} to Habermas' regulative ideal of deliberation.
I test, how --- if at all --- people change their beliefs and preferences on taxation, if they participate in a daylong deliberation on the topic. 

\subsection{Taxation and Welfare}

Taxation is a particularly important area of democratic rule because ``in the modern world, taxation \emph{is} the social contract'' \citep[1, emphasis in original]{Martin2009a}
\footnote{
	Even though social scientists have since paid little attention to it \citep[xiii]{Tilly2009}. %K191
}.
In the mixed economies of \gls{OECD}-style welfare regimes, governments must be able to draft some privately owned resources to finance public goods, to redistribute incomes and to otherwise interfere in market outcomes.
Taxation is the preferred institution to govern the state-market interface \citep{MusgThet1959,Stiglitz2011}.
If welfare states are to offer their sovereigns attractive tradeoffs between efficiency, equity and sustainability, taxes must be able to raise revenue and redistribute incomes without unduly altering competitive market exchanges.

Instead, taxation everywhere in the \gls{OECD} is in crisis.
As alternative sources of economic relief --- monetary expansion and sovereign debt --- are maxed out, structural misalignments persist, and previously forestalling (asset) bubbles have burst into their days of reckoning, public revenues appear to be strictly limited by the longtime coming contradictions of current tax regimes \citep{Streeck2013}. %not sure about this soure, must read it
The popular mixture of (progressive) income, (proportional) consumption and (regressive) payroll taxes appears to offer only harshly unattractive tradeoffs between equity and efficiency \citep{McCafferyHines2010}, as bases have shrunk and schedules flattened \citep{Ganghof2006}.
At the same time --- possibly partly as a result --- inequalities of incomes and wealth
\footnote{
	Data on the distribution of wealth is conspicuously hard to come by \citep[158]{Crouch2004} or ordinally summarized in deciles, rendering much of the inequality invisible.
}
have widened \citep{Butterwegge,Wagner2007,Grabka2007}.

If governments cannot raise the resources necessary to meet democratic demands without incurring prohibitive costs, the social contract is fraying \citep{Crouch2004} and, in so far as paralyzed taxation disturbs the state-market balance of the mixed economy, violates output congruence of a democracy \citep[compare][190]{Zurn-2000-aa}.%so so shource
Whichever way governments now turn, absent better tax, they will violate the post-war capitalist compact of stable, widely shared growth \citep{Pierson2002,StreeckMertens2010}. %so-so source

\subsection{Taxation and Deliberation}

Yet, even if experts could agree on a better tax --- such as a promising, post-paid, cash-flow based \gls{PCT} \citep{Seidman1997,Frank2005a,McCaffery2002,Graetz2009} --- the tradeoffs involved in choosing or calibrating a tax regime would remain irreducibly political.
It may, for example, not be possible to let experts decide where in the circular flow of the economy taxes should be levied; on factor incomes, household consumption, or household wealth --- let alone at which schedule.
\cite{McCaffery2005}, for instance, illustrates how the choice of base can be wrought with vexing complexity and deeply implicated by normative judgements on the return to capital as ``ordinary savings'' or genuine ``yields''.
Even to the extent that tax choice can be left to economic a-priori or empirical analyses, of, for example, \glspl{DWL}, such expertise relies on contentious views of human nature \citep[for a review,][]{Persky1995}, and observations of nature are, in any event, devoid of moral messages \citep[43]{Gould1982}.

Deliberative democracy is an attractive and fitting prescription to resolve or at least clarify such disagreements because it rests them with ordinary citizens, equips them with relevant information and stresses reasonable arguments.

What emerges today from citizens, their legislatures and public spheres, may however be more ``analytic muddle of tax'' \citep[862]{McCaffery2005} than ``communicative rationality'' \citep{Habermas-1984}.
The beliefs held and issues considered in the public \citep[for example,][]{Caplan2007}, appear to bear little resemblance to the abstractions suggested by economists \cite[for example,][]{Harberger1974}.
People seem to ignore fundamental choices in tax \citep[for example,][]{McCafferyHines2010}, and may fall for inconsequential alternatives \citep[for example,][]{SausgruberTyran2011}, possibly even err systematically against their own material interest --- if they are on the lower economic rungs \citep[for a German example,][]{Kemmerling2009}.
Here, too, (input) congruence of democratic rule may be violated if and to the extent that people are confused about policy choices \citep[compare][90]{Zurn-2000-aa}.

\subsection{Raw and Enlightened Thinking and Speech}

I here propose to investigate a supposed difference in current, and (more) ideal speech thinking about tax by subjecting a group of diverse citizens to a deliberative forum on the topic, which, in one succinct summary \cite[25ff]{Steenbergen2003}, is marked by 
\begin{enumerate}
	\item reasonably accurate and complete \emph{information},
	\item substantive \emph{balance} of possibly competing arguments and perspectives presented,
	\item \emph{diversity} of participants and their positions taken,
	\item a sincere weighting of arguments by participants, or \emph{conscientiousness}
	\item equal consideration of all arguments, regardless of which participant offered them \cite[127]{Fishkin2009}.
\end{enumerate} %K1799

Such a difference $\Delta$ between ``raw'' and ``enlightened'' thinking, %cite mccaffery 
beliefs \citep{Caplan2007} and preferences \citep{Fishkin2009} is worth knowing for several reasons:
\footnote{
	\cite{PrzeworskiSalomon1995} suggest to explicate what readers are going to learn as a result of the proposed project, and why that would be worth knowing.
}
\begin{enumerate}
	\item 
		There may be \emph{agreeable}, doable but hypothetical taxes, which may (Pareto)-improve over current taxes \citep{Harberger1974} or may be fairer \citep{Rawls-1971} and more sustainable \citep{Solow1956}, including a \glsfirst{PCT} \citep{Mill1848,McCaffery2002,Frank2005a,Seidman1997}, a \glsfirst{LVT} \citep{George1879,Buiter1988} and a \glsfirst{NIT} \citep{Friedman1962}.
		By comparison, existing taxes
		\footnote{
			In the \gls{OECD}-world, those are varying combinations of pre-paid, proportional consumption taxes such as \gls{VAT} and income taxes, often with different schedules for labor, capital and corporate incomes. 
			Wealth taxes, often only on some asset classes, exist in some jurisdictions, but --- with the exception of local US property taxes --- are not a large revenue source.
		}
		appear unnecessarily wasteful, inequitable and unsustainable.
		These suboptimal tax regimes may present democratic sovereigns with needlessly harsh tradeoffs between efficiency, equity, sustainability and other competing ends.
		The means of an intact mixed economy \citep{MusgThet1959,Stiglitz2011} may consequently erode or even force retrenched welfare states into ``permanent austerity'' \citep{Pierson2002,StreeckMertens2010} as it becomes harder for them to reconcile planned allocation and market exchange through taxation.
		
		Public confusion about taxation may be a contributing factor, or even a sufficient --- but not a necessary --- condition for the non-existence of these supposedly superior taxes. 
	\item
		Progressivity in tax, especially, may hinge upon an enlightened, ``new understanding of tax'' \citep{McCaffery2005}, as the historic mainstay of redistribution --- the \gls{PIT} --- withers away in the face of its inherent contradictions \citep{McCafferyHines2010}.
		People may understand tax \emph{systematically} different from how they would under (more) ideal speech, which may cause them to choose less effectively progressive taxes than they otherwise might, or otherwise inadvertently act even against their material self-interest.
		Additionally, the difference between raw and enlightened thinking about tax may, in itself, depend on the socio-economic status of citizens.
		Taken together, such socially structured and structuring differences in understanding tax may contribute to further, or perpetuate existing social inequality, even under nominal political equality.
	\item
		The aggregative and representative institutions of liberal-pluralist democracy may be partly outmatched by the vexing complexity \citep{Merton-1968-aa} and tightly concentrated interest \citep{Olson-1971-aa} of late market economies, such as in taxation, violating both their input and output legitimacy \citep{Scharpf1997} as well as basic democratic prescriptions for ``enlightened understanding'' \citep{Dahl-1989-aa}.
		Empirical political sociology bears out such popular dissatisfaction, not with democracy it self, but with incumbents, enamored by special interest, and inattentive to the public good \citep{NyeJr.1997,Norris2011,PutnamPharr-2000-aa}.
		If such a ``post-democratic'' constellation \citep{Crouch2004} were to plague the polity in its thinking about tax, it may well affect other policy areas too.
\end{enumerate}

\section{State of the Field}
This positive aspect of this thesis sits at the intersection of two fields of empirical research: 
\begin{enumerate}
	\item Practical experiments with different deliberative fora on varying policy areas,
	\item Experimental and survey research about popular understandings of tax.
\end{enumerate}

In the following, I briefly report key findings from these two fields and identify avenues for further research.

\subsection{Empirical Deliberation}

Deliberative fora have been tried on a number of policy areas and places, including renewable energy in Texas (USA) \citep{LehrGuild-2003-aa}, participatory budgeting in Porto Allegre \citep{CoelhoPozzono-2005-aa} (Brasil), nanotechnology \citep{Powell2008}, electoral reform in British Columbia (Canada) \citep{WarrenPearse-2008-aa}, policing and education in Chicago (Illinois) \citep{Fung-2003-ab}, and city planning in Philadelphia (Pennsylvannia) \citep{Sokoloff2005}. 
They also vary in their institutional design \citep[reviewed in][]{Fung-2003-ac}.
These include several small-n, multi-day designs, such as Citizen Juries \citep{SmithWales-2000-aa} or Planning Cells \citep{Dienel-1999-aa}, where a few randomly selected citizens hear expert witness and deliberate amongst themselves for a couple of days, all facilitated by trained, non-expert moderators and prepared by a balanced briefing book to draft a Citizen's Report or similar consensus recommendation.
The otherwise similar Consensus Conferences \citep{Einsiedel2000} are often (much) longer, feature extended learning periods and are typically held on highly technical issues
\footnote{
	Consensus Conferences were first developed and hosted in Denmark by the Technology Assessment Board to collect informed popular input for regulating science and technology.
}.
Deliberation has also been tried in large-n, shorter designs with a more quantitative methodology in \gls{DP}s \citep{FishkinFarrar-2005-aa}, where hundreds of randomly selected citizens receive issue books, deliberate in small, moderated groups and plenary sessions, hear expert witnesses and --- instead of reaching a consensus --- cast a secret vote or fill out a questionnaire, after the typically daylong deliberation concludes. 
Often, these survey results are compared to answers provided by participants before the event, and/or to a control group of non-participants, allowing quasi-experimental
\footnote{
	Because participants can always drop out at will and will know when they are in the treatment condition, treatment cannot be strictly randomly assigned.
	}
within-subject or between-subject statistical inferences. 
There is quite limited experience with long, large-n designs, including --- to my knowledge --- only the \glsfirst{CA} \citep{WarrenPearse-2008-aa} that spanned several months, and included several hundred participants who were offered stipends and accommodation, participated in extended learning sessions, attended public hearings and ultimately agreed to recommend a new electoral system (a \gls{STV}) for the province \citep{Citizen-2004-aa}.

%include tables from google doc.

Across these different designs and purposes, deliberative fora have been shown to improve knowledge of the subject matter in question, to change --- but not bifurcate --- beliefs \citep{Fishkin2009}, to yield more orderly and orthogonal preferences structures \citep{Farrar2003} and to boost a sense of political efficacy or mutual trust even among disempowered citizens \citep{Karpowitz2009}.

These encouraging findings on the capacity of deliberation to strengthen democratic rule notwithstanding, much of deliberative practice may be criticized for often focusing on the immediate, tangible and remaining ``confined to the realm of neighborhood and locale'' \citep[17]{FungWright-2001-aa}. 
This includes not only the overtly ``human-scale'' democratic efforts \citep[759]{Boggs-1997-aa} of participatory budgeting \citep{Sousa-Santos-1998-aa}, city planning \citep{Sokoloff2005} or public stewardship of forests \citep{Cheng2005}, but also the much-lauded \glspl{DP}, which have, even when they were about clearly large-scale issues such as the \gls{EU} \citep{Fishkin2009} or renewable energy use \citep{LehrGuild-2003-aa} mostly shied away from the very structured choices and remote abstractions which policy must engage. 
These \glspl{DP} have, for example, demonstrated that more people prefer \gls{EU} enlargement and more people would be willing to pay a premium for renewables \citep[157]{Fishkin2009}, but did \emph{not} ask --- and probably not deliberate on, either --- whether the \gls{EU} was an \gls{OCA}, and if not, what fiscal complements might have to be implemented \citep[for example,][]{Mundell1961}, or by which process \emph{competing} renewable energies should be chosen, especially if they depend on economies of scale and network effects \citep[for example, ][]{Krugman-1990-aa}.
Clearly, these are only two haphazardly selected examples of abstractions (\gls{OCA}) and choices (infant industry regulation) relevant to these topics, but equally clearly, someone, somewhere in the democratic process has to make these calls. %fishkin quote in the above is K2010
Some of these more technocratic considerations may be rightly relegated to experts, but for others --- including the abstractions of tax --- this may not be possible or desirable as discussed in the above.

To show its stripes --- or limits (!) --- deliberative democracy must be tried on such highly structured choices and abstract issues, too.
So far, with the notable exception of some Danish (small-n) Consensus Conferences and the (large-n) \gls{CA} \citep[on its complexity,][]{Blais2008}, such applications are lacking in empirical experiments with deliberation. 
If deliberative practice continues to blank out governing abstractions and structured choices --- the very stuff of the modern world and its political rule --- it risks moving ``in a defensive and insular direction, laying bare a process of conservative retreat beneath a facile rhetoric of grassroots activism'' \citep[759]{Boggs-1997-aa}.

%taxation is that case

\subsection{Misunderstanding Taxation}

Survey and experimental researchers have investigated how people misunderstand taxation and its abstractions.
\cite{McCafferyBaron2003}, tracking the heuristics and biases research program \citep{KahnemanEtAl1982} suggest that voters fall short of \citeauthor{VonNeumannMorgenstern1944}-rationality in their choice of tax: they incorrectly believe that taxation has a trivial, ``flypaper'' incidence \citep{McCafferyBaron2004b}, they favor (indirect) taxes not labeled or not visible, they fail to aggregate different taxes --- especially the proportional or regressive \gls{Payroll} and \gls{VAT} --- confuse progression in absolute and percentage terms and inconsistently prefer bonuses over penalties. 
Worried, \cite{McCafferyBaron2004} suspect that politicians may exploit these and other misunderstandings and that current tax systems may be suboptimal as a result of voter irrationality.
They also suggest that people may overestimate effective progression, and that tax regimes may, as a result, be less redistributive than voters intend and believe them to be \citep{McCafferyBaron2004}.

In a similar vein, \cite{SausgruberTyran2011} show that people mistake the nominal incidence of a tax with its effective incidence on the buyers and sellers in a taxed market transaction
\footnote{
	By definition, taxes in a market economy --- a pleonasm --- must \emph{always} be levied as some potential market exchange occurs.
	This is straightforward for income and consumption taxes, where the earning and spending transactions are taxed, but also holds for wealth taxes and even other instruments of expropriation (such as inflation).
	Taxation always requires some real (or imputed) market valuation of the given tax base, which in turn necessitates an actual or hypothetical transaction between market participants.}. %add reference to liquidity effects
In their laboratory experiment, participant buyers frequently tax sellers more than themselves, forgetting that --- under perfect competition, especially factor rent and price flexibility --- buyers and sellers will share any given tax on their transaction irrespective of the nominal incidence \footnote{%add reference to perfect competition conditions
	This according to the \gls{Tax LSE}}.
Under the resulting ``tax-shifting bias'' \citep{SausgruberTyran2011}, or, equivalently ``flypaper theory of taxation'' \citep{McCafferyBaron2003}, participant buyers often choose suboptimally high taxes on sellers, which they end up paying (partly) themselves.
\citeauthor{SausgruberTyran2011} furthermore find that participants are less likely to tax-shift if they are given accurate information on tax burdens exogenously and if they have experienced its income-reducing effects in the laboratory marketplace. 
Troublingly for deliberative democrats, they also find that (albeit minimally defined) initial deliberation does not reduce tax-shifting, but that it ``makes initially held opinions more extreme rather than correct'' \citep[164]{SausgruberTyran2011}, corroborating concerns for (deliberative) group polarization \citep{Sunstein1991}.

These are rigorously researched and disconcerting findings that have not received nearly the attention they deserve from welfare state scholars and other political economists. 
These and other misunderstandings of tax may constitute partial, possibly sufficient --- but not necessary --- causes for changing tax regimes and constricted, or retrenched welfare states.
	%note the absence of arbitrage mechanisms

These are also --- much like the sponsoring heuristics and biases program --- no definitive, comprehensive list of misunderstandings, but an evolving, loosely connected set of relatively low-level cognitive effects.
More research is needed: 
\begin{itemize}
	\item how do people (mis)understand some of the broader and more complex issues bearing on taxation, such as economic growth, savings rates or government and market failures? 
	\item how --- if at all --- do the suggested low-level cognitive heuristics and biases filter up to the very structured choices of base, schedule and timing of taxation asked of the \gls{OECD} polities?
\end{itemize}

As these pressing questions are addressed to develop a coherent account of social change from misunderstanding taxation, research
designs as those by \citeauthor{McCafferyBaron2004} or \citeauthor{SausgruberTyran2011} may well face two kinds of limits:
\begin{enumerate}
	\item The complexity of higher-order considerations may overwhelm both designs. 
	
	Laboratory models in behavioral economics as those by \citeauthor{SausgruberTyran2011} have the distinct advantage of directly testing the misunderstood causal effect in question (in this case, \gls{Tax LSE}), and rendering participants with an immediate experience of the effect (here, lower incomes).
	As the causal effects in question become more complex --- for example, liquidity or employment decisions under different taxes --- laboratory models face ever harsher tradeoffs between internal and external validity \citep[for a review and dissenting opinion, see][]{Jimenez-Buedo2010}.
	Trivially, economy-wide choices --- for example, between a \gls{PIT} and \gls{PCT} --- are very hard to model in the laboratory; if they were any easier, economics would be a non-issue.
	
	The within-subject designs of cognitive psychology as those by \citeauthor{McCafferyBaron2004} also offer great internal validity when the heuristics and biases are low-level.
	Participants rate their agreement with, or the truth, of several statements they are presented with.
	Experimenters vary the context or wording of substantively identical statements to identify contradictions or mental shortcuts within the answers of individual participants.
	\citeauthor{McCafferyBaron2004} provide supposedly ``de-biasing'' information as within-subject treatments.
	The problem with higher-level concerns --- say, whether to tax income or consumption --- is that equivalent de-biasing treatments would eventually need to be an introductory economics course, covering at least the circular flow in the economy, some macroeconomics of aggregate demand and the Haig-Simons identity of income.
	As the required syllabus grows, it not only becomes impractical for participants with limited time and patience, but it also ignites a combinatorial explosion of necessary treatments to figure out just which (of supposedly many) insights from the de-biasing did the trick.
	
	\item More problematically yet for the deliberative-minded researcher, both designs cannot problematize, let alone resolve, any \emph{remaining} substantive disagreements over taxation and its economic abstractions.
	Both behavioral economics and cognitive psychology stipulate that there is \emph{a} (\gls{vNM}-)rationality, and merely chronicle whatever human deviations from it they find.
	Again as with the heuristics and biases research program in general, researchers and deciders must be aware of these deviations from rationality in tax, but these deviations do not add up to a theory, much less an account of social change in taxation or democracy.
	We know that \citeauthor{VonNeumannMorgenstern1944} say one (internally consistent) thing, and other homo sapiens mostly something else, but we learn little about the conditions of this divergence, let alone ways to reconcile it.
	
	Things get even thornier for a heuristics and biases research program of tax, when higher-level concerns are introduced and --- as they likely will --- turn out to be controversial. 
	\citeauthor{McCafferyBaron2004} or \citeauthor{SausgruberTyran2011} posit expert rationality against their participants, and, whenever they differ, find their participants at fault.
	This works well enough for disentangling percentage and absolute tax burdens, and tolerably for demonstrating a tax-shifting bias\footnote{
	Tellingly, \citeauthor{SausgruberTyran2011} already take pains to remind readers that \gls{Tax LSE} only holds when factor rents and prices are fully flexible --- a condition that is as controversial as is it is difficult to verify, even after 60 years of macroeconomic debate about little else \citep{Wapshott2011}.},
but leaves the researcher empty-handed when faced with genuine disagreement \emph{between} experts, or challenges as the legitimacy of expertise.
	To be sure, there \emph{may} well be \emph{the} \gls{vNM}-rational, human-nature-compatible optimal taxation, that renders all other proposals regrettable misunderstandings --- but this hypothesized agreement and its communicative conditions must be tested, not axiomatized.
	The heuristics and biases research program can inspire a sociological account of misunderstood taxation, but it cannot bring it to its conclusion.
	When faced with inevitably controversial higher-level issues such as the choice between a \gls{PIT} and a \gls{PCT}, both cognitive psychology and behavioral economics will smack of expertocracy and remain fruitless because they allow no recourse to a second-order theory \citep[125]{GutmannThompson-2004-aa} --- such as Habermasian deliberation --- to resolve disagreement.	 
\end{enumerate}

\section{Project Description}
%project description
%universe, cases, variables
%temporal, spatial, social limits

The deliberative experiments on taxation proposed here builds on these two strands of research as much as it probes its limits:

\begin{enumerate}
\item 
	A deliberative forum on taxation, including its structured choices and abstractions, adds another --- yet rare --- experiment on a technocratic and complex policy issue to the empirical record.
	From this perspective, deliberation is the research topic, and taxation is a \emph{case} to test the limits and capacities of this prescription for political {rule}
	\footnote{
		Taxation is a good case, because --- as Franziska Deutsch succinctly noted in 2011 --- it affects everyone, but most people know little about it.
	}.
	
\item Lastly, deliberating taxation will provide data to better understand possible misunderstandings of tax and, thereby, clarify remaining controversies over it.
	From this perspective, optimal taxation and popular deviations therefrom are the research topic, and deliberation provides the \emph{method} to investigate, or even resolve or clarify these differences in a normatively attractive manner \citep{Rawls-1971,Habermas-1984}.
\end{enumerate}

%add visualization of the above and below paragraph?

These two research topics all fold into the aforementioned research question: how will people understand taxation differently, if they participate in a deliberative process \emph{closer} to the ideals of communicative action than the status quo of representative democracy?

That process, for the purposes of this dissertation, is a \gls{DP}, the proclaimed ``gold standard'' of deliberation \citep{Mansbridge2010}, illustrated in \autoref{fig:deliberative-poll-method(short)}.

A random --- or at least, diverse --- sample of about 20 citizens from the Bremen area will be invited to participate in a daylong deliberation on basic choices of schedule and base of taxation, as summarized in \autoref{tab:tax-overview}.

Prior to the event, participants will receive an accessibly written, substantively balanced briefing book, outlining important alternatives, abstractions and desiderata of tax design.
The briefing book will be vetted by a diverse panel of experts.

Upon arrival, participants will be greeted by the convener and asked to fill out a questionnaire, including simply worded questions about their desired tax base and schedule
\footnote{
	Ideally participants will also be asked to fill out the same questionnaire \emph{before} they receive the briefing book.
	Alternatively, a control group of non-participants may be asked to fill out the questionnaire before and after the deliberation.
}.

For example, some items might read:
\begin{quote}
	\emph{``How much, on a scale from 1--5 do you agree with the following statements?
		\begin{enumerate}
			\item People should pay taxes as a percentage of what they earn, both from their work and their investments, including such things as rents, dividends and interest.
			\item People should pay taxes as a percentage of what they consume, or ``use up'', including such items as food, private cars or haircuts.
			They should not be taxed on things they buy as investments or that do not get used up quickly, including such items as a house, a computer for work or a share in a company.
			\item People should pay taxes as a percentage of what they own subtracted from their debts, including their houses (minus the mortgage), their cash (minus their overdraft loan) or their companies.''
		\end{enumerate}
		}
\end{quote}
Taken together, these items in the questionnaires will ``map'' participants on the possible tax choices summarized in \autoref{tab:tax-overview}.
The questionnaire will also illicit some socio-economic variables, broader attitudes \emph{towards}, as well as knowledge \emph{about} taxation and a self-assessment of autonomy and competence on these matters.

After filling out the questionnaire, the topic of the deliberation will be introduced by two moderators, both of which are non-experts in the field.
The moderators again introduce a simplified version of \autoref{tab:tax-overview}, and walk participants through the combinations of base and schedule --- including presently used \gls{PIT}, \gls{VAT} and \gls{Payroll} --- but abstain from causal, or normative statements about these taxes.
The table will also be prominently displayed on premise for easy reference by participants.
Participants are then tasked to deliberate the basic choices of base and schedule.

Deliberation alternates throughout the day between small-group settings of 5--8 persons and larger plenary sessions. 
Moderators assist the participants and encourage the norms of deliberation --- including respect, equal participation and argumentative reciprocity --- but do not clarify or comment on substantive questions.
Moderators also make sure that no pressure for a collective decision is exerted.
Participants are invited to collect questions for a question and answer session held mid-day with a balanced panel of economic experts.
The deliberation concludes in a plenary session where participants are invited to reflect on the experience, as well as to identify further questions for deliberation.
Participants fill out a questionnaire again, including some of the questions from the initial questionnaire, as well as some additional items concentrating on the deliberative experience, especially the perceived autonomy, equality and competence of participants.
 are recompensed for their time and thanked by the convener.

The \gls{DP} will be held 2--5 times on different days, with different people. 

\begin{landscape}
 \begin{figure}[htbp]
    \begin{center}
	\includegraphics[width=1\linewidth]{tax-overview}  
	\caption{Incidence and Redistribution of Modern Taxes}
	\label{tab:tax-overview}
	\end{center}
	\input{./tex/tax-table-description.tex}
\end{figure}
\end{landscape} %this float seems too large

Crucially, a \gls{DP} does not ordinarily include any structured learning, aside from the (self-studied) brochure and (ad-hoc) expert  question and answer session.
This may be problematic for forming ``enlightened'' attitudes (\citeauthor{Fishkin2009}'s words) on highly abstract and structured choices as those offered by taxation. 
\citet[64]{Warren2008}, reporting on the \gls{CA}, notes that such learning phases are ``crucial to its ability to  render a decision, and indeed'' and that it \emph{can be done}: ``the decision was a learned and sophisticated one''; 
``Most members transformed themselves from lay citizens with little knowledge of electoral systems into experts over a period of several months (\ldots)'' \citeyearpar[64]{Warren2008}. %k1513
On the other hand, it will be difficult to guarantee the quality and substantive balance of any such learning phase, and it may easily degrade the deliberative forum into a classroom setting, in which --- in this case --- ``the economics teacher knows best'', bringing in its own structural inequalities of arguments.

This difficult tradeoff is put to an empirical test in a slight aberration from a standard \gls{DP}.
During some of the \gls{DP} runs, participants will be subjected to a short, structured learning intervention around midday, covering an isolated causal argument or conceptual clarification pertinent to tax choice.
These learning interventions, too, will be vetted by a diverse panel of economics experts.
The topics of the learning interventions are related to the later hypothesized set of misunderstandings, including, for example, the expected (non-)effects from taxing consumption on aggregate demand.

\section{Research Design}
%see below
%cases, people, selection
%methods of observation and/or inference

The \gls{DP}, in its original formulation, is a quasi-experimental design (see \autoref{fig:deliberative-poll-method(short)}) and invites quantitative, within-subjects analyses of questionnaire responses pre- and post-treatment.
In addition, the deliberations will be (audio-)visually recorded and other process data (such as notes) will be retained.
This data will be --- selectively --- transcribed and subjected to a qualitative analysis.

\begin{figure}[htbp]
    \centering
	\includegraphics[width=1\linewidth]{deliberative-poll-method(short)}  
	\caption{Method of the Deliberative Poll}
	\label{fig:deliberative-poll-method(short)}
\end{figure}

\subsection{Quantitative Data}

Both research questions, about 1) the remaining disagreements about tax and 2) the ability of deliberative democracy to cover abstract and structured issues can be rolled up into one set of hypotheses.

\citeauthor{Fishkin2009} suggests that deliberation increases knowledge, and changes preferences as a function of such knowledge gain. 
Similarly, I hypothesize that people will gain in knowledge about taxation, including resolving some systematic misunderstandings of concepts and causal relationships, some variants of which are already documented \citep{McCafferyBaron2003,Caplan2007}.
I furthermore hypothesize that \emph{as a function} of these knowledge gains, people will change their preferences on the desired base and schedule of taxation.
Additionally, deliberation will increase self-assessed autonomy and competence of participants, and they will find the presented alternatives of base and schedule more meaningful.
Learning interventions will further drive knowledge gain and improve self-assessments, but --- if it is sufficiently balanced --- \emph{will not} interact with attitude change.
%Both research questions \ref{itm:resolve-better-tax} and \ref{itm:prove-deliberative-democracy} can be rolled up in one set of hypotheses:

\begin{enumerate}
    \item \label{itm:think-different}
		If ordinary citizens are given the possibility to deliberate welfare and taxation, they will think differently about it.
		Specifically,
		\begin{enumerate}
		
			\item \label{itm:knowledge-gain}
				\emph{Knowledge Gain.}
				People will gain in knowledge.
				Specifically relevant for tax choice %(\autoref{fig:tax-with-misunderstandings}, p.~\pageref{fig:tax-with-misunderstandings}), %requires that table to be in here
				\begin{enumerate}
				
					\item \label{itm:bastard-keynesianism} 
						\emph{Bastard Keynesianism.}
						People will understand that an economy can have an arbitrary (sub-\citeauthor{Solow1956}) savings rate, and that --- equivalently --- if lowered slowly, aggregate \emph{consumption} will not depress aggregate \emph{demand}.
				
					\item \label{itm:real-myopia}
						\emph{Real Myopia.} %aka technocratic myopia
						People will understand that the future prosperity of an economy is, in part, determined by present \emph{net} investment, and that --- equivalently --- \emph{real}, not \emph{nominal} (for example, \gls{GDP}) indicators track the savings rate of an economy.
				
					\item \label{itm:bastard-hayekianism} %aka government vs markets
						\emph{Bastard Hayekianism.} %make naming consistent
						People will understand that an economy can adopt an arbitrary government quota (if, possibly, at a cost), and that --- equivalently --- not \emph{all} (but some) taxed resources are lost.
						Taxation is not exclusively or unconditionally a negative-sum proposition.
				
					\item \label{itm:flyper-theory} %aka tax-aversion
						\emph{Flypaper Theory.}
						People will understand that ontologically and empirically, only and always \emph{natural} persons bear the burdens of taxation, and that --- equivalently --- the flypaper theory of tax incidence is false.
						
						This hypothesis is similar to, but not identical to \citet{McCafferyBaron2003}'s findings of \emph{tax aversion}, by which people mistakenly prefer taxes not labeled as such and/or indirect taxes and a \emph{disaggregation effect}, by which people fail to integrate taxes on same bases.
						%might have to feature tax aversion separately, because that is already in the visualizatiuon
				
					\item \label{itm:ordoliberal-mess}
						\emph{Ordoliberal Mess.} %not hygiene, that's the opposite
						People will understand that Pigouvian and general-revenue taxes follow opposing logics, and that --- equivalently --- a Pigouvian tax \emph{should not} raise revenue.
						
						This hypothesis is similar to, but not identical to \citeauthor{McCafferyBaron2003}'s findings of a \emph{disaggregation effect}, by which people fail to integrate different taxes even on same bases.  %this is (\cite{McCaffery2003}: 19ff)
						%is this really right?
						
					\item \label{itm:false-omnipotence}%clarify labeling
						\emph{False Omnipotence.} %used to be negative some 
						People will understand that taxation will often (if not always) cause unintended consequences in markets, or that --- equivalently --- taxation can cause a negative-sum \glspl{DWL}. 
						Taxation is not exclusively or unconditionally a zero-sum proposition.
						
					\item \label{itm:false-omniscience}
						\emph{False Omniscience.}
						Relatedly, people will understand that taxable income and wealth (but not consumption) cannot be measured independent of market-pricing, or that --- equivalently --- any taxation based on fiat evaluations may have unintended consequences in markets.
				\end{enumerate}
			
			
%			\item \label{itm:preference-structuration}
%				\emph{Preference structuration.}
%				Partly as a function of the above knowledge gain, people will have better-structured preferences over taxation.
				
%				Specifically, and related to the cyclical preferences \citep[334]{Farrar2010} and other aggregation failures of pluralist democracy \citep[for example,][]{Condorcet1785,Arrow1963},

%				\begin{enumerate}
					
%					\item \label{itm:vNM-consistent}
%						\emph{\glsfirst{vNM}-Consistency.}
%						\emph{Individually}, people will have ordinal preferences that more closely resemble \gls{vNM}-consistency.
				
%					\item \label{itm:single-peakedness}
%						\emph{Single Peakedness.}
%						\emph{In the aggregate}, people will have preferences that more closely resemble single-peakedness.
				
%					\item \label{itm:orthogonal-dimensions}
%						\emph{Orthogonal Dimensions.}
%						\emph{In the aggregate}, people will have preferences that more closely resemble orthogonal factors.
%				\end{enumerate}
				
			\item \label{itm:attitude-change}
				\emph{Attitude Change.}
				Partly as a function of the above effects, %(as in \autoref{fig:tax-with-misunderstandings}, p.~\pageref{fig:tax-with-misunderstandings}), %not working
				 people will rate a \glsfirst{PCT}, a \glsfirst{WT}, \glsfirst{LVT} and a \glsfirst{NIT} more highly, and a \glsfirst{PIT}, \gls{VAT} and \glsfirst{Payroll} as less desirable compared to their original preferences.
		\end{enumerate}
	
	\item \label{itm:meta-assessment}
		\emph{Meta Assessment}
			\begin{enumerate}
				\item \label{itm:autonomy}
					\emph{Autonomy.} People will consider themselves more effectively autonomous in discussing and choosing basic tax regimes.
				\item \label{itm:competence}
					\emph{Competence.} People will consider themselves more competent in discussing and choosing basic tax regimes.
					
				\item \label{itm:meaningful-choice}
					\emph{Meaningful Choice.}
				People will find the choices of tax base and schedule to be more meaningful to express their political autonomy.			
			\end{enumerate}	
	\item \label{itm:learning-interventions}
		\emph{Learning Interventions.} A learning interaction targeted at one of the above-listed knowledge gains will further that knowledge gain.

	\item \label{itm:interaction-effects}
		\emph{Interaction Effects.}
		The above effects will:
			\begin{enumerate}
				\item \label{itm:interact-equity}
					\emph{Not interact} with people's equity beliefs, or --- equivalently --- people will not change their thinking about tax as a function of their allocative preferences.
				\item \label{itm:interact-ses}
					\emph{Interact negatively} with people's socio-economic status, or --- equivalently --- people with greater socio-economic status will change their thinking about tax the least.
%				\item \label{itm:interact-cognitive-ability}
%					\emph{Interact negatively} with people's cognitive ability, or --- equivalently --- people with less cognitive ability will change their thinking about tax the least.
				\item \label{itm:interventions} 
					\emph{Intervention-interaction}
						\begin{enumerate}
							\item Of the above effects, meta assessments will \emph{interact positively} with learning interventions.
							\item Of the above effects, attitude changes will \emph{not interact} with learning interventions over and above the effect mediated by knowledge gain.			
						\end{enumerate}
			\end{enumerate}
\end{enumerate}
    
%\begin{landscape}
% \begin{figure}[htbp]
%    \begin{center}
%	\includegraphics[width=1\linewidth]{tax-with-misunderstandings}  
%	\caption[The Vector Field of Taxes with Misunderstandings]{The Vector Field of Taxes with Misunderstandings}
%	\label{fig:tax-with-misunderstandings}
%	\end{center}
%%	\input{./tex/tax-table-description.tex}
%\end{figure}
%\end{landscape}

\subsection{Qualitative Data}

Selected transcripts from the (video-)taped deliberations will be subjected to a discourse analysis.
The objective is two-fold.

First, to ascertain the quality of deliberation on a complex subject matter \emph{without} a learning phase, the arguments exchanged and concepts used will be compared to a set of widely identified disagreements and broadly shared abstractions amongst economists of public finance.
Of course, any difference between those two \emph{discourses} would not necessarily imply that either the \gls{DP} fell short of communicative rationality, nor would it negate any downstream knowledge gains or attitude change.
That would require that said expert arguments are subjected to a thorough deliberation, which this project does not allow.
Still, barring such further legitimization of expert knowledge, such differences in discourse may inspire further, and more narrowly formulated research questions about --- possibly less-than-rational --- public understandings of tax.

Secondly, discourse analysis of selected transcripts will allow me to hold deliberation to a higher, more than merely procedural standard.
By tracking individual exchanges, I can adopt --- as \citet[90]{GutmannThompson-2004-aa} advise --- a middle way of substantive and procedural standards and consider the reciprocity of arguments.
Ideally, deliberators should not merely balance their airtime, but they should exchange reasons that are mutually comprehensible \citep[3]{GutmannThompson-2004-aa}. %k177
More than merely intelligible, permissible arguments to reach deliberative agreement must raise validity \citep{Habermas-1984} and moral \citep{Rawls-1971} claims \emph{universally} acceptable to everyone, or, in \citeauthor{Fishkin2009}'s simpler words ``the considerations offered in favor of, or against, a proposal, candidate or policy [must] be answered in a substantive way by \emph{those who advocate a different position}'' \citep[55, emphasis added]{Fishkin2009}. %k550

A hypothetical exchange from deliberation taxation illustrates the point:

\begin{dialogue}

\speak{Foundationalist}
\footnote{
	Inspired by the Philosophy of Robert \cite{Nozick1974}.
} 
I believe that everyone should be entitled to incomes generated from uncoerced exchange with others.

\speak{Social Democrat}
\footnote{
	Not very inspired, and by no one in particular.
} 
Yeah, obviously we are not going to take away all of a person's income, they should be entitled to it, but maybe we'll take some of it because we need that for education.
That's not going to hurt them.

\speak{Georgist}
\footnote{
	Inspired by the Philosophy of Henry \cite{George1879}
} 
I agree there may be something inviolable about owning the value added in free exchange with others.
But, Foundationalist, do you think that should also extend to owning unimproved land and natural resources?
It seems to me that \emph{no one} has really created these values, and whoever reaps rents from them does so \emph{only} because of the coercive powers that be.
\end{dialogue}

Formally or procedurally speaking, both Social Democrat and Georgist refer to Foundationalist's argument and speak for roughly the same amount of time, yet their speech acts are of strikingly different quality.
Social Democrat, rather sloppily, reiterates the argument and proceeds to balance it with another good (education) and suggests a compromise.
We can see how the dialogue would proceed; Foundationalist --- presumably deontologically bound to inviolable property --- would refuse to trade off property against opportunity.
Talking past one another, Social Democrat and Foundationalist may not have exchanged, let alone agreed on universal validity claims.
By contrast, Georgist --- maybe provisionally --- embraces the foundationalist's argument and points to an unexpected implication \emph{in foundationalist} terms.
It is much harder to see how their dialogue would proceed; presumably, Foundationalist would not --- ex ante --- have been open to large-scale land reform or taxation (as Georgism implies), and may not much warm to the idea.
Still, she may find it necessary to either clarify her position further, or to concede Georgist's point, because his may just be the ``forcelessly forceful'' better argument.
Foundationalist and Georgist are grappling with the formulations and implications of what \emph{may} become a universally acceptable moral claim, and in so doing, have crept a little closer to communicative rationality.

A \citeauthor{Habermas-1984}ian standard of communicative rationality has already been applied to deliberation \citep{Ratner-2008-aa}, but not to transcribed process data. 
Applying it here in a discourse analysis will reveal communicative rationality --- or lack thereof --- in action, and allow me to track  more or less successful embodiments of this regulative ideal to their substantive conclusion in choice of tax base and schedule.

\section{Feasibility}
This project is theoretically, methodologically and organizationally demanding, but it can be completed in about 2 years.
In my prior work on the mixed economy and taxation, I have already distilled a compact set of abstractions and choices to be considered by deliberators.
Existing work provide a good basis for the briefing material, learning interventions, questionnaire development and qualitative analyses.
Happily, these analyses abstract away much of the political controversy over economic issues, clearly state remaining expert dissent and, rather than definitive answers, present a catalogue of questions, tradeoffs and clarifications.
Prior work and derived material should hopefully be roughly consensual amongst experts, and thoroughly balanced.
Still, additional vetting and advice from a panel of academic economists will be required, especially because my highly selective training is mostly auto-didactic.

Because I have no practical experience with deliberative formats, ensuring conditions for genuine communicative action will be hard: `the potential for abuse [by facilitators] is real, and crafting an appropriate conceptual and institutional response will be difficult'' \citep[25]{GutmannThompson-2004-aa}.
Together with the moderators the forum must be meticulously planned and discussed with experienced organizers.
Ideally, I should attend a deliberative experiment to gain some first-hand impressions.

Moreover, participant acquisition and sample attrition will likely pose problems, requiring early planning and concerted outreach to local organizations and leaders.

Lastly, data gathering and analysis may easily outmatch available resources.
I must filter data and focus the analysis on a manageable size.

\section{Payoff}
This is probably an unconventional, and possibly risky project, but it has promise, too.

If, given the right design, deliberative democracy can enable citizens to rule on complex issues, political scientists will have a very able, and attractive hypothetical to compare with, and deliberative experimenters should have more courage to venture out to more topics facing our sovereigns.
Not just as social scientists, but as citizens too, we must know whether the once historical achievement of aggregative democracy is now withering away under the assault of tightly concentrated special interest and obscuring complexity.
If it can show its stripes, deliberative democracy may well be our last, and also our best hope, to reveal the perils of pluralism, then to live up to our greater capacity for communicative action.

If, under a normatively more attractive democratic process, people were to resolve some misunderstandings about, and agree on different, but doable and desirable taxes, welfare state research and political economy would have to explain a much greater retrenchment and democratic failure.
Not only as social scientists, but as citizens, too, we must know that better tax we could agree on, and if it exists, what is keeping us from it.
Taxation, underneath it all, \emph{is} the social contract \citep{SchumpeterSwedberg-1942-aa}, and its vitality will determine the prospects for modern progress.

%
%\cite{Genschel2004} and Achim \cite{Kemmerling2009} in particular have published on the effects of economic liberalization on national tax regimes. Steffen \cite{Ganghof2006}, also losely associated with the CRC 597, has published extensively on the changes undergoing national income taxation under conditions of economic globalization.
%
%%McCaffery
%	%our tax system is a disgrace, and has been so for decades. The way we tax is complicated, inefficient, and unfair. Yet whenever elected officials in Washington actually try to do something about tax, they tinker at best. At worst, they make the system even more annoying. We need fundamental, comprehensive tax reform, not ad hoc tinkering. Two, there is a widening gap between the rich and the not-rich in this country. It may surprise many readers to learn that there is a deep connection between these two facts. Tax as it is today is a cause of the wealth gap. Tax as it could be tomorrow would narrow it. That'sRead more at location 50   • Delete this highlight
%	%Note: this is key. It's fucked up, but systematically so. Edit
%
%%Cite Eliasoph on why we need a big ass issue; not a small scale issue. You're leading to despondence. Quote early passages of Eliasoph. Also argue with Eliasoph "Charles said …" that by all means trying to CHANGE things, and only discuss things you can change, has drawbacks.
%
%%Note how the DP on energy -- which, at first, seems similar -- is actually not. It asked, among other things, whether people would be willing to pay more on their monthly utility bills for wind \citep[loc. 2013]{Fishkin2009}, but not more technical stuff than that.
%
%%Rosenberg 2007: Rosenberg has the radical inter-subjective formulation that I think I'm looking for:
%	%6: ``In this deliberative conception, equality and autonomy require each other. On the one hand, equality is a necessary precondition of autonomy. It is only in a cooperative exchange between equals that the self-expression and critical selfreflection required for the self-reflective construction of one’s understandings and interests is possible.
%
%%25: ``the potential for abuse [by facilitators] is real, and crafting an appropriate conceptual and institutional response will be difficult'' 
%
%%Mansbridge 2010
%	%   * points out that deliberative polling mostly does not include radical solutions, perspectives (that could be a problem)
%

%Any such reduction would violate the spirit of deliberation; citizens could not set their own agenda.
%Any such reduction may well fail to elucidate my hypothesized misunderstandings: I imagine “understanding tax” as a bucket, which, if leaky will drain to the lowest level. For example, if deliberators would discuss most aspects of taxation, but skip over the limits (!) of price controls, they may well opt for price controls in lieu of, or in addition to taxation.

%Meaningfully deliberating tax requires a lot of knowledge, and that knowledge is unlikely to spring merely from the act of deliberation, but it needs some input.

%\citep[101]{GutmannThompson-2004-aa} ``Reciprocity is to justice in political ethics what replication is to truth in scientific ethics. A finding of truth in science requires replicability, which calls for public demonstration. A finding of justice in political ethics requires reciprocity, which calls for public deliberation. Deliberation is not sufficient to establish justice, but deliberation at some point in history is necessary.''	